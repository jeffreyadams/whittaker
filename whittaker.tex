%edited starting 8/27/17 for re-submission to Duke
\documentclass[10pt,leqno]{article}
\usepackage{verbatim}
\usepackage{amssymb, amscd}
\usepackage{mathtools}
%\usepackage{amsrefs}
\usepackage{rotating}
\usepackage{xcolor}
\usepackage{amsmath}
\usepackage{mathrsfs}
\usepackage{enumerate}
\usepackage[colorlinks=true, linktoc=page, citecolor=blue, linkcolor=blue, urlcolor=blue]{hyperref}

%\usepackage{showkeys}
\usepackage{tabularx}
\setlength\extrarowheight{4pt}   %spacing in tables
\usepackage{theorem}
\usepackage[matrix,tips,frame,color,line,poly,curve]{xy}
\renewcommand{\labelenumi}{(\arabic{enumi})}
\newcommand\kappaarrow[2]{#1\overset\kappa\rightarrow#2}
\newtheorem{theorem}[equation]{Theorem}
\newtheorem{corollary}[equation]{Corollary}
\newtheorem{definition}[equation]{Definition}
\newtheorem{lemma}[equation]{Lemma}
\newtheorem{desideratum}[equation]{Desideratum}
\newtheorem{conjecture}[equation]{Conjecture}
\newtheorem{proposition}[equation]{Proposition}
\newtheorem{remark}[equation]{Remark}
{\theorembodyfont{\rmfamily}\newtheorem{theoremplain}[equation]{Theorem}
\newtheorem{remarkplain}[equation]{Remark}
\newtheorem{editorialremarkplain}[equation]{Editorial Remark}
\newtheorem{exampleplain}[equation]{Example}
\newtheorem{corollaryplain}[equation]{Corollary}
\newtheorem{mytable}[equation]{Table}
}

\renewcommand{\sec}[1]{\section{#1}
\renewcommand{\theequation}{\thesection.\arabic{equation}}
  \setcounter{equation}{0}}
\newcommand{\subsec}[1]{\subsection{#1}
\renewcommand{\theequation}{\thesubsection.\arabic{equation}}
  \setcounter{equation}{0}}

\newcommand{\subsubsec}[1]{\subsubsection{#1}
\renewcommand{\theequation}{\thesubsection.\arabic{equation}}
  \setcounter{equation}{0}}

% Danger, Will Robinson!
\def\danger{\begin{trivlist}\item[]\noindent%
\begingroup\hangindent=3pc\hangafter=-2%\clubpenalty=10000%
\def\par{\endgraf\endgroup}%
\hbox to0pt{\hskip-\hangindent\dbend\hfill}\ignorespaces}
\def\enddanger{\par\end{trivlist}}

\newcommand{\Gext}{\negthinspace\negthinspace\phantom{a}^\delta G}
\newcommand{\thetaG}{\negthinspace\negthinspace\phantom{a}^\theta
  G(\C)}
\newcommand{\thetaK}{\negthinspace\negthinspace\phantom{a}^\theta K(\C)}
\newcommand{\qed}{\hfill $\square$ \medskip}
\newenvironment{proof}[1][Proof]{\noindent\textbf{#1.} }{\qed}
\newcommand\exact[3]{1\rightarrow #1\rightarrow #2\rightarrow #3\rightarrow1}
\newcommand{\Aut}{\mathrm{Aut}}
\newcommand{\Inv}{\mathrm{Invol}}
\newcommand{\sgn}{\mathrm{sgn}}
\newcommand{\diag}{\mathrm{diag}}
\newcommand{\gr}{\mathrm{gr}}
\newcommand{\Out}{\mathrm{Out}}
\newcommand{\Int}{\mathrm{Int}}
\renewcommand{\int}{\mathrm{int}}
\newcommand{\Hom}{\mathrm{Hom}}
\newcommand{\kernel}{\mathrm{kernel}}
\newcommand{\Ad}{\mathrm{Ad}}
\newcommand{\ad}{\mathrm{ad}}
\newcommand{\zinv}{\mathrm{inv}}
\newcommand{\SRF}{\mathrm{SRF}}
\newcommand{\Gad}{G_\mathrm{ad}}
\newcommand{\Gsc}{G_\mathrm{sc}}
\newcommand{\Zsc}{Z_\mathrm{sc}}
\newcommand{\Ztor}{Z_\mathrm{tor}}
\newcommand{\Gbar}{\overline G}
\newcommand{\Kad}{K_\mathrm{ad}}
\newcommand{\Gal}{\mathrm{Gal}}
\newcommand{\Norm}{\mathrm{Norm}}
\newcommand{\Cent}{\mathrm{Cent}}
\newcommand{\Stab}{\mathrm{Stab}}
\newcommand{\I}{\mathcal I}
\renewcommand{\O}{\mathcal O}
\newcommand{\R}{\mathbb R}
\newcommand{\C}{\mathbb C}
\newcommand{\Z}{\mathbb Z}
\newcommand{\W}{\mathbb W}
\newcommand{\Ztwo}{\mathbb Z_2}
\newcommand{\N}{\mathcal N}
\newcommand{\Q}{\mathbb Q}
\newcommand{\E}{\mathbb E}
\newcommand{\G}{G}
\renewcommand{\H}{\mathbb H}
\newcommand{\h}{\mathfrak h}
\newcommand{\n}{\mathfrak n}
\renewcommand{\sl}{\mathfrak s\mathfrak l}
\renewcommand{\P}{\mathfrak p}
\renewcommand{\a}{\mathfrak a}
\newcommand{\zk}{\mathfrak z_\mathfrak k}
\newcommand{\A}{\mathbb A}
\newcommand{\K}{\mathcal K}
\newcommand{\B}{\mathcal B}
\renewcommand{\k}{\mathfrak k}
\newcommand{\spint}{\widetilde{Spin}}
\newcommand{\ch}[1]{#1^\vee}
\newcommand\sigmaqc{\sigma_{\text{qc}}}
\newcommand\thetaqc{\theta_{\text{qc}}}
\newcommand{\Fgal}{F_{\text{gal}}}
\newcommand{\Falg}{F_{\text{alg}}}
\newcommand{\cl}{\mathit{cl}}
\newcommand{\Lie}{\mathrm{Lie}}
\newcommand{\opp}{\text{-opp}}

\renewcommand{\t}{\mathfrak t}
\newcommand{\su}{\mathfrak s\mathfrak u}
\newcommand{\g}{\mathfrak g}
\newcommand\inv{^{-1}}
\newcommand\wh{\widehat}
\newcommand{\GL}{\text{GL}}
\newcommand{\SL}{\text{SL}}
\newcommand{\SO}{\text{SO}}
\newcommand{\SU}{\text{SU}}
\newcommand{\Spin}{\text{Spin}}
\newcommand{\chGGamma}{\phantom{a}^\vee G^\Gamma}
\newcommand{\GGamma}{G^\Gamma}
\newcommand{\s}{\mathfrak s}
\newcommand{\w}{\mathfrak w}
\newcommand{\AV}{\mathrm{AV}}
\newcommand{\WF}{\mathrm{WF}}
\newcommand{\AC}{\mathrm{AC}}
\newcommand{\AVann}{\mathrm{AV}_{\mathrm{ann}}}
\newcommand{\GK}{\mathrm{GK}}
\newcommand{\Op}{\O_p}
\newcommand{\Kostant}[1]{\mathcal{K}(#1)}

\begin{document}
\title{Whittaker Models for Real Groups}
\author{Jeffrey Adams \& Alexandre Afgoustidis}
\maketitle

Let $G$ be a connected complex reductive group, defined over $\R$, and consider the group $G(\R)$ of real points. 

Suppose $\pi$ is an irreducible representation of~$G(\R)$. Recall $\pi$ has a \emph{Whittaker model} when there exists a pair $(B(\R), \chi)$ where 
\begin{enumerate}[(i)]
\item $B$ is a Borel subgroup of~$G$ defined over~$\R$, with real points $B(\R)$,
\item $\eta$ is a character of the nilpotent radical~$N(\R)$ of~$B(\R)$, subject to the condition that~$\eta$ is nontrivial on each simple root space,
\end{enumerate}
such that~$\pi$ embeds into the induced representation $\mathrm{Ind}_{N(\R)}^{G(\R)}(\chi)$. Of course this can happen if and only if there exists a Borel subgroup $B \subset G$ satisfying~(i); by definition this condition means~$G$ is \emph{quasisplit}, which we assume from now on. 

We also say~$\pi$ is \emph{generic} when it has a Whittaker model. In that case, the pair $(B, \chi)$ is not uniquely determined by~$\chi$. However the $G(\R)$-conjugacy class of $(B(\R), \eta)$ is uniquely determined by the equivalence class of~$\pi$. We will call it the Whittaker datum attached to~$\pi$, denoted $\mathfrak{w}(\pi)$. More generally, by a  {\it Whittaker datum} we will mean a $G(\R)$-conjugacy class of pairs  $(B(\R),\eta)$ satisfying (i)-(ii). 

We may view $\mathfrak{w}(\pi)$ as a useful invariant of generic representations. For instance, suppose~$\pi$ is a discrete series representation of~$G(\R)$. Then $\mathfrak{w}(\pi)$ singles out~$\pi$ among the discrete series representations of~$G(\R)$ having  the same infinitesimal and central characters. In the language of the local Langlands correspondence \cite{Langlands_CIRRAC, Contragredient}, this means the Whittaker datum singles out~$\pi$ within its $L$-packet. 

More generally, Whittaker data play a subtle but key role in the local Langlands correspondence for reductive groups over arbitrary local fields \cite{Borel, Kaletha_Taibi}, including the nonarchimedean case. For instance, given a Whittaker datum~$\mathfrak{w}$, it is expected that every tempered $L$-packet contains precisely one $\mathfrak{w}$-generic representation, which can be used as a basepoint for the internal structure of the $L$-packet~\cite{Kaletha_genericity}.

Returning to real groups, there are of course other invariants that one may attach to an irreducible representation~$\pi$. We shall be concerned with two of them, defined using very different (and rather subtle) ideas. The first is the \emph{wavefront set}, which originally comes from microlocal analysis of the matrix coefficients. The second is the \emph{associated variety}, which is defined using algebro-geometric methods. Both consist of nilpotent elements of the dual of the Lie algebra of~$G$, and are defined as coadjoint orbits (of two different kinds). 

Now suppose~$\pi$ is a generic discrete series representation of~$G(\R)$. Then the Whittaker datum $\mathfrak{w}(\pi)$, the wavefront set $\WF(\pi)$ and the associated variety $\AV(\pi)$ basically contain the same information: given one of the three invariants, it is possible to reconstruct the other two. This is well known to the experts, but some aspects of this discussion are not  easy to extract from the literature, especially not if one wants to be explicit about how to pass back and forth between the three invariants. We have felt it worthwhile to give an account of the matter here. 

We shall do two things: 
\begin{enumerate}
\item Give an elementary proof that passage from~$\pi$ to $\mathfrak{w}(\pi)$, $\WF(\pi)$ and $\AV(\pi)$ defines bijections between generic discrete series in an $L$-packet for $G(\R)$, Whittaker data for $G(\R)$, and the appropriate sets of nilpotent orbits;
\item Given one of the invariants $\mathfrak{w}(\pi)$, $\WF(\pi)$ and~$\AV(\pi)$, explain in the simplest possible terms how to reconstruct the other two. 
\end{enumerate}
Most statements in this note were already known, and we have tried to give new and simplified proofs for as many of them as we could. In the cases where we had to call on more difficult statements from the literature, we tried to give convenient references. 

\section{Some invariants of irreducible $(\g, K)$-modules}

\footnote{AA 02-IV-2024: I didn't make any effort at polishing this in recent weeks; we should of course rewrite this section once the rest is stable.}

Suppose $\pi$ is an irreducible $(\g,K)$-module. Then we can associate to~$\pi$ several invariants, which depend only on the equivalence class of~$\pi$: the associated variety, the Gelfand---Kirillov dimension, and the wavefront set.  We do not give full definitions, but try to point to convenient places in the literature.\footnote{AA 20-II-2024: of course if we write this, we should maybe add a few basic references for the notions.}

\subsection{Some notation}

Let~$\sigma$ be the anti-holomorphic involution defining~$G(\R)$. 
Let $\g=\Lie(G), \g(\R)=\Lie(G(\R))$. Choose a Cartan involution $\theta$ for $G$. This means:
$\theta$ is an algebraic involution, commuting with~$\sigma$,
and $G(\R)^\theta$ is a maximal compact subgroup of~$G(\R)$. Set $K=G^\theta$, $K(\R)=G(\R)^\theta=K^\sigma$.
Let $\s$ be the $-1$ eigenspace of $\theta$ acting on~$\g$.

Let $\N$ be the set of nilpotent elements of $\g$.
This is the closure of the principal nilpotent orbit $\Op$.

 \subsection{The invariants}


Let $I_\pi$ be the annihlator of $\pi$ in the universal enveloping algebra.
Let $\AVann(\pi)$ be the associated variety of $I_\pi$. 
This is the closure of a single complex nilpotent orbit $\O$.

The associated variety of $\pi$, denoted $\AV(\pi)$, is the  the union of a set of
$K$-orbits on $\O\cap\s$.

The Gelfand---Kirillov dimension~$\GK(\pi)$ of~$\pi$ can be defined in several different ways~\cite{vogan-gelfand-kirillov}; but one of them is
$$
GK(\pi)=\dim(\AV(\pi))=\frac12\dim(\AVann(\pi)).
$$
(All dimensions are complex unless otherwise stated.)

For all of these facts see \cite{vogan_bowdoin}, in particular Corollary 4.7 and Theorem 8.4.

Another invariant of an irreducible $(\g, K)$-module~$\pi$ is the \emph{wave-front set} of $\pi$, written $\WF(\pi)$. It is the union of a (finite) set of nilpotent $G(\R)$ orbits in
$i\g(\R)^*=i\Hom_\R(\g(\R),\R)$. See \cite{howe_wave_front}, \cite{bv_local_structure}, \cite{HarrisHeOlafsson}.

\subsection{Large representations} Set $N=\dim(\N)=\dim(G)-\mathrm{rank}(G)$. The maximal Gelfand--Kirillov dimension of a representation is $N/2$.
We say $\pi$ is {\it large} if $\GK(\pi)=N/2$. See \cite[Section~6]{Vogan78}. 


\begin{lemma}
  \label{l:large}
  The following conditions are equivalent.
  \begin{enumerate}
    \item $\pi$ is large;
\item $\AVann(\pi)=\N=\overline{\Op}$;
\item $\dim(\AVann(\pi))=N$;
  \item $\dim(\AV(\pi))=N/2$;
\item $\GK(\pi)=N/2$;

\end{enumerate}
\end{lemma}
See \cite{vogan_bowdoin}.

\subsection{The invariants for large discrete series} 


The following uses rather deep results of representation theory.

\begin{proposition} \label{invariants_ds}
Let~$\pi$ be the $(\g, K)$-module for a large discrete series representation of~$G(\R)$.  
\begin{enumerate}[(a)]
\item There is a unique Whittaker datum~$\mathfrak{w}$ such that~$\pi$ is $\mathfrak{w}$-generic.
\item $\AV(\pi)$ is the closure of a single $K$-orbit on~$\mathcal{O}_p \cap \mathfrak{s}$.
\item  $\WF(\pi)$ is the closure of a single $G(\R)$-orbit on $\mathcal{O}_p \cap  i \g(\R)$.
\end{enumerate}
\end{proposition}

\begin{proof} For (a), by \cite{vogan-gelfand-kirillov} and \cite{kostant_whittaker}
a representation $\pi$ is large if and only if it admits a Whittaker model for some Whittaker datum,
and by \cite[Lemma 14.14]{abv} a  large discrete series representation admits a unique Whittaker datum.

Part~(b) is one of the main results of~\cite{vogan_bowdoin}: see \cite[Theorem ?]{vogan_bowdoin}.\footnote{AA 19-II-2024: I'd welcome some help in navigating~\cite{vogan_bowdoin}: the relevant places seem to be  Sections 4, 5 and 8, but I have a few questions.}

Part~(c) follows from~\cite{rossmann_limit_orbits}.\footnote{AA 14-II-2024: maybe a little more detail would be welcome, the Rossmann paper does not seem to state the result in these terms.}
\end{proof}



\section{The dictionary for large discrete series}


\subsection{Statement of the results}  



Given a large discrete series representation~$\pi$ we consider the invariants of~$\pi$ defined by Proposition~\ref{invariants_ds}: a Whittaker datum  $\mathfrak{w}(\pi)$, a $K$-orbit $\AV^\flat(\pi)$ in $\mathcal{O}_p \cap \mathfrak{s}$, and a $G(\R)$-orbit $\WF^\flat(\pi)$ in $(\mathcal{O}_p \cap  i \g(\R))$.

Here is our main statement.

\begin{theorem} \label{th:main} Suppose~$\Pi$ is an $L$-packet of discrete series for~$G(\R)$. 
The maps $\pi \mapsto \mathfrak{w}(\pi)$, $\pi \mapsto \AV^\flat(\pi)$ and $\pi\mapsto \WF^\flat(\pi)$ induce bijections between:
\begin{enumerate}
\item[(1)] The set~$\Pi_{\mathrm{large}}$ of large discrete series representations in~$\Pi$ ;
\item[(2)] The set of Whittaker data for $G(\R)$ ;
\item[(3)] The set $(\Op \cap \s)/K$ of~$K$-orbits on $\mathcal{O}_p \cap \s$.
\item[(4)] The set $(\Op \cap i\g(\R))/G(\R)$ of~$G(\R)$-orbits on $\mathcal{O}_p \cap  i \g(\R)$.
\end{enumerate}
\end{theorem}

Before embarking on the proof, let us point out that the maps in the theorem are defined using rather deep results of representation theory. However, we shall prove in Section~\ref{sec:explicit} that all bijections $(i) \leftrightarrow (j)$, for $i,j \in \{(1), \dots, (4)\}$, can be spelled out in elementary terms. Furthermore we shall prove, using elementary arguments, that (3) $\leftrightarrow$ (4) coincides with the Kostant--Sekiguchi correspondence.  



Let us now explain the strategy of the proof, which is mainly an exercise in putting together some references that are rather scattered in the literature. What we shall actually do is: introduce a finite group~$Q_\sigma(G)$, and point out that 
\begin{itemize}
\item[(i)] it acts simply transitively on the sets~(1)--(4), and 
\item[(ii)] the maps $\pi \mapsto \mathfrak{w}(\pi)$, $\pi \mapsto \AV^\flat(\pi)$ and $\pi\mapsto \WF^\flat(\pi)$  are equivariant for these actions.
\end{itemize}
The theorem follows immediately from these two observations.\footnote{AA 01-IV-2024 --- perhaps it would be reader-friendly to point out the basic fact on group actions that we are using here: suppose a group $Q$ acts on two sets $A, B$ and let $f\colon A \to B$ be an equivariant map. If $Q$ acts transitively on~$B$  then~$f$ is surjective, and if $Q$ acts transitively on~$A$ and freely on~$B$ then~$f$ is injective.  }

\subsection{The groups $Q_\sigma(G)$ and $Q_{\theta}(G)$}

\subsubsection{Definition of the two groups}Let $Z=Z(G)$ be the center of $G$, and let $\Gad$ be the complex reductive group~$G/Z$. The group $Z(\R)=Z^{\sigma}$ is equal to the center of $G(\R)$. The automorphism~$\sigma$ of~$G$ preserves $Z$, and descends to an automorphism~$\sigma_{\ad}$ of~$\Gad$. We set $\Gad(\R)=\Gad^{\sigma_{\ad}}$; we may view it as the set of inner automorphisms of~$G$ which are defined over~$\R$, and this contains the group $G(\R)_{\mathrm{ad}}=G(\R)/Z(\R)$ as a normal subgroup of finite index\footnote{AA 15-II-2024: is there a trivial way to explain why finite index?}.
Define 
$$
Q_\sigma(G)=\Gad(\R)/G(\R)_{\mathrm{ad}}.
$$
This finite group may be viewed as a group of outer automorphisms of~$G(\R)$. 

We shall need a variant of~$Q_{\sigma}(\R)$ in which we replace the real form~$\sigma$ by the Cartan involution~$\theta$. As above, since~$\theta$ is an automorphism of~$G$, it preserves~$Z$,  and induces an involution~$\theta_{\ad}$ of~$\Gad$. Recall $K=G^{\theta}$; consider the intersection $Z_K=Z \cap K$, and set  $K_{\ad}=K/Z_K$. The inclusion  $K \hookrightarrow G$ induces a map $K \to \Gad$, whose image is a subgroup of~$\Gad^{\theta_{\ad}}$ and whose kernel is~$Z_K$. Therefore it factors to an injective morphism $\iota\colon \Kad \hookrightarrow\Gad^{\theta_{\ad}} $, whose image is a normal subgroup of~$\Gad^{\theta_{\ad}}$. Using it we will identify $\Kad$ with a normal subgroup of $\Gad^{\theta_{\ad}}$. Define
\[ Q_\theta(G) = \Gad^{\theta_{\ad}}/\Kad.\]
Unlike $Q_{\sigma}(G)$, this group does not have an obvious action on~$G(\R)$. Nevertheless the groups $Q_{\sigma}(G)$ and $Q_{\theta}(G)$ are isomorphic. We will spell out an explicit isomorphism below, using group cohomology arguments.

\subsubsection{Interpretation in terms of group cohomology}

The groups $Q_{\sigma}(G)$ and $Q_{\theta}(G)$ are instances of the following construction: if~$\tau$ is an involutive automorphism of~$G$, then it preserves~$Z$ and induces an involution~$\tau_{\ad}$ of~$\Gad$; we may then set $Q_{\tau}(G)=\Gad^{\tau_{\ad}}/(G^{\tau})_{\ad}$, where $(G^{\tau})_{\ad}$ is the image of~$G^{\tau}$ in~$\Gad$. In this situation~$Q_{\tau}(G)$ has a simple interpretation in Galois cohomology, which we now recall. 

Let us first set up some notation. Let~$\Gamma$ be the Galois group~$\Z/2\Z$ of~$\R$. When~$A$ is a group and $\tau$ is an involutive automorphism of~$A$, we may consider the cohomology sets $H^0_\tau(\Gamma, A)$ and $H^1_\tau(\Gamma, A)$ attached to the action of~$\Gamma$ on~$G$ using~$\tau$.  These are pointed sets, and not groups in general if~$A$ isn't abelian. We may view $H^0_\tau(\Gamma, A)$ as the fixed-point-set $A^{\tau}$, and $H^1_\tau(\Gamma,A)$ as the quotient of $A^{-\tau} = \{ a \in A, a\tau(a)=1\}$ by the equivalence relation~$\sim$ generated by $a \sim x a \tau(x^{-1})$ for all $x \in G$. 


We now take~$A=G$, and continue to assume~$\tau$ is an involutive automorphism. Then the short exact sequence $1\rightarrow Z \rightarrow G \rightarrow \Gad\rightarrow 1$ 
gives rise to a ``long" exact sequence of pointed sets:
\begin{equation} \label{long_ptset}
1\rightarrow Z^\tau \rightarrow G^\tau \rightarrow \Gad^{\tau_{\ad}} \rightarrow H^1_\tau(\Gamma,Z)\rightarrow H^1_\tau(\Gamma,G)\rightarrow H^1_{\tau_\ad}(\Gamma,\Gad).
\end{equation}
By definition the connecting map  $ \Gad^{\tau_{\ad}} \rightarrow H^1_\tau(\Gamma,Z)$ sends $g \in \Gad^{\tau_{\ad}}$ to the equivalence class of $g \tau_{\ad}(g^{-1})$ in $H^1_\tau(\Gamma, Z)$, and the meaning of all the other maps is obvious. 
Going through the definitions we see that  two elements  $g_1, g_2$ have the same image if and only if $g_1 g_{2}\inv \in (G^\tau)_{\ad}$. 
Therefore the image of the connecting map is in canonical bijection with $Q_\tau(G)$. Since~\eqref{long_ptset} is exact we get a canonical bijection
\begin{equation} \label{interp_q_cohomology} Q_{\tau}(G) \simeq \ker(H^1_\tau(\Gamma,Z)\rightarrow H^1_\tau(\Gamma,G)).\end{equation}
The previous argument is in the category of pointed sets, and on the right-hand side $H^1_\tau(\Gamma,G))$  is not a group in general. Therefore, at first sight, the map~\eqref{interp_q_cohomology} is only a set-theoretic bijection. However, 
$H^1_\tau(\Gamma,Z)$ is a group since $Z$ is abelian; thus both sides of~\eqref{interp_q_cohomology} have canonical abelian group structures, and chasing definitions one easily sees that~\eqref{interp_q_cohomology} is actually an isomorphism of abelian groups.
 

\subsubsection{Isomorphism between $Q_{\sigma}(G)$ and $Q_{\theta}(G)$}

Using the group cohomology interpretation in the previous paragraph, we can construct a canonical  isomorphism between $Q_{\sigma}(G)$ and $Q_{\theta}(G)$.
For this recall $\theta\sigma = \sigma\theta$ and $G^{\sigma\theta}$ is a compact real form of~$G$.
Furthermore $\sigma$ and $\theta$ coincide on $G^{\sigma\theta}$,
therefore  $H_{\sigma}(\Gamma, G^{\sigma\theta})$ and  $H_{\theta}(\Gamma, G^{\sigma\theta})$ are the same set,
which we will denote by  $H_{\sigma\mid\theta}(\Gamma, G^{\sigma\theta})$.
Now the inclusion $G^{\sigma\theta} \hookrightarrow G$ induces canonical maps 
\begin{equation} \label{bij_cohom} H^{1}_{\sigma}(\Gamma, G) \longleftarrow H_{\sigma\mid\theta}(\Gamma, G^{\sigma\theta}) \longrightarrow H^{1}_{\theta}(\Gamma, G).\end{equation}
By \cite[Corollary~4.4 and Corollary 4.7]{galois}, both of these maps are \emph{bijections}.
Replacing~$G$ by $Z$ we get bijections $H^{1}_{\sigma}(\Gamma, Z) \leftarrow H_{\sigma\mid\theta}(\Gamma, Z^{\sigma\theta}) \rightarrow H^{1}_{\theta}(\Gamma, Z)$.
These fit with the bijections~\eqref{bij_cohom} into a commutative diagram
\[
\begin{CD}
H^{1}_{\sigma}(\Gamma, Z) @<<< H_{\sigma\mid\theta}(\Gamma, Z^{\sigma\theta})@>>> H^{1}_{\theta}(\Gamma, Z)
 \\
@V{\varphi_{\sigma}}VV @VVV @V{\varphi_{\theta}}VV  \\
H^{1}_{\sigma}(\Gamma, G) @<<< H_{\sigma\mid\theta}(\Gamma, G^{\sigma\theta})@>>> H^{1}_{\theta}(\Gamma, G)
\end{CD}
\]
where $\varphi_{\sigma}$ and $\varphi_{\theta}$ are induced by the inclusion $Z \hookrightarrow G$. Since the horizontal arrows are isomorphisms in the category of pointed sets, they induce a canonical bijection between the kernels of $\varphi_{\sigma}$ and $\varphi_{\theta}$; by~\eqref{interp_q_cohomology} this gives a canonical bijection 
\[ Q_{\sigma}(G) \simeq Q_{\theta}(G).\] 
As discussed after~\eqref{interp_q_cohomology}, this is actually an isomorphism of abelian groups. 




\subsection{Actions of $Q_\sigma(G)$ on representations and invariants}


\subsubsection*{On representations} 
Since~$Q_\sigma(G)$ can be viewed as a group of automorphisms of~$G(\R)$,  it has a canonical action on equivalence classes of representations of~$G(\R)$, and therefore on~$(\g, K)$-modules. This action preserves  the property of being large\footnote{AA 15-II-2024: should say something about this using the fact that it comes from automorphisms, either here or in the section on large reps}. It also preserves $L$-packets by~\cite[Lemma 6.18]{Contragredient}, and acts transitively on an~$L$-packet\footnote{AA 01-IV-2023: this point was missing, and I think details should be included. I think the transitivity is easy using  the description of discrete series L-packets in the Congragredient paper in terms of Harish-Chandra parameters, plus the fact that every element of~$Q_{\sigma}(G)$ has a representative in the Weyl group of a given elliptic maximal torus. The point about the action preserving $L$-packets also follows from this, I think; and perhaps it gives a better argument for the preservation of $L$-packets than [Contragredient, Lemma 6.18], which looks better suited to the $\theta$-version.}.

\subsubsection*{On~Whittaker data}  By definition of Whittaker data, the action of~$Q_{\sigma}(G)$ on $G(\R)$ induces a canonical action on the set of Whittaker data. By \cite[(14.15)]{abv} this action is simply transitive. 

It is an immediate consequence of the definitions that the map $\pi \mapsto \w(\pi)$, taking a large discrete series representation to the Whittaker datum for which it has a Whittaker model, is equivariant for the actions of $Q_{\sigma}(G)$ on (1)--(2). 


\subsubsection*{On real nilpotent orbits} 

To define the action of~$Q_{\sigma}(G)$ on~(4), first observe that~$\Gad$ acts on~$\g$ by the adjoint action. This action preserves~$\Op$, and its restriction to~$\Gad(\R)$ preserves~$\Op\cap i\g(\R)$. Since elements on a given~$G(\R)_{\mathrm{ad}}$-orbit are in the same~$G(\R)$-orbit, this induces an action of $Q_{\sigma}(G)=\Gad(\R)/G(\R)_{\mathrm{ad}}$ on~$(\Op\cap i\g(\R))/G(\R)$.

\begin{proposition}\label{prop:action_on_real_orbits}
\begin{enumerate} 
\item The action of~$Q_{\sigma}(G)$ on $(\Op\cap i\g(\R))/G(\R)$ is simply transitive.
\item The map $\pi \mapsto \WF(\pi)$ is $Q_{\sigma}(G)$-equivariant.
\end{enumerate}
\end{proposition}

\begin{proof}
The proof of~(1) is a standard argument in group cohomology. Suppose $\omega,\omega'$ are $G(\R)$ orbits in $\Op\cap i\g(\R)$,
and choose $X\in \omega,X'\in\omega'$. Then there exists $g\in G$ such that $\Ad(g)(X)=X'$. Since $X\in \g(\R)$ the condition $X'\in i\g(\R)$ 
is $g\inv \sigma(g)\in \Stab_G(X)$.  Since $X$ is principal $\Stab_G(X)=ZU$ where $U$ is a unipotent group. 
A standard fact is that $H^1_\sigma(\Gamma,U)=1$ \cite[Chap.~III, Proposition~6]{Serre_Galois}, and from this we see that, after multiplying $g$ on the right by an element of $u$, we can assume $g\inv \sigma(g)\in Z$. This is equivalent to: the image of $g$ in $\Gad$ is in $\Gad(\R)$. On the other hand $\omega=\omega'$ 
if and only if $X'=\Ad(g)X$ for some $g\in G(\R)$. This competes the proof of (1).

Part~(2) of the proposition is comes from general properties of the wavefront set in microlocal analysis, and from the fact that the action of $Q_{\sigma}(G)$ comes from automorphisms of~$G(\R)$. More precisely, if~$q$ is an element of~$Q_{\sigma}(G)$ and $\tilde{q}$ is a representative of~$q$ in $\Gad(\R)$, then the action of~$q$ on equivalence classes of $(\g, K)$-modules comes from the action of the automorphism $\mathrm{int}(\tilde{q})$ of~$G(\R)$ on $(\g, K)$-modules. Now the wavefront set of a distribution on a manifold satisfies general covariance properties under diffeomorphisms of the manifold: this follows from Hörmander's original definitions, see e.g. \cite[Section 2, p.~800]{HarrisHeOlafsson}. Applying this to the present situation, and going through the basics as in   \cite[Section~2]{HarrisHeOlafsson}, it follows that $\pi \mapsto \WF(\pi)$ is equivariant under $Q_{\sigma}(G)$.
\end{proof}

It may be interesting to point out a variant on the proof of~(1).\footnote{AA 01-IV-2024: well, is it interesting to keep it in the paper, or should we comment it out...? } Given $X \in \Op \cap i\g(\R)$, the map 
\begin{align} \label{bij_q_2} (\Omega_{X} \cap i\g(\R))/G(\R) & \to \kernel\left(H^1_\sigma(\Gamma,\mathrm{Stab}_G(X))\rightarrow H^1_\sigma(\Gamma,G)\right) \\ \mathrm{Ad}(g) \cdot X & \mapsto \text{class of $ g\sigma(g^{-1})$ in $H^1_\sigma(\Gamma,\mathrm{Stab}_G(X))$} \nonumber \end{align} 
is a bijection: see \cite[Lemma 5.2]{galois}. 
Using this we get bijections
$$
\begin{aligned}
(\Op\cap i\g(\R))/G(\R)&\simeq \ker(H^1_\sigma(\Gamma,\Stab_G(X))\rightarrow H^1_\sigma(\Gamma,G))\\
&=
\ker(H^1_\sigma(\Gamma,Z)\rightarrow H^1_\sigma(\Gamma,G))\\
&\simeq Q_{\sigma}(G)
\end{aligned}
$$
where the middle line uses the equality $H^1_\sigma(\Gamma,\Stab_G(X))=H^1_\sigma(\Gamma,Z)$, which comes from the fact that $\Stab_G(X)=ZU$  as explained in the proof. The resulting bijection  $(\Op\cap i\g(\R))/G(\R)\simeq Q_\sigma(\R)$ is equivariant (for the actions of $Q_\sigma(G)$ on $(\Op\cap i\g(\R))/G(\R)$ described above, and the action of  $Q_\sigma(G)$ on itself by multiplication), and this gives another proof of~(1). 

\subsection{Actions of $Q_\theta(G)$ on representations and invariants}

To discuss the map taking a $(\g, K)$-module to the associated variety, it seems better to use $Q_{\theta}(G)$ rather than~$Q_{\sigma}(G)$. 


 \subsubsection*{On $(\g, K)$-modules} 
 
 Recall $Q_{\theta}(\R)=\Gad^{\theta_{\ad}}/\Kad$. By definition $\Gad^{\theta_{\ad}}$ is contained in the normalizer $\mathrm{Norm}_G(K)$; this determines a canonical action of $\Gad^{\theta_{\ad}}$ on~$(\g, K)$-modules.
 Any element of~$\Kad$ takes a $(\g,K)$-module to an equivalent one, and this gives an action of $Q_{\theta}(\R)$ on equivalence classes of irreducible $(\g, K)$-modules.
 This preserves the property of being a discrete series, and of being large.
 Furthermore, by \cite[Lemma 6.18 and Remark 6.19]{Contragredient} the action preserves $L$-packets.
 Therefore $Q_{\theta}(\R)$ acts transitively on the large representations in a discrete series $L$-packet.
  
\subsubsection*{On nilpotent $K$-orbits.} 



Let us define an action of~$Q_{\theta}(G)$ on $(\Op \cap \s)/K$. First, the adjoint action of~$\Gad$ on~$\g$ preserves both $\Op$ and~$\s$. Second, the action of~$K=G^\theta$ on $\Op \cap \s$ is trivial on~$Z_K=Z^\theta$, and factors through an action of~$\Kad$. It is then easy to check that if two elements of $\Op \cap \s$ lie in on the same $\Kad$-orbit, then for every $g \in \Gad^{\theta_{\ad}}$, the elements  ${g} \cdot x$ and ${g} \cdot y$ lie on the same $\Kad$-orbit. This determines an action of~$Q_\theta(G)$ on $(\Op \cap \s)/K$, which is the one featured in the next statement.

\begin{proposition}\label{prop:action_on_K_orbits}
\begin{enumerate} 
\item The action of~$Q_{\theta}(G)$ on $(\Op \cap \s)/K$ is simply transitive.
\item The map $\pi \mapsto \AV(\pi)$ is $Q_{\theta}(G)$-equivariant.
\end{enumerate}
\end{proposition}

The proof of~(1) is exactly the same as that  to that of Proposition~\ref{prop:action_on_real_orbits}(1), replacing~$\sigma$ by the Cartan involution $\theta$.

As for~(2), it follows directly from the definition of the associated variety. This uses filtrations of the universal enveloping algebra of~$\g$ which are all invariant under $\Gad^{\theta_{\ad}}$: see for instance the Introduction of~\cite{vogan_bowdoin}. Inspecting the definitions in \emph{loc. cit.} it becomes clear that the map $\pi \mapsto \AV(\pi)$, taking a large discrete series $(\g,K)$-module  (or rather its equivalence class) to its associated variety, is equivariant under~$Q_{\theta}(G)$, proving~(2). 
\qed


\section{Explicit versions of the dictionary}\label{sec:explicit}

\subsection{Whittaker data to real orbits ((2) $\leftrightarrow$ (4))}

First we describe the set of Whittaker data more explicitly.
Let $B$ be a Borel subgroup of $G$. Let $N$ be the nilradical of $B$, and $\overline N$ the opposite
nilpotent subgroup. Let $\n,\overline\n$ be the Lie algebras of $N,\overline N$.
If $B$ is defined over $\R$ we consider $N(\R),\n(\R), \overline\n(\R)$, etc.
Let $\kappa(\,,\,)$  be the Killing form. 
For $X\in i\overline \n(\R)$ define a unitary character $\psi_X$ of $N(\R)$ by:
$$
\psi_X(e^Y)=e^{2\pi \kappa(X,Y)}\quad(Y\in \n(\R)).
$$
The map $X\rightarrow \psi_X$ is an isomorphism between $i\overline\n(\R)$ and the unitary characters of $N(\R)$.
It is easy to see $\psi_X$ is non-degenerate if and only if $X\in i\g(\R)$ is a principal nilpotent element.
In this case we write $(B(\R),X)$ for the Whittaker datum $(B(\R),\psi_X)$.

It is clear that $(B(\R),X)\mapsto G(\R)\cdot X$ is a bijection between Whittaker data and
$(\Op(\R)\cap i\g(\R))/G(\R)$. The main result of \cite{matumoto} (Theorem A) says:

\begin{lemma}\label{lem:matumoto}
The bijection (2)$\leftrightarrow$(4) takes the Whittaker datum $(B(\R),X)$ to
the principal orbit $G(\R)\cdot X$. 
 \end{lemma} 



\subsection{Large discrete series to $K$-orbits ((1) $\leftrightarrow$ (3))}

Suppose $\pi$ is a  discrete series representation. Choose a compact Cartan subgroup $T$, and let
$\lambda\in\t^*$ be the Harish-Chandra parameter of $\pi$. Let $\Delta$ be the set of roots of $\t$ in $\g$, 
let $\Delta^+(\lambda)=\{\alpha\mid \langle\lambda,\ch\alpha\rangle>0\}$,
and let $S(\lambda)\in\Delta^+(\lambda)$ be the simple roots.
For $\alpha\in \Delta$ let $\g_\alpha$ be the corresponding root space, and choose a non-zero element $X_\alpha\in\g_\alpha$ for each $\alpha$.

\begin{lemma}[(1)$\leftrightarrow$(3)]\label{l:pi_to_av}
Suppose $\pi$ is a large discrete series representation with Harish-Chandra parameter $\lambda\in\t^*$.
Set

\begin{equation}
  \label{e:Epi}
  E_\pi=\sum_{\alpha\in S(\lambda)}X_\alpha\in \N.
\end{equation}

Then $E_\pi\in\s$ is a regular nilpotent element, and
$$
\AV(\pi)=\overline{K\cdot E_\pi}.
$$
\end{lemma}

\begin{proof}
If $\pi$ is any discrete series representation, with Harish-Chandra parameter $\lambda$, let
$\n_{\lambda}=\sum_{\alpha\in\Delta^+(\lambda)}\g_\alpha$.
Then by \cite[Proposition 6.8]{vogan_irreducibility} $AV(\pi)=K\cdot(\n_\lambda\cap\s)$. 

Now assume $\pi$ is large. This implies $X_\alpha\in\s$ for all $\alpha\in S(\lambda)$, so $E_\pi\in \s$. 
By~\cite{kostant_tds} $E_\pi$ is a regular nilpotent element, so the closure of $K\cdot E_\pi$ in $\s$ is equal to $K\cdot(\n_\lambda\cap\s)$. 
\end{proof}




\subsection{Large discrete series to real orbits ((1) $\leftrightarrow$ (4))}


Suppose $\pi$ is a discrete series representation.
Choose a compact Cartan subgroup $T$, and let $\lambda\in\t^*$ be the Harish-Chandra parameter of $\pi$.
Identify $\lambda$ with an element $H_\pi\in i\t\subset i\g$ using the isomorphism $\t\simeq \t^*$ induced by the Killing form. 
Define
\begin{equation} \label{semisimple_orbit_HC} \mathcal{O}_{\pi}=\quad \text{$G(\R)$-orbit of the Harish-Chandra parameter $H_\pi$}.\end{equation}

Consider the \emph{asymptotic cone} of $\mathcal{O}_\pi$:
\[ \AC(\mathcal{O}_\pi) = \dots \]

\begin{lemma} \label{lem:WF_and_AC}
Let~$\pi$ be a discrete series representation of $G(\R)$. The wave-front set $\WF(\pi)$ is equal to the asymptotic cone $\AC(\mathcal{O}_\pi)$.
\end{lemma}

\subsection{Large discrete series to Whittaker data ((1) $\leftrightarrow$ (2))}


We recall a recent result of the first author (JDA) and Kaletha, which spells out the connection between large discrete series and Whittaker data using  {\it Kostant sections}. Let~$X$ be a regular nilpotent element of~$\g$, and suppose~$X$ fits into an $\SL(2)$-triple $(X, H, Y)$. The Kostant section~$\Kostant{X}$ is the affine subspace $X + \mathrm{Cent}_{\g}(Y)$ of~$\g$. It depends on the choice of $\SL(2)$-triple, but if $X \in \g(\R)$, then the $G(\R)$-conjugacy class of~$\Kostant{X}$ depends only on the $G(\R)$-conjugacy class of~$X$. 


\begin{proposition}[\cite{adams_kaletha}]\label{JeffTasho_criterion}
Suppose $\pi$ is a generic discrete series representation of~$G(\R)$. Suppose~$\pi$ be a Whittaker datum for~$G(\R)$, and let~$X \in i\g(\R)$ be a regular nilpotent element such that $\w = \w_X$. 
Then $\pi$ is $\w$-generic if and only if $\Kostant{X}$ meets $\mathcal{O}_\pi$. 
\end{proposition}


This confirms that  the corresponding statement  in the $p$-adic case \cite{debacker_reeder_generic, kaletha_epipelagic}
applies to the real case as well.


\subsection{Connection between ((3) $\leftrightarrow$ (4)) and the Kostant--Sekiguchi correspondence } 

Theorem~\ref{th:main} gives a bijection between the set $(\Op \cap \s)/K$ of principal nilpotent~$K$-orbits and the set $(\Op \cap i\g(\R))/G(\R)$ of principal nilpotent~$G(\R)$-orbits. 

It is well known that a natural bijection between nilpotent $K$-orbits and nilpotent $G(\R)$-orbits can be defined in more elementary terms. This is the Kostant--Sekiguchi correspondence~\cite{sekiguchi}. Furthermore, it is a deep theorem of Schmid and Vilonen (the main result of~\cite{SV1}) that the wavefront set and the associated variety are exchanged by this correspondence. (As a historical aside, the Kostant--Sekiguchi correspondence was discovered on the basis that the correspondence $\AV(\pi) \leftrightarrow \WF(\pi)$ might be expressed in elementary terms.)

Therefore, a particular case of Schmid and Vilonen's theorem says: 
\begin{proposition} \label{Sekiguchi_result} The bijection (3)--(4) is induced by the Kostant--Sekiguchi correspondence.
\end{proposition} 
We shall do two things here. First, we will quote a result of \cite{AVAV} which makes this case of the Kostant--Sekiguchi correspondence completely explicit. Second, we shall give a new and elementary proof of Proposition~\ref{Sekiguchi_result}: this avoids the use of Schmid and Vilonen's difficult results, and instead uses the results above (in particular Proposition~\ref{JeffTasho_criterion}) as the main ingredients. 


\subsubsection{The Sekiguchi correspondence for principal nilpotent orbits}\label{sec:concrete_sek}
\footnote{AA 2024-IV-15: should correct the formulas here.}
Let us spell out the special case of the Sekiguchi correspondence which is of interest here. 
Suppose $E_\R\in \Op\cap i\g(\R)$. Then we may choose an $SL(2)$-triple $(H_\R,E_\R,F_\R)$ with $F_\R$  contained in $i\g(\R)$,
which implies $H_\R\in \g(\R)$. 
After conjugating by~$G(\R)$ we may further assume $\theta(E_\R)=-F_\R$.
The Sekiguchi correspondence takes the $G(\R)$-orbit of~$E_\R$ to the $K$-orbit of
$$
E_\theta=\frac12(-iE_\R-iF_\R+H_\R)\in \N\cap \s.
$$

Computing the map in the other direction goes as follows.
Suppose $E_\theta\in \Op\cap\s$. Choose an $SL(2)$-triple  $(E_\theta H_\theta,F_\theta)$ with
$F_\theta\in\s$, which implies $H_\theta\in\k$. After conjugating by $K$ we may further assume $\sigma(E_\theta)=F_\theta$.
Then the Sekiguchi correspondence takes the $K$-orbit of $E_\theta$ to the $G(\R)$-orbit of 
\begin{equation}\label{def_E_R}
E_\R=\frac i2(E_\theta-F_\theta-H_\theta)\in\N\cap i\g(\R).
\end{equation}

\subsubsection{A simple proof of Proposition~\ref{Sekiguchi_result}}\label{sec:sekiguchi_proof}

Consider the principal nilpotent element~$E_\pi$ defined in~\eqref{e:Epi}. We define another regular nilpotent element  $E_\R$ by using $E_\pi$ as $E_\theta$ in~\eqref{def_E_R}. We know that the associated variety of $\pi$ is the closure of $K \cdot E_\pi$ (Lemma~\ref{l:pi_to_av}), so the image of $\AV(\pi)$ by the Sekiguchi correspondence is the closure of the $G(\R)$-orbit of $E_\R$. We want to check that $\WF(\pi)$ 

By Lemma~\ref{lem:matumoto} it is enough to check that $\pi$ is $\w$-generic for the Whittaker datum $\w$ corresponding to~$E_\R$. By Proposition~\ref{JeffTasho_criterion} this is the case if and only if the Kostant section $\Kostant{E_\R}$ meets $\mathcal{O}_\pi$. 

Now, we know that $\Kostant{E_\R}=E_\R+\mathrm{Cent}_\g(F_\R)$ contains $E_\R-F_\R = H_\theta$\footnote{AA 2024-IV-15: insert the correct version once we've checked the Sekiguchi formulas in the previous paragraph. }.
In general the orbits $\mathcal{O}_\pi$ and $G(\R) \cdot H_\theta$ are disjoint. 
Nevertheless $H_{\lambda}$ and $H_\theta=[E_\theta, F_\theta]$ are in the same Weyl chamber\footnote{AA 2024-IV-15: should probably explain a bit more why this is correct.}.
Therefore the result follows from:  

\begin{proposition}\label{prop:AC_chamber} Let $\mathcal{C} \subset \t$ be a large open Weyl chamber.
If $H$ and $H'$ are  points of~$\mathcal{C}$ and $\mathcal{O}, \mathcal{O}'$ are the   $G(\R)$-orbits of $H$ and $H'$, then $\AC(\mathcal{O})=\AC(\mathcal{O}')$.
\end{proposition}

We shall prove this in the rest of this section, based on the following lemma.

\begin{lemma}\label{lem:Kost_and_AC} Let $\mathcal{O}$ be a regular semisimple orbit and  let $X$ be a regular nilpotent element. Then $\Kostant{X}$ meets~$\mathcal{O}$ if and only if $X$ is in the asymptotic cone~$\AC(\mathcal{O})$. 
\end{lemma}
\begin{proof} See if we spell it  out (one direction is in Adams--Kaletha, but parts of the arguments are similar so it might be more convenient include both directions...). Maybe it's actually possible just refer to Fresse--Mehdi with pointers. \end{proof}

The connection between Lemma~\ref{lem:Kost_and_AC} and Proposition~\ref{prop:AC_chamber} is that Lemma~\ref{lem:Kost_and_AC}  implies the  following observation on the behavior of the asymptotic cone of regular semisimple orbits: 

\begin{lemma}\label{lem:AC_containment} Let~$\Omega$ be a regular nilpotent $G(\R)$-orbit, and let $\mathcal{E}_\Omega \subset \g$ be the union of all regular semisimple orbits~$\mathcal{O}$ such that $\AC(\mathcal{O})$ contains~$\Omega$. Then $\mathcal{E}_\Omega$ is open. 
\end{lemma}

\begin{proof}
Let $F$ be an element of~$\Omega$, and let $\mathcal{O}$  be a regular semisimple orbit such that $\AC(\mathcal{O})$ contains~$\Omega$.
Then $\Kostant{F}$ meets $\mathcal{O}$ by  Lemma~\ref{lem:Kost_and_AC}. 
Fix $X$ in $\mathcal{O} \cap \Kostant{F}$.
We have $X =F+Y$ where $Y$ belongs to the centralizer $\mathrm{Cent}_\g(F)$.
Since this centralizer is transverse to the $G(\R)$-orbits and has dimension $\dim(\g(\R))-\dim(\t(\R))$,
there exists an open neighborhood of~$X$ consisting of (regular semisimple) orbits  which meet $\Kostant(F)$.
Using Lemma~\ref{lem:Kost_and_AC} again, we see that all those orbits $\mathcal{O}'$ satisfy $F \in \AC(\mathcal{O}')$,
which implies that $\Omega = G(\R) \cdot F$ is contained in $\AC(\mathcal{O}')$.
This proves that $\mathcal{E}_\Omega$ contains an open neighborhood of any of its points, hence is open.   \end{proof}


Let us prove Proposition~\ref{prop:AC_chamber}. Given an element $H$ of $\mathcal{C}$ we shall write $\mathcal{O}_H$ for the $G(\R)$-orbit of~$H$. For any regular nilpotent orbit~$\Omega$ let $E^{\mathcal{C}}_\Omega$  denote the set of $H \in \mathcal{C}$ such that $\AC(\mathcal{O}_H)$ contains $\Omega$; by Lemma~\ref{lem:AC_containment}  we know that   $E^{\mathcal{C}}_\Omega$ is an open subset of~$\mathcal{C}$. 

Now let $\mathcal{S}$ be the unit sphere of $\t$ for the norm induced by the Killing form, and $\mathcal{S} \cap \mathcal{C}$ be the intersection of $\mathcal{S}$ with the given chamber. This is an open subset of $\mathcal{S}$. The previous arguments show that the the radial projection $\mathcal{S}_{\Omega}$ of  $E^{\mathcal{C}}_\Omega$ on $\mathcal{S}$ is an open subset of $\mathcal{S} \cap \mathcal{C}$. 

Given regular nilpotent orbits $\Omega$ and $\Omega'$, we deduce that $\mathcal{S}_\Omega \cap \mathcal{S}_{\Omega'}$ is an open subset of the $\mathcal{S} \cap \mathcal{C}$.

Now consider the set $\Sigma$ of all points on $\mathcal{S} \cap \mathcal{C}$ which arise as the radial projection of the Harish-Chandra parameter~$H$ of some large discrete series representation. Since the Harish-Chandra parameters for discrete series are (a translate of) an integral lattice, the set $\Sigma$ is dense in $\mathcal{S} \cap \mathcal{C}$. Therefore if the open set $\mathcal{S}_\Omega \cap \mathcal{S}_{\Omega'}$ is nonempty, there must exist at least one large discrete series representation~$\pi$ whose Harish-Chandra projects to $\mathcal{S}_\Omega \cap \mathcal{S}_{\Omega'}$. By Lemma~\ref{lem:WF_and_AC} this means $\Omega$ and $\Omega'$ must be contained in $\WF(\pi)$, which implies $\Omega = \Omega'$ since the wavefront set is the closure of a single orbit. 

We conclude that if $\Omega \neq \Omega'$, then $\mathcal{S}_\Omega$ and $\mathcal{S}_\Omega$ are disjoint, and therefore $\mathcal{E}^\t_{\Omega}$ and $\mathcal{E}^t_{\Omega'}$ are disjoint open subsets of~$\mathcal{C}$. Since $\mathcal{C}$ is connected and equal to the union of all subsets $\mathcal{E}^{\mathcal{C}}_{\Omega}$, the chamber $\mathcal{C}$ must be equal to a single $\mathcal{E}^\t_\Omega$. This concludes the proof of Proposition~\ref{prop:AC_chamber}, and therefore also of Proposition~\ref{Sekiguchi_result}.


\subsubsection{Where to put this, if we keep it?}


As we saw the correspondence $\pi \leftrightarrow \WF(\pi)$ can be described using the asymptotic cone of the semisimple orbit of the Harish-Chandra parameter. We used this in the proof of Proposition~\ref{Sekiguchi_result}. Now, the results of Sections~\ref{sec:concrete_sek} and~\ref{sec:sekiguchi_proof} provide another description of $\WF(\pi)$ from the Harish-Chandra parameter: we can compose the maps (1)$\leftrightarrow$(3)$\leftrightarrow$(4). This gives the following rather concrete conclusion.

\begin{lemma}
  Let $\pi$ be a large discrete series representation.
Choose a compact Cartan subgroup $T$ and let $\lambda\in \t^*$ be the Harish-Chandra parameter of $\pi$.
Define $E_\theta=E_\pi$ as in \eqref{e:Epi}, and choose an $SL(2)$-triple $(H_\theta,E_\theta,F_\theta)$ as usual.
After conjugating by $K$ we may assume $\sigma(E_\theta)=F_\theta$.  Then\footnote{AA 14-II-2023: no closure?}
$$
\WF(\pi)=G(\R)\cdot\frac i2(E_\theta-F_\theta-H_\theta)\in\N\cap i\g(\R)
$$
Conversely if $E_\R\in \Op\cap i\g(\R)$, choose an $SL(2)$-triple $(H_\R,E_\R,F_\R)$ which (after conjugating by $G(\R)$) satisfies $\theta(E_\R)=-F_\R$.
Let
$$
H_\theta=E_\R-F_\R.
$$
Then $H_\theta\in\k$ is a regular semisimple element.
Set $\t=\Cent_{\g}(H_\theta)$, and let
$$
\Delta^+=\{\alpha\in\Delta(T,G)\mid  \alpha(H_\theta)>0\}
$$
Then $\pi$ is the discrete series representation in $\Pi$ whose Harish-Chandra parameter is dominant for $\Delta^+$. 
\end{lemma}

\bibliographystyle{plain}
\bibliography{Refs}

\end{document}
%%% Local Variables: 
%%% mode: latex
%%% TeX-master: t
%%% End: s_{1}:0 -> -1 6


