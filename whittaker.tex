%edited starting 8/27/17 for re-submission to Duke
\documentclass[10pt,leqno]{article}
\usepackage{verbatim}
\usepackage{amssymb, amscd}
\usepackage{mathtools}
%\usepackage{amsrefs}
\usepackage{rotating}
\usepackage{xcolor}
\usepackage{amsmath}
\usepackage{mathrsfs}
\usepackage{enumerate}
\usepackage[colorlinks=true, linktoc=page, citecolor=blue, linkcolor=blue, urlcolor=blue]{hyperref}

%\usepackage{showkeys}
\usepackage{tabularx}
\setlength\extrarowheight{4pt}   %spacing in tables
\usepackage{theorem}
\usepackage[matrix,tips,frame,color,line,poly,curve]{xy}
\renewcommand{\labelenumi}{(\arabic{enumi})}
\newcommand\kappaarrow[2]{#1\overset\kappa\rightarrow#2}
\newtheorem{theorem}[equation]{Theorem}
\newtheorem{corollary}[equation]{Corollary}
\newtheorem{definition}[equation]{Definition}
\newtheorem{lemma}[equation]{Lemma}
\newtheorem{desideratum}[equation]{Desideratum}
\newtheorem{conjecture}[equation]{Conjecture}
\newtheorem{proposition}[equation]{Proposition}
\newtheorem{remark}[equation]{Remark}
{\theorembodyfont{\rmfamily}\newtheorem{theoremplain}[equation]{Theorem}
\newtheorem{remarkplain}[equation]{Remark}
\newtheorem{editorialremarkplain}[equation]{Editorial Remark}
\newtheorem{exampleplain}[equation]{Example}
\newtheorem{corollaryplain}[equation]{Corollary}
\newtheorem{mytable}[equation]{Table}
}

\renewcommand{\sec}[1]{\section{#1}
\renewcommand{\theequation}{\thesection.\arabic{equation}}
  \setcounter{equation}{0}}
\newcommand{\subsec}[1]{\subsection{#1}
\renewcommand{\theequation}{\thesubsection.\arabic{equation}}
  \setcounter{equation}{0}}

\newcommand{\subsubsec}[1]{\subsubsection{#1}
\renewcommand{\theequation}{\thesubsection.\arabic{equation}}
  \setcounter{equation}{0}}

% Danger, Will Robinson!
\def\danger{\begin{trivlist}\item[]\noindent%
\begingroup\hangindent=3pc\hangafter=-2%\clubpenalty=10000%
\def\par{\endgraf\endgroup}%
\hbox to0pt{\hskip-\hangindent\dbend\hfill}\ignorespaces}
\def\enddanger{\par\end{trivlist}}

\newcommand{\Gext}{\negthinspace\negthinspace\phantom{a}^\delta G}
\newcommand{\thetaG}{\negthinspace\negthinspace\phantom{a}^\theta
  G(\C)}
\newcommand{\thetaK}{\negthinspace\negthinspace\phantom{a}^\theta K(\C)}
\newcommand{\qed}{\hfill $\square$ \medskip}
\newenvironment{proof}[1][Proof]{\noindent\textbf{#1.} }{\qed}
\newcommand\exact[3]{1\rightarrow #1\rightarrow #2\rightarrow #3\rightarrow1}
\newcommand{\Aut}{\mathrm{Aut}}
\newcommand{\Inv}{\mathrm{Invol}}
\newcommand{\sgn}{\mathrm{sgn}}
\newcommand{\diag}{\mathrm{diag}}
\newcommand{\gr}{\mathrm{gr}}
\newcommand{\Out}{\mathrm{Out}}
\newcommand{\Int}{\mathrm{Int}}
\renewcommand{\int}{\mathrm{int}}
\newcommand{\Hom}{\mathrm{Hom}}
\newcommand{\kernel}{\mathrm{kernel}}
\newcommand{\Ad}{\mathrm{Ad}}
\newcommand{\ad}{\mathrm{ad}}
\newcommand{\zinv}{\mathrm{inv}}
\newcommand{\SRF}{\mathrm{SRF}}
\newcommand{\Gad}{G_\mathrm{ad}}
\newcommand{\Gsc}{G_\mathrm{sc}}
\newcommand{\Zsc}{Z_\mathrm{sc}}
\newcommand{\Ztor}{Z_\mathrm{tor}}
\newcommand{\Gbar}{\overline G}
\newcommand{\Kad}{K_\mathrm{ad}}
\newcommand{\Gal}{\mathrm{Gal}}
\newcommand{\Norm}{\mathrm{Norm}}
\newcommand{\Cent}{\mathrm{Cent}}
\newcommand{\Stab}{\mathrm{Stab}}
\newcommand{\I}{\mathcal I}
\newcommand{\mH}{\mathcal H}
\renewcommand{\O}{\mathcal O}
\newcommand{\R}{\mathbb R}
\newcommand{\C}{\mathbb C}
\newcommand{\Z}{\mathbb Z}
\newcommand{\W}{\mathbb W}
\newcommand{\Ztwo}{\mathbb Z_2}
\newcommand{\N}{\mathcal N}
\newcommand{\Q}{\mathbb Q}
\newcommand{\E}{\mathbb E}
\newcommand{\G}{G}
\renewcommand{\H}{\mathbb H}
\newcommand{\h}{\mathfrak h}
\newcommand{\n}{\mathfrak n}
\renewcommand{\sl}{\mathfrak s\mathfrak l}
\renewcommand{\P}{\mathfrak p}
\renewcommand{\a}{\mathfrak a}
\newcommand{\zk}{\mathfrak z_\mathfrak k}
\newcommand{\A}{\mathbb A}
\newcommand{\K}{\mathcal K}
\newcommand{\B}{\mathcal B}
\renewcommand{\k}{\mathfrak k}
\newcommand{\spint}{\widetilde{Spin}}
\newcommand{\ch}[1]{#1^\vee}
\newcommand\sigmaqc{\sigma_{\text{qc}}}
\newcommand\thetaqc{\theta_{\text{qc}}}
\newcommand{\Fgal}{F_{\text{gal}}}
\newcommand{\Falg}{F_{\text{alg}}}
\newcommand{\cl}{\mathit{cl}}
\newcommand{\Lie}{\mathrm{Lie}}
\newcommand{\opp}{\text{-opp}}

\renewcommand{\t}{\mathfrak t}
\newcommand{\su}{\mathfrak s\mathfrak u}
\newcommand{\g}{\mathfrak g}
\newcommand\inv{^{-1}}
\newcommand\wh{\widehat}
\newcommand{\GL}{\text{GL}}
\newcommand{\SL}{\text{SL}}
\newcommand{\SO}{\text{SO}}
\newcommand{\SU}{\text{SU}}
\newcommand{\Spin}{\text{Spin}}
\newcommand{\chGGamma}{\phantom{a}^\vee G^\Gamma}
\newcommand{\GGamma}{G^\Gamma}
\newcommand{\s}{\mathfrak s}
\newcommand{\w}{\mathfrak w}
\newcommand{\AV}{\mathrm{AV}}
\newcommand{\Wh}{\mathrm{Wh}}
\newcommand{\WF}{\mathrm{WF}}
\newcommand{\AC}{\mathrm{AC}}
\newcommand{\AVann}{\mathrm{AV}_{\mathrm{ann}}}
\newcommand{\GK}{\mathrm{GK}}
\newcommand{\Op}{\O_p}
\newcommand{\Kostant}[1]{\mathcal{K}(#1)}
\newcommand{\ECom}{\mathcal{E}^{\mathcal{C}}_\Omega}
\begin{document}
\title{Whittaker Models for Real Groups}
\author{Jeffrey Adams \& Alexandre Afgoustidis}
\maketitle

Let $G$ be a connected complex reductive group, defined over $\R$, and consider the group $G(\R)$ of real points. 

We consider pairs $(B,\eta)$ where 
\begin{enumerate}[(i)]
\item $B$ is a Borel subgroup of~$G$ defined over~$\R$, with real points $B(\R)$,
\item $\eta$ is a character of the nilpotent radical~$N(\R)$ of~$B(\R)$, subject to the condition that~$\eta$ is nontrivial on each simple root space.
\end{enumerate}
The existence of a Borel subgroup satisfying (i) is the definition of $G$ being quasisplit, which we assume holds from now on.

We recall what it means for a representation $\pi$ to have a $(B,\eta)$-Whittaker model in Section~\ref{sec:invariants}.
This condition only depends on the conjugacy class of the pair, 
so we  consider pairs $(B,\eta)$ modulo conjugation by $G(\R)$, and refer to such a conjugacy class as a 
as a {\it Whittaker datum}.
This is a small finite set.
If $\w$ is a Whittaker datum, it makes sense to talk about $\pi$ having a $\w$-Whittaker model. We let $\Wh(\pi)$  be the set of
Whittaker data $\w$ such that $\pi$ has a $\w$-Whittaker model. We say $\pi$ is {\it generic} if $\Wh(\pi)$ is non-empty.

The set $\Wh(\pi)$ is
a useful invariant of generic representations.
For instance if $\Pi$ is a tempered $L$-packet for $G(\R)$ and $\w$ is a Whittaker datum then $\Pi$ contains a unique representation $\pi$
having a $\w$-Whittaker model. This plays a role in describing the internal structure of $L$-packets.
If $\Pi$ is a discrete series $L$-packet then each generic representation in the packet has a unique
Whittaker model, so the generic representation in $\Pi$ are parametrized by their Whittaker models.
It is conjectured that these properties hold in the $p$-adic case as well, and this is known in many cases.

There are other invariants attached to an
irreducible representation $\pi$.  We will be concerned with two of
them, defined very differently. The first is the \emph{wavefront set} $\WF(\pi)$
which is an analytic invariant.
The second is the \emph{associated variety} $\AV(\pi)$, which is
defined  algebraically.
Both consist of nilpotent elements of the dual of the Lie algebra of~$G$, and are defined as
coadjoint orbits (of two different kinds).
The invariants $\Wh(\pi), \WF(\pi)$ and $\AV(\pi)$ are closely related to each other.

We now focus on the case of a generic discrete series representation
$\pi$ of $G(\R)$, so $\Wh(\pi)$ is a single Whittaker datum.  In this
case the invariants $\Wh(\pi), \WF(\pi)$ and $\AV(\pi)$ determine each
other. This is well known to experts, but not so easy to to extract
from the literature. The purpose
of this paper is to give a self-contained account of these matters.

More precisely, we will do two things:
\begin{enumerate}
\item Give an elementary proof that  passage from~$\pi$ to $\mathfrak{w}(\pi)$, $\WF(\pi)$ and $\AV(\pi)$ defines bijections between generic discrete series in an $L$-packet for $G(\R)$, Whittaker data for $G(\R)$, and the appropriate sets of nilpotent orbits;
\item Given one of the invariants $\Wh(\pi)$, $\WF(\pi)$ or $\AV(\pi)$, explain in the simplest possible terms how to explicitly reconstruct the other two.
\end{enumerate}
Many results in this note are well known. We have tried to give new and simplified proofs for as many of them as we could, and otherwise give convenient references.

\section{Some invariants of irreducible representations}

\subsection{Basic notation}

\subsubsection*{Involutions} 


Let~$\sigma$ be the anti-holomorphic involution defining~$G(\R)$.
Let $\g=\Lie(G), \g(\R)=\Lie(G(\R))$. 
Choose a Cartan involution $\theta$ for $G$. This means:
$\theta$ is an algebraic involution, commuting with~$\sigma$,
and $G(\R)^\theta$ is a maximal compact subgroup of~$G(\R)$. Set $K=G^\theta$, $K(\R)=G(\R)^\theta=K^\sigma$.
Let $\s$ be the $-1$ eigenspace of $\theta$ acting on~$\g$
(we use the same notation for $\sigma,\theta$ lifted to $\g$). 

\subsubsection*{Nilpotent orbits} 
\label{s:nilpotentorbits}
Let $\N$ be the set of nilpotent elements of $\g$. 

We need to pass to the dual Lie algebra $\g^*$. We follow  \cite[Section 2]{AVAV}.
Define $\N^*\subset \g^*$ to be the set of elements whose orbit is a cone.
We identify $\g$ and $\g^*$ using  a non-degenerate invariant bilinear form on $\g$. Given an element~$X$ in $\g$ we sometimes write $X^\ast$ for the corresponding element of~$\g^\ast$. 

The identification $\g \simeq \g^\ast$ restricts to a bijection $\N\simeq \N^*$. Although this correspondence depends on the choice of form,
the induced bijection on the level of $G$-orbits is canonical.

We define $\O_p$ to be the \emph{principal nilpotent orbit} in $\N^*$: this is the unique orbit whose closure is  $\N^*$.

We may assume the chosen bilinear form on~$\g$ is $\theta$-invariant. In this case the identification between $\g=\mathfrak k+\s$ and $\g^*=\mathfrak k^*\oplus \s^*$ determines a $K$-equivariant  (but non-canonical) isomorphism between $\s$ and $\s^\ast$. This induces
a canonical bijection between $K$-orbits on $\s$ and on $\s^*$.


\subsubsection*{Representations} 
We need to pass back and forth between representations of $G(\R)$ and $(\g,K)$-modules.
See \cite{greenbook} for details.

Suppose $\pi$ is an irreducible admissible representation of $G(\R)$  on a Hilbert space. 
The underlying
$(\g,K)$-module is on the space of $K$-finite vectors in
$V$; we will denote it by~$\pi_K$ in this paper. 

Conversely if $\pi$ is an irreducible $(\g,K)$-module it can be
realized as the $K$-finite vectors of a Hilbert space representation, which we
refer to as the Hilbert space globalization of $\pi$.

The notion of equivalence of $(\g, K)$-modules is defined in such a way that two equivalent $(\g, K)$-modules have equivalent Hilbert space globalizations, and that any two equivalent admissible representations of~$G$ have equivalent underlying $(\g, K)$-modules.

\subsection{The invariants}\label{sec:invariants}


\subsubsection*{Associated variety and Gelfand--Kirillov dimension}


Suppose $\pi$ is an irreducible $(\g,K)$-module.

The associated variety of $\pi$, denoted $\AV(\pi)$, is a closed, $K$-invariant set in~$\s^*$; so it is a finite union of
nilpotent $K$-orbits on $\s^*$. For the definition, and many properties, see \cite[Section 5]{vogan_bowdoin}. 


A slightly different invariant is the associated variety of the annihilator. Let $I_\pi$ be the annihlator of $\pi$ in the universal enveloping algebra.
Let $\AVann(\pi)\subset\N^*\subset \g^*$ be the associated variety of $I_\pi$.
This is the closure of a single complex nilpotent orbit $\O$.
See \cite[Section 1, Theorem 4.7]{vogan_bowdoin}.

The Gelfand---Kirillov dimension~$\GK(\pi)$ of~$\pi$ can be defined in several different ways~\cite{vogan-gelfand-kirillov}; but one of them is
$$
GK(\pi)=\dim(\AV(\pi))=\frac12\dim(\AVann(\pi)).
$$
(All dimensions are complex unless otherwise stated.)


We define the associated variety of a Hilbert space representation to be that of  its underlying $(\g,K)$-module.


\subsubsection*{Wavefront set}

Now suppose $\pi$ is an irreducible admissible representation of $G(\R)$ on a Hilbert space $\mH$. 
The \emph{wave-front
  set} $\WF(\pi)$ is an analytic invariant determined by the
singularities of the global character of $\pi$ near the identity. See \cite{howe_wave_front}, \cite{bv_local_structure},
\cite{HarrisHeOlafsson} for the definition. 
It is a closed $G(\R)$-invariant subset of the nilpotent cone in
$i\g(\R)^*$. As such it is the union of a finite number of nilpotent $G(\R)$-orbits.

We define the wave front set of a $(\g,K)$-module to be that of its 
Hilbert space globalization.

\subsubsection*{Whittaker models}

We consider the space $\mH^\infty$ of smooth vectors in $\mH$, and its dual $\mH^{-\infty}$. We say $\pi$ has a $(B,\eta)$-Whittaker model
if there exists $v\in \mH^{-\infty}$ such that $\pi(n)(v)=\eta(n)v$ for all $n$ in the nilradical $N(\R)$ of $B(\R)$.
See \cite{matumoto}. 

We write $\Wh(\pi)$ for the set of Whittaker data~$\w$ such that $\pi$ has a $\w$-Whittaker model.

We define the Whittaker data  of a $(\g,K)$-module to be that of its 
Hilbert space globalization.


\subsection{Large representations} Set $N=\dim(G)-\mathrm{rank}(G)$. This is the dimension of the nilpotent cone~$\N$. The maximal Gelfand--Kirillov dimension of a representation is $N/2$.
We say $\pi$ is {\it large} if $\GK(\pi)=N/2$. See \cite[Section~6]{vogan-gelfand-kirillov}. 


\begin{lemma}
  \label{l:large}
  The following conditions are equivalent.
  \begin{enumerate}
    \item $\pi$ is large;
\item $\AVann(\pi)=\N^*=\overline{\Op}$;
\item $\dim(\AVann(\pi))=N$;
  \item $\dim(\AV(\pi))=N/2$;
  \item $\GK(\pi)=N/2$;
    \item $\pi$ is generic.

\end{enumerate}
\end{lemma}
See \cite{vogan_bowdoin}, in particular Corollary 4.7 and Theorem 8.4.



\subsection{Discrete series and Harish-Chandra parameters} \label{sec:discrete_series}

Recall $G(\R)$ has relative discrete series representations if and only if it has a relatively compact Cartan subgroup, i.e. there exists a $\theta$-stable Cartan subgroup~$T$ of~$G$  such that~$T(\R) \cap G_d$ is  compact, where $G_d$ is the derived group of~$G$. From now on, we assume $G(\R)$ does have discrete series representation: we fix  a Cartan subgroup $T \subset G$ satisfying the conditions above, and let~$\mathfrak{t}$ denote the Lie algebra of~$T$, which is a Cartan subalgebra of~$\g$. 

Choose a  positive system~$\Delta^+$  for the roots of~$\t$ in~$\g$, and let~$\rho \in \mathfrak{t}^\ast$ be the half-sum of positive roots. Let~$\Lambda \subset \mathfrak{t}^\ast$ be the set of elements~$\lambda \in \mathfrak{t}^\ast$ which are \emph{regular} (i.e. the linear form $\lambda$ does not vanish on any coroot) and such that $\lambda-\rho$  is the differential of a character of~$T(\R)$. The action of the Weyl group~$W=W(\g, \t)$ preserves~$\Lambda$. Let~$W_K\subset W$ be the subgroup $\mathrm{Norm}_K(T)/T$. 


Then any discrete series representation of~$G(\R)$ has a \emph{Harish-Chandra parameter}, which is an element of~$\Lambda/W_K$.\footnote{AA 2024-VI-14: perhaps we should add that if $\pi$ is a discrete series representation, then the Harish-Chandra parameter and the central character are enough to specify~$\pi$ completely, and refer to \cite[Section 8]{AV1} and \cite{Contragredient} for full details on the parametrization.   }

Recall that the \emph{Weyl chambers} of~$\mathfrak{t}^\ast$ are the connected components of the set of regular elements. Each Weyl chamber specifies a positive root system for the roots of $\mathfrak{t}$ in~$\g$, and we say a Weyl chamber is \emph{large} if the corresponding simple roots are all noncompact. 

If~$\lambda \in \mathfrak{t}^\ast$ is a representative for the Harish-Chandra parameter of a discrete series representation $\pi$, then~$\pi$ is generic if and only if the Weyl chamber containing~$\lambda$ is large. See~\cite[Section 6]{vogan-gelfand-kirillov}.


\subsection{The invariants for generic discrete series} 

\begin{proposition} \label{invariants_ds}
  Suppose $\pi$ is a generic discrete series representation of $G(\R)$. Then:
\begin{enumerate}[(a)]
\item $\Wh(\pi)$ is a single Whittaker datum;
\item $\AV(\pi)$ is the closure of a single $K$-orbit on~$\mathcal{O}_p \cap \s^*$;
\item  $\WF(\pi)$ is the closure of a single $G(\R)$-orbit on $\mathcal{O}_p \cap  i \g(\R)^*$.
\end{enumerate}
\end{proposition}

\begin{proof} For (a), by \cite{vogan-gelfand-kirillov} and \cite{kostant_whittaker}
a representation $\pi$ is large if and only if it admits a Whittaker model for some Whittaker datum,
and by \cite[Lemma 14.14]{abv} a  large discrete series representation admits a unique Whittaker datum.

For (b) see \cite[Proposition A.9]{AV1}.

Part~(c) follows from~\cite[Lemma B]{rossmann_limit_orbits}. Let us explain how to go from there to  assertion~(c), using standard facts on the asymptotic cone recalled below (\S\,{}\ref{sec:asymp_cone}).

Let~$\mathfrak{t}^+$ be a Weyl chamber in~$\mathfrak{t}$ containing a representative for the Harish-Chandra parameter of~$\pi$. Write $\lim_{x \to 0(\mathfrak{t}^+)} G(\R)\cdot x$ for the set of all elements $y \in \g$ such that every $G$-invariant neighborhood of~$y$ contains all elements of~$\mathfrak{t}^+$ sufficiently close to zero. By Lemma B in~\cite{rossmann_limit_orbits}, if $\mathfrak{t}^+$ is a large Weyl chamber, then $\lim_{x \to 0(\mathfrak{t}^+)} G\cdot x$ is a single $G$-orbit.  Since this does happen when~$\pi$ is a generic discrete series representation, to prove assertion~(c)  it is enough to check the closure of $\lim_{x \to 0(\mathfrak{t}^+)} G\cdot x$ contains $\WF(\pi)$ in that case. Now, the equivalence between conditions  (L1) and (L2) on page 2 of~\cite{rossmann_limit_orbits} implies that the closure of $\lim_{x \to 0(\mathfrak{t}^+)} G\cdot x$ contains the asymptotic cone of the orbit of the Harish-Chandra parameter (see  Section~\ref{sec:asymp_cone} below). Therefore assertion~(c) follows from the fact that this asymptotic cone is equal to  $\WF(\pi)$  (Lemma~\ref{lem:WF_and_AC} below).


\end{proof}

Because of this result we abuse notation slightly as follows. If~$\pi$ is a generic discrete series representation of~$G(\R)$, we write $AV(\pi)=\O$ to indicate that $AV(\pi)$ is the closure of the $K$-orbit $\O\subset \s^*$.
Similarly we write $\WF(\pi)=\O$ to indicate that $\WF(\pi)$ is the closure of the $G(\R)$-orbit $\O\subset i\g(\R)^*$.



\section{The dictionary for generic discrete series}


\subsection{Statement of the dictionary}  



Given a generic discrete series representation~$\pi$ we consider the invariants of~$\pi$ defined by Proposition~\ref{invariants_ds}: a Whittaker datum  $\Wh(\pi)$, a $K$-orbit $\AV(\pi)$ in $\mathcal{O}_p \cap \s^*$, and a $G(\R)$-orbit $\WF(\pi)$ in $\mathcal{O}_p \cap  i \g(\R)^*$.


\begin{theorem} \label{th:main} Suppose~$\Pi$ is an $L$-packet of discrete series for~$G(\R)$. 
The maps $\pi \mapsto \Wh(\pi)$, $\pi \mapsto \AV(\pi)$ and $\pi\mapsto \WF(\pi)$ induce bijections between:
\begin{enumerate}
\item[(1)] The set~$\Pi_{\mathrm{gen}}$ of generic discrete series representations in~$\Pi$ ;
\item[(2)] The set of Whittaker data for $G(\R)$ ;
\item[(3)] The set of~$K$-orbits on $\mathcal{O}_p \cap \s^*$.
\item[(4)] The set of~$G(\R)$-orbits on $\mathcal{O}_p \cap  i \g(\R)^*$.
\end{enumerate}
In particular if $\pi$ is a large discrete series representation then each of these three invariants of $\pi$ determines the other two. 
\end{theorem}

Before embarking on the proof, let us point out that the maps in the Theorem are defined using rather deep results of representation theory. However, we shall show in Section~\ref{sec:explicit} that all bijections $(i) \leftrightarrow (j)$, for $i,j \in \{(1), \dots, (4)\}$, can be spelled out in elementary terms. Furthermore we shall prove, using elementary arguments, that (3) $\leftrightarrow$ (4) coincides with the Kostant--Sekiguchi correspondence.  



Let us now explain the strategy of the proof, which is mainly an exercise in putting together some references that are rather scattered in the literature. What we shall actually do is: introduce a finite group~$Q(G)$, and point out that 
\begin{itemize}
\item[(i)] it acts transitively on~(1), and simply transitively on (2)--(4);
\item[(ii)] the maps $\pi \mapsto \Wh(\pi)$, $\pi \mapsto \AV(\pi)$ and $\pi\mapsto \WF(\pi)$  are equivariant for these actions.
\end{itemize}
The theorem follows immediately from these two observations, and from the following basic fact on group actions.

\begin{lemma} Suppose a group $Q$ acts on two sets $A, B$ and let $f\colon A \to B$ be an equivariant map. 
\begin{itemize}
\item[(a)] If $Q$ acts transitively on~$B$  then~$f$ is surjective;
\item[(b)] If $Q$ acts transitively on~$A$ and freely on~$B$, then~$f$ is injective.
\end{itemize}
In particular, if~$Q$ acts transitively on~$A$ and simply transitively on~$B$, then~$f$ is a bijection and the action of~$Q$ on~$A$ is simply transitive.
\end{lemma}

\subsection{The groups $Q_\sigma(G)$ and $Q_{\theta}(G)$}

\subsubsection{Definition of the two groups, and canonical isomorphism}

Let $Z=Z(G)$ be the center of $G$, and let $\Gad$ be the complex reductive group~$G/Z$.
Then $Z(\R)=Z^\sigma$ is the center of $G(\R)$. We set $\G(\R)_\ad=G(\R)/Z(\R)$.
On the other hand  $\sigma$ factors to an automorphism  of $\Gad$, still denoted~$\sigma$, and we let  $\Gad(\R)=(\Gad)^\sigma$.
The projection $G\rightarrow \Gad$ restricts to a map $G^\sigma\rightarrow (G/Z)^\sigma$, with kernel $Z^\sigma$.
This induces a canonical injection $(G^\sigma)_\ad\hookrightarrow (\Gad)^\sigma$, and we define
$$
Q_\sigma(G)=(\Gad)^\sigma/(G^\sigma)_\ad=\Gad(\R)/\G(\R)_\ad.
$$
This finite group may be viewed as a group of outer automorphisms of~$G(\R)$.

A similar discussion applies to the involution $\theta$.
Then $Z^\theta$ is the center of $K=G^\theta$ and  $\theta$ factors to $\Gad$.
We define $(G^\theta)_\ad=G^\theta/Z^\theta$ and
$$
Q_\theta(G)=(\Gad)^\theta/(G^\theta)_\ad =(\Gad)^\theta/\Kad.
$$
This is a group of outer automorphisms of $K$. 

An important fact is that $Q_\sigma(G)$ and $Q_\theta(G)$ are canonically isomorphic. This is a generalization of the well known fact that
$G(\R)/G(\R)_0\simeq K/K_0$ where the subscript $0$ denotes identity component. 

\begin{lemma}\label{lem:iso_q}
\begin{itemize}
\item[(a)] 
Every element of $Q_\sigma(G)=(\Gad)^\sigma/(G^\sigma)_\ad$ has a representative in $(\Gad)^\sigma$ which is also in $(\Gad)^\theta$.
\item[(b)] Given $x \in Q_{\sigma}(G)$, and a representative $g\in (\Gad)^\sigma \cap (\Gad)^\theta$ as in~(a), the image of~$g$ in  $Q_\theta(G)$ depends only on~$x$, and not on  the choice of~$g$. The corresponding map $x \mapsto \varphi(x)$, from $Q_{\sigma}(G)$ to $Q_{\theta}(G)$, is a group isomorphism. 
\end{itemize}
\end{lemma}


The proof uses an interpretation of $Q_\sigma(G)$ and $Q_\theta(G)$ in terms of group cohomology, which we now spell out. 

\subsubsection{Interpretation in terms of group cohomology}

The groups $Q_{\sigma}(G)$ and $Q_{\theta}(G)$ are instances of the following construction: if~$\tau$ is an involutive automorphism of~$G$, then it preserves~$Z$ and induces an involution of~$\Gad$, still denoted $\tau$; we may then set $Q_{\tau}(G)=(\Gad)^{\tau}/(G^{\tau})_{\ad}$, where $(G^{\tau})_{\ad}$ is the image of~$G^{\tau}$ in~$\Gad$. In this situation~$Q_{\tau}(G)$ has a simple interpretation in Galois cohomology, which we now recall. 

Let us first set up some general notation. When~$A$ is a group and $\tau$ is an involutive automorphism of~$A$, consider the cohomology sets $H^0(\tau, A)$ and $H^1(\tau, A)$ attached to the action of $\Z/2\Z$ on~$A$ by~$\tau$. We shall view   $H^0(\tau, A)$ as the fixed-point-set $A^{\tau}$, and $H^1(\tau,A)$ as the quotient of $A^{-\tau} = \{ a \in A, \tau(a)=a^{-1}\}$ by the equivalence relation $a \sim x a \tau(x^{-1})$ for all $x \in A$. The set $H^1(\tau, A)$ has a distinguished point $1$, coming from the identity of~$A$; but in general it just a pointed set, and not a group if~$A$ is nonabelian. 


We now take~$A=G$, and continue to assume~$\tau$ is an involutive automorphism. Then the short exact sequence $1\rightarrow Z \rightarrow G \rightarrow \Gad\rightarrow 1$ 
gives rise to a ``long" exact sequence of pointed sets:
\begin{equation} \label{long_ptset}
1\rightarrow Z^\tau \rightarrow G^\tau \rightarrow \Gad^{\tau} \overset{\psi_\tau}{\longrightarrow} H^1(\tau,Z)\to H^1(\tau,G)\rightarrow H^1(\tau,\Gad).
\end{equation}
In~\eqref{long_ptset} the connecting map  $ \psi_\tau\colon \Gad^{\tau} \rightarrow H^1(\tau,Z)$ is defined as follows: if $\gamma \in \Gad^{\tau}$ is the coset $gZ$, then $g\tau(g^{-1}) \in Z^{-\tau}$, and the connecting map sends~$\gamma$ to the class of $g \tau(g^{-1})$ in $H^1(\tau, Z)$. The kernel of~$\psi_\tau$ is $(G^\tau)_{\ad}$. Since~\eqref{long_ptset} is exact we deduce that $\psi_\tau$ induces a canonical  bijection
\begin{equation} \label{interp_q_cohomology} Q_{\tau}(G) \to  \ker\left(H^1(\tau,Z)\overset{\varphi_\tau}{\longrightarrow} H^1(\tau,G)\right)\end{equation}
where $\varphi_\tau$ is induced by the inclusion $Z^{-\tau} \hookrightarrow G^{-\tau}$. 

%The previous argument is in the category of pointed sets, and on the right-hand side $H^1(\tau,G)$  is not a group in general. Therefore, at first sight, the map~\eqref{interp_q_cohomology} is only a set-theoretic bijection. However, 
%$H^1(\tau,Z)$ is a group since $Z$ is abelian; thus both sides of~\eqref{interp_q_cohomology} have canonical abelian group structures, and chasing definitions one easily sees that~\eqref{interp_q_cohomology} is actually an isomorphism of abelian groups.

%We can spell out the inverse of~\eqref{interp_q_cohomology} as follows. Suppose~$\zeta \in H^1(\tau, Z)$ is the equivalence class of  $z \in Z^{-\tau}$. By definition we have $\varphi_\tau(\zeta)=1$ if and only if there exists $g \in G$ such that $z = g \tau(g^{-1})$.  In that case $\tau(g) =z^{-1}g$, therefore the projection of~$g$ in $\Gad=G/Z$ is an element $\tilde{q}$ of~$\Gad^{\tau}$. It follows immediately from the definitions that the class of $\tilde{q}$ in $Q_{\tau}(G)=\Gad^\tau/(G^\tau)_\ad$ depends only on $\zeta$ and not on the choice of $g$ or $z$. If $q$ denotes this equivalence class, then the map $\zeta \to q$ is the inverse of~\eqref{interp_q_cohomology}. 
 

\subsubsection{Proof of Lemma~\ref{lem:iso_q}}

The proof of Lemma~\ref{lem:iso_q} uses Galois cohomology for the compact real form.  Recall $\theta\sigma = \sigma\theta$, and $G^{\sigma\theta}$ is a compact real form of~$G$. 
The involutions $\sigma$ and $\theta$ coincide on $G^{\sigma\theta}$,
therefore  $H^1(\sigma, G^{\sigma\theta})$ and  $H^1(\theta, G^{\sigma\theta})$ are the same set,
which we will denote by  $H(\ast, G^{\sigma\theta})$.
Now the inclusion $G^{\sigma\theta} \hookrightarrow G$ induces canonical maps 
\begin{equation} \label{bij_cohom} H^{1}(\sigma,  G) \longleftarrow H(\ast, G^{\sigma\theta}) \longrightarrow H^{1}(\theta,  G).\end{equation}
By \cite[Corollary~4.4 and Corollary 4.7]{galois}, both of these maps are \emph{bijections}.
Replacing~$G$ by $Z$ we get bijections $H^{1}(\sigma, Z) \leftarrow H(\ast, Z^{\sigma\theta}) \rightarrow H^{1}_{\theta}(\Gamma, Z)$.
These fit with the bijections~\eqref{bij_cohom} into a commutative diagram
\[
\begin{CD}
H^{1}(\sigma, Z) @<<< H^1(\ast, Z^{\sigma\theta})@>>> H^{1}(\theta, Z)
 \\
@V{\varphi_{\sigma}}VV @VVV @V{\varphi_{\theta}}VV  \\
H^{1}(\sigma, G) @<<< H^1(\ast, G^{\sigma\theta})@>>> H^{1}(\theta,  G)
\end{CD}
\]
where $\varphi_\sigma$, $\varphi_\theta$ are the maps in \eqref{interp_q_cohomology}.

Let us now prove part~(a) of Lemma~\ref{lem:iso_q}. Let $\gamma$ be an element of $(\Gad)^\sigma$; what we need to show is that there exists $\gamma_0 \in (G^\sigma)_{\ad}$ such that $\gamma \gamma_0^{-1}$ is $\theta$-invariant. 

Let $g$ be an element of $G$ such that $\gamma = gZ$; then $z=g \sigma(g^{-1})$ is an element of $Z^{-\sigma}$. Let $\zeta$ be its class in $H^{1}(\theta, Z)$. Let $\zeta_c$ be the inverse image of $\zeta$ under the bijection $H(\ast, Z^{\sigma\theta})\to H^{1}(\sigma, Z) $, and let $z_c\in (Z^{\sigma\theta})^{-\sigma}$ be a representative of $\zeta_c$. By definition there exists $z_0 \in Z$ such that $z = z_0 z_c \sigma(z_0^{-1})$. Since $\zeta_c$ is in the kernel of the map $H^1(\ast, Z^{\sigma\theta}) \to H^1(\ast, G^{\sigma\theta})$, there must exist  $k \in G^{\sigma\theta}$  such that $z_c = k \sigma(k^{-1})$. We see that $z = g \sigma(g)^{-1}$ is equal to $z_0 k \sigma((z_0k)^{-1})$, in other words 
\[ z_0^{-1} k^{-1} g \in G^{\sigma}.\]
Let $\gamma_0$ be the image of $z_0^{-1} k^{-1} g$ in $\Gad$. Then $\gamma_0 \in (G^{\sigma})_\ad$, and $\gamma_0$ is equal to the image of $k^{-1} g$ in $\Gad$. The element $\kappa=\gamma \gamma_0^{-1}$ of $\Gad$  is therefore $\sigma$-invariant, and since $\kappa=kZ$, we also have $(\theta\sigma)(\kappa)=\kappa$; we conclude that $\theta(\kappa) = \kappa$, q.e.d. 

Let us prove~(b). To show that when $g\in (\Gad)^\sigma \cap (\Gad)^\theta$ is a representative of $x \in Q_{\sigma}(G)$,  the image of~$g$ in  $Q_\theta(G)$ depends only on~$x$, it is enough to show that  $(G^\sigma)_\ad \cap (\Gad)^\theta \subset (G^\theta)_{\ad}$. This follows from \cite[Proposition 5.4]{galois}, applied to the action of~$G$ on~$\Gad$ and the obvious involutions. We conclude that taking $x \in Q_{\sigma}(G)$ to the image $\varphi(x)$ of $g$ in $Q_{\theta}(G)$ gives a well-defined map $\varphi\colon Q_{\sigma}(G)\to Q_{\theta}(G)$, which is clearly a group homomorphism. Exchanging the roles of $\sigma$ and $\theta$, we get a well-defined map $Q_{\theta}(G) \to Q_{\sigma}(G)$, which is the inverse of $\varphi$; therefore $\varphi$ is an isomorphism.

%where $\varphi_{\sigma}$ and $\varphi_{\theta}$ are induced by the inclusion $Z \hookrightarrow G$. Since the horizontal arrows are isomorphisms in the category of pointed sets, they induce a canonical bijection between the kernels of $\varphi_{\sigma}$ and $\varphi_{\theta}$. 
%
%; by~\eqref{interp_q_cohomology} this gives a canonical bijection 
%\[ Q_{\sigma}(G) \simeq Q_{\theta}(G).\]


\subsection{Actions of~$Q(G)$}


\subsubsection*{On representations} 



Note that $Q_\sigma(G)$ is a group of automorphisms of $G(\R)$. It therefore acts on representations of $G(\R)$. 
Similarly $Q_\theta(G)$ is a group of automorphisms of $(\g,K)$, and acts on $(\g,K)$-modules.\footnote{AA 30-IV-2024: shouldn't this be:  $\Gad^\sigma$ and $\Gad^\theta$  act on $G(\R)$-representations or $(\g,K)$-modules respectively, and this induces actions of $Q_{\sigma}(G)$ and $Q_{\theta}(G)$ on equivalence classes...? JDA this is the same thing}
These two actions are intertwined by the isomorphism of Lemma~\ref{lem:iso_q}:

\begin{lemma}\label{lem:action_q}
Suppose $\pi$ is the equivalence class of an irreducible Hilbert space representation of $G(\R)$. 
Suppose $q\in Q_\sigma(G)$. 
Then $(\pi^q)_K\simeq (\pi_K)^{\phi(g)}$.
\end{lemma}

\begin{proof}
Immediate from Lemma~\ref{lem:iso_q} and the definition of the actions.
\end{proof}


Because of Lemmas~\ref{lem:iso_q} and \ref{lem:action_q}, we write $Q(G)$ for the group $Q_\sigma(G)$ or $Q_\theta(G)$, depending on the situation.



\subsubsection*{On real nilpotent orbits} 

To define the action of~$Q_{\sigma}(G)$ on~(4), first observe that~$\Gad$ acts on~$\g$ by the adjoint action. This action preserves~$\Op$, and its restriction to~$\Gad(\R)$ preserves~$\Op\cap i\g(\R)^*$. Since elements on a given~$G(\R)_{\mathrm{ad}}$-orbit are in the same~$G(\R)$-orbit, this induces an action of $Q_{\sigma}(G)=\Gad(\R)/G(\R)_{\mathrm{ad}}$ on~$(\Op\cap i\g(\R)^*)/G(\R)$.

\begin{proposition}\label{prop:action_on_real_orbits}
\begin{enumerate} 
\item The action of~$Q_{\sigma}(G)$ on $(\Op\cap i\g(\R)^*)/G(\R)$ is simply transitive.
\item The map $\pi \mapsto \WF(\pi)$ is $Q_{\sigma}(G)$-equivariant.
\end{enumerate}
\end{proposition}

\begin{proof}
The proof of~(1) is a standard argument in group cohomology. Suppose $\omega,\omega'$ are $G(\R)$ orbits in $\Op\cap i\g(\R)^*$,
and choose $X\in \omega,X'\in\omega'$. Then there exists $g\in G$ such that $\Ad(g)(X)=X'$. Since $X\in \g(\R)^*$ the condition $X'\in i\g(\R)^*$ 
is $g\inv \sigma(g)\in \Stab_G(X)$.  Since $X$ is principal $\Stab_G(X)=ZU$ where $U$ is a unipotent group. 
A standard fact is that $H^1(\sigma, U)=1$ \cite[Chap.~III, Proposition~6]{Serre_Galois}, and from this we see that, after multiplying $g$ on the right by an element of $u$, we can assume $g\inv \sigma(g)\in Z$. This is equivalent to: the image of $g$ in $\Gad$ is in $\Gad(\R)$. On the other hand $\omega=\omega'$ 
if and only if $X'=\Ad(g)X$ for some $g\in G(\R)$. This competes the proof of (1).

Part~(2) of the proposition is comes from general properties of the wavefront set in microlocal analysis, and from the fact that the action of $Q_{\sigma}(G)$ comes from automorphisms of~$G(\R)$. More precisely, if~$q$ is an element of~$Q_{\sigma}(G)$ and $\tilde{q}$ is a representative of~$q$ in $\Gad(\R)$, then the action of~$q$ on equivalence classes of $(\g, K)$-modules comes from the action of the automorphism $\mathrm{int}(\tilde{q})$ of~$G(\R)$ on $(\g, K)$-modules. Now the wavefront set of a distribution on a manifold satisfies general covariance properties under diffeomorphisms of the manifold: this follows from Hörmander's original definitions, see e.g. \cite[Section 2, p.~800]{HarrisHeOlafsson}. Applying this to the present situation, and going through the basics as in   \cite[Section~2]{HarrisHeOlafsson}, it follows that $\pi \mapsto \WF(\pi)$ is equivariant under $Q_{\sigma}(G)$.
\end{proof}

%It may be interesting to point out a variant on the proof of~(1).\footnote{AA 01-IV-2024: well, is it interesting to keep it in the paper, or should we comment it out...? } Given $X \in \Op \cap i\g(\R)$, the map 
%\begin{align} \label{bij_q_2} (\Omega_{X} \cap i\g(\R))/G(\R) & \to \kernel\left(H^1_\sigma(\Gamma,\mathrm{Stab}_G(X))\rightarrow H^1_\sigma(\Gamma,G)\right) \\ \mathrm{Ad}(g) \cdot X & \mapsto \text{class of $ g\sigma(g^{-1})$ in $H^1_\sigma(\Gamma,\mathrm{Stab}_G(X))$} \nonumber \end{align} 
%is a bijection: see \cite[Lemma 5.2]{galois}. 
%Using this we get bijections
%$$
%\begin{aligned}
%(\Op\cap i\g(\R))/G(\R)&\simeq \ker(H^1_\sigma(\Gamma,\Stab_G(X))\rightarrow H^1_\sigma(\Gamma,G))\\
%&=
%\ker(H^1_\sigma(\Gamma,Z)\rightarrow H^1_\sigma(\Gamma,G))\\
%&\simeq Q_{\sigma}(G)
%\end{aligned}
%$$
%where the middle line uses the equality $H^1_\sigma(\Gamma,\Stab_G(X))=H^1_\sigma(\Gamma,Z)$, which comes from the fact that $\Stab_G(X)=ZU$  as explained in the proof. The resulting bijection  $(\Op\cap i\g(\R))/G(\R)\simeq Q_\sigma(\R)$ is equivariant (for the actions of $Q_\sigma(G)$ on $(\Op\cap i\g(\R))/G(\R)$ described above, and the action of  $Q_\sigma(G)$ on itself by multiplication), and this gives another proof of~(1). 

On the other hand $Q(G)$, realized as $Q_\theta(G)$ is a group of automorphisms of $(\g,K)$.
It induces actions on $\s^*$, $\Op$ and $\Op\cap\s^*$, and on $(\g,K)$-modules.

\begin{proposition}\label{prop:action_on_K_orbits}
\begin{enumerate} 
\item The action of~$Q(G)$ on $(\Op \cap \s^*)/K$ is simply transitive.
\item The map $\pi \mapsto \AV(\pi)$ is $Q(G)$-equivariant.
\end{enumerate}
\end{proposition}

The proof of~(1) is exactly the same as that  to that of Proposition~\ref{prop:action_on_real_orbits}(1), replacing~$\sigma$ by the Cartan involution $\theta$.

As for~(2), it follows directly from the definition of the associated variety. This uses filtrations of the universal enveloping algebra of~$\g$ which are all invariant under $\Gad^{\theta_{\ad}}$: see for instance the Introduction of~\cite{vogan_bowdoin}. Inspecting the definitions in \emph{loc. cit.} it becomes clear that the map $\pi \mapsto \AV(\pi)$, taking a large discrete series $(\g,K)$-module  (or rather its equivalence class) to its associated variety, is equivariant under~$Q_{\theta}(G)$, proving~(2). 
\qed


\section{Explicit versions of the dictionary}\label{sec:explicit}

\subsection{Whittaker data to real orbits ((2) $\leftrightarrow$ (4))}

First we describe the set of Whittaker data more explicitly.
Let $B$ be a Borel subgroup of $G$. Let $N$ be the nilradical of $B$, and $\overline N$ the opposite
nilpotent subgroup. Let $\n,\overline\n$ be the Lie algebras of $N,\overline N$.
If $B$ is defined over $\R$ we consider $N(\R),\n(\R), \overline\n(\R)$, etc.
For $X\in i\overline \n(\R)^*$ define a unitary character $\psi_X$ of $N(\R)$ by:
$$
\psi_X(e^Y)=e^{X(Y)}\quad(Y\in \n(\R)).
$$
The map $X\rightarrow \psi_X$ is an isomorphism between $i\overline\n(\R)^*$ and the unitary characters of $N(\R)$.
It is easy to see $\psi_X$ is non-degenerate if and only if $X\in i\g(\R)^*$ is a principal nilpotent element.
In this case we write $(B(\R),X)$ for the Whittaker datum $(B(\R),\psi_X)$.

It is clear that $(B(\R),X)\mapsto G(\R)\cdot X$ is a bijection between Whittaker data and
$(\Op(\R)\cap i\g(\R)^*)/G(\R)$. The main result of \cite{matumoto} (Theorem A) says:

\begin{lemma}\label{lem:matumoto}
The bijection (2)$\leftrightarrow$(4) takes the Whittaker datum $(B(\R),X)$ to
the principal orbit $G(\R)\cdot X$. 
 \end{lemma} 



\subsection{Large discrete series to $K$-orbits ((1) $\leftrightarrow$ (3))}

Suppose $\pi$ is a  discrete series representation. Let
$\lambda\in\t^*$ be a representative for the Harish-Chandra parameter of $\pi$ (Section~\ref{sec:discrete_series}). Let $\Delta$ be the set of roots of $\t$ in $\g$, 
let $\Delta^+(\lambda)=\{\alpha\mid \langle\lambda,\ch\alpha\rangle>0\}$,
and let $S(\lambda)\in\Delta^+(\lambda)$ be the simple roots.
For $\alpha\in \Delta$ let $\g_\alpha$ be the corresponding root space, and choose a non-zero element $X_\alpha\in\g_\alpha$ for each $\alpha$.

\begin{lemma}[(1)$\leftrightarrow$(3)]\label{l:pi_to_av}
Suppose $\pi$ is a large discrete series representation with Harish-Chandra parameter $\lambda\in\t^*$.
Set

\begin{equation}
  \label{e:Fpi}
  F_\pi=\sum_{\alpha\in S(\lambda)}X_{-\alpha}\in \N.
\end{equation}

Then $F_\pi\in\s$ is a regular nilpotent element. Let $F_\pi^\ast\in \N^*$ be the corresponding element (cf. Section \ref{s:nilpotentorbits}).
Then 
$$
\AV(\pi)=\overline{K\cdot F_\pi^\ast}.
$$
\end{lemma}

See
\cite[Propositions A.7 and A.9]{AV1}.


\subsection{Large discrete series to real orbits ((1) $\leftrightarrow$ (4))}\label{sec:asymp_cone}


Suppose $\pi$ is a discrete series representation. Let $\lambda\in\t^*$ be a representative of the Harish-Chandra parameter of $\pi$.
Define\footnote{AA 2024-VI-14: should perhaps decide if we fix a large chamber from the start and take the HC parameter to live there, or at least be careful about consistency in notation...}
\begin{equation} \label{semisimple_orbit_HC} \mathcal{O}_{\pi}=G(\R)\cdot\lambda.\end{equation}

Consider the \emph{asymptotic cone} of $\mathcal{O}_\pi$ (see e.g. \cite[Section 3]{AVAV}) :
\[ \AC(\mathcal{O}_\pi) = \left\{ v \in \g^* \ : \ \text{$\exists (\varepsilon_n, x_n)\in (\R_+ \times \mathcal{O}_\pi)^\mathbb{N}$ such that $\varepsilon_n \to 0$ and $\varepsilon_n x_n \to v$}\}  \right\}. \]
This is a union of nilpotent $G(\R)$-orbits on $i\g(\R)^*$.

\begin{lemma} \label{lem:WF_and_AC}
Let~$\pi$ be a discrete series representation of $G(\R)$. The wave-front set $\WF(\pi)$ is equal to the asymptotic cone $\AC(\mathcal{O}_\pi)$.
\end{lemma}

For this see \cite{HarrisHeOlafsson}; in our case the result is due to Rossmann and can be extracted from \cite{RossmannPicard}, Theorem B and C. \qed

In Section~\ref{sec:alternate_WF} below we will  give another simple description of~$\WF(\pi)$, based on the description of $\AV(\pi)$ in the previous paragraph and on the Sekiguchi correspondence. 

\subsection{Large discrete series to Whittaker data ((1) $\leftrightarrow$ (2))}


We recall a recent result of the first author and Tasho Kaletha, which spells out the connection between large discrete series and Whittaker data using  {\it Kostant sections}. Let~$X$ be a regular nilpotent element of~$\g$, and choose elements $H, Y \in \g$ such that $(X,H,Y)$ is an  $\SL(2)$-triple. The Kostant section~$\Kostant{X}$ is the affine subspace $X + \mathrm{Cent}_{\g}(Y)$ of~$\g$. This depends on the choice of $\SL(2)$-triple, but if $X \in \g(\R)$, then the $G(\R)$-conjugacy class of~$\Kostant{X}$ depends only on the $G(\R)$-conjugacy class of~$X$. 


\begin{proposition}[\cite{adams_kaletha}]\label{JeffTasho_criterion}
Suppose $\pi$ is a generic discrete series representation of~$G(\R)$. Suppose~$\pi$ be a Whittaker datum for~$G(\R)$, and let~$X \in i\g(\R)^*$ be a regular nilpotent element such that $\w = \w_X$. 
Then $\pi$ is $\w$-generic if and only if $\Kostant{X}$ meets $\mathcal{O}_\pi$. 
\end{proposition}


This confirms that  the corresponding statement  in the $p$-adic case \cite{debacker_reeder_generic, kaletha_epipelagic}
applies to the real case as well.


\subsection{Connection between ((3) $\leftrightarrow$ (4)) and the Kostant--Sekiguchi correspondence } 

Theorem~\ref{th:main} gives a bijection between the set $(\Op \cap \s^*)/K$ of principal nilpotent~$K$-orbits and the set $(\Op \cap i\g(\R)^*/G(\R)$ of principal nilpotent~$G(\R)$-orbits. 

It is well known that a natural bijection between nilpotent $K$-orbits and nilpotent $G(\R)$-orbits can be defined in more elementary terms. This is the Kostant--Sekiguchi correspondence~\cite{sekiguchi}. Furthermore, it is a deep theorem of Schmid and Vilonen (the main result of~\cite{SV1}) that the wavefront set and the associated variety are exchanged by this correspondence. (As a historical aside, the Kostant--Sekiguchi correspondence was discovered on the basis that the correspondence $\AV(\pi) \leftrightarrow \WF(\pi)$ might be expressed in elementary terms.)

Therefore, a particular case of Schmid and Vilonen's theorem says: 
\begin{proposition} \label{Sekiguchi_result} The bijection (3)--(4) is induced by the Kostant--Sekiguchi correspondence.
\end{proposition} 
We shall do two things here. First, we will quote a result of \cite{AVAV} which makes this case of the Kostant--Sekiguchi correspondence completely explicit. Second, we shall give a new and elementary proof of Proposition~\ref{Sekiguchi_result}: this avoids the use of Schmid and Vilonen's difficult results, and instead uses the results above (in particular Proposition~\ref{JeffTasho_criterion}) as the main ingredients. 


\subsubsection{The Sekiguchi correspondence for principal nilpotent orbits}\label{sec:concrete_sek}
Let us spell out the special case of the Sekiguchi correspondence which is of interest here. We follow~\cite[Section~2]{AVAV}.


Suppose $\xi_\R\in \Op\cap i\g(\R)^\ast$, and write $\xi_\R = F_\R^\ast$ where $F_\R$ is a principal nilpotent element. We may choose $H_\R, E_\R \in \g$ such that $(H_\R,E_\R,F_\R)$ is an $\mathrm{SL}(2)$-triple and  $E_\R$  contained in $i\g(\R)$,
which implies $H_\R\in \g(\R)$. 
After conjugating by~$G(\R)$ we may further assume $\theta(E_\R)=-F_\R$. Set 
\[   F_\theta=\frac12(-E_\R+F_\R+H_\R).\]
Then the Sekiguchi correspondence takes the $G(\R)$-orbit of~$\xi_\R$ to the $K$-orbit of $\xi_\theta=F_\theta^\ast$. 

Computing the map in the other direction goes as follows.
Suppose $\xi_\theta \in \Op\cap \s^\ast$, and let $F_\theta \in \s$ be such that $\xi_\theta=F_\theta^\ast$. Choose $E_\theta, H_\theta \in \g$ such that $(E_\theta, H_\theta, F_\theta)$ is an $SL(2)$-triple with $E_\theta \in \s$,  which implies $H_\theta\in\k$. After conjugating by $K$ we may further assume $\sigma(E_\theta)=F_\theta$.
Define 
\begin{equation}\label{def_F_R}
  F_\R=\frac12(-E_\theta+F_\theta-H_\theta)\\
\end{equation}
Then the Sekiguchi correspondence takes the $K$-orbit of $\xi_\theta$ to the $G(\R)$-orbit of $\xi_\R = F_\R^\ast$. 

\subsubsection{A simple proof of Proposition~\ref{Sekiguchi_result}}\label{sec:sekiguchi_proof}

Consider the principal nilpotent element~$F_\pi$ defined in~\eqref{e:Fpi}. Let us use this as~$F_\theta$ in the previous subsection: define $E_\theta = \sum_{\alpha \in S(\lambda)}X_\alpha$ and $H_\theta  = [E_\theta, F_\theta]$; then $E_\theta \in \s$ and $(E_\theta, H_\theta, F_\theta)$ is an $\mathrm{SL}(2)$-triple. Define another $\mathrm{SL}(2)$-triple $(E_\R, F_\R, H_\R)$ by
\[ \begin{aligned}
  F_\R&=\frac12(-E_\theta+F_\theta-H_\theta)\\
  E_\R&=\frac12(E_\theta-F_\theta-H_\theta)\\
  H_\R&=E_\theta+F_\theta.
\end{aligned}\]
 
We know that  $\AV(\pi)=K \cdot \xi_\pi$ (where $\xi_\pi\in\g^*$ corresponds to $F_\pi$, see Lemma~\ref{l:pi_to_av}), so the image of $\AV(\pi)$ by the Sekiguchi correspondence is 
 $G(\R)\cdot F_\R$. We want to check that this is equal to $\WF(\pi)$. 

By Lemma~\ref{lem:matumoto} it is enough to check that $\pi$ is $\w$-generic for the Whittaker datum $\w$ corresponding to~$F_\R$. By Proposition~\ref{JeffTasho_criterion} this is the case if and only if the Kostant section $\Kostant{F_\R}$ meets $\mathcal{O}_\pi$. 

Now, we know that $\Kostant{F_\R}=F_\R+\mathrm{Cent}_\g(E_\R)$ contains $F_\R+E_\R = -H_\theta$.
In general the orbits $\mathcal{O}_\pi$ and $G(\R) \cdot (-H_\theta)$ are disjoint. 
Nevertheless $H_{\lambda}$ and $-H_\theta=[F_\theta, E_\theta] = \sum_{\alpha \in S(\lambda)} [X_{-\alpha,} X_\alpha]H_\alpha$ are in the same Weyl chamber. 
Therefore the result follows from:  

\begin{proposition}\label{prop:AC_chamber} Let $\mathcal{C} \subset \t$ be a large open Weyl chamber.
If $H$ and $H'$ are  points of~$\mathcal{C}$ and $\mathcal{O}, \mathcal{O}'$ are the   $G(\R)$-orbits of $H$ and $H'$, then $\AC(\mathcal{O})=\AC(\mathcal{O}')$.
\end{proposition}

We shall prove this in the rest of this section, based on the following lemma.

\begin{lemma}\label{lem:Kost_and_AC} Let $\mathcal{O}$ be a regular semisimple orbit and  let $X$ be a regular nilpotent element. Then $\Kostant{X}$ meets~$\mathcal{O}$ if and only if $X$ is in the asymptotic cone~$\AC(\mathcal{O})$. 

Consequently each of the three invariants $\Wh(\pi),\AV(\pi)$ and $\WF(\pi)$ are constant for $\pi=\pi(\lambda)$ a large discrete series representation, and $\lambda$ in a fixed large Weyl chamber.
\end{lemma}
\begin{proof} 
If $X\in \AC(\mathcal O)$ then $\Kostant{X}$ meets $\mathcal O$ by \cite[Lemma 3.2.3]{adams_kaletha}.

Suppose $\Kostant{X}$ meets $\mathcal O$.
This means there is an $SL(2)$-triple $(X,H,Y)$, and an element $Z\in \Cent_\g(Y)$ such that $W=X+Z \in \O$. 
By \cite{Kostant59}, the centralizer  $\mathrm{Cent}_{\g}(Y)$ is abelian; furthermore there exists a Cartan subalgebra $\mathfrak{h} \subset \g$, and a positive system $\Delta$ for the roots of $\mathfrak{h}$ in $\g$, such that $E$ is contained in the sum of $\Delta$-positive root spaces, while $\mathrm{Cent}_{\g}(F)$ is contained in the sum of $\Delta$-negative roots.  Therefore, if we set $h(t) = \exp(tH)$ for $t \in \R_+$, then 
\[\begin{cases} \mathrm{Ad}(h(t)) \cdot F \longrightarrow 0 & \text{as $t \to +\infty$, while} \\
 \mathrm{Ad}(h(t)) \cdot E = f(t) E & \text{ with $f(t) >0$ and $f(t)\to +\infty$ as $t \to \infty$.}\end{cases}\]
Set $Y(t) = \mathrm{Ad}(h(t)) Y$; then $f(t) \left[ E + Y(t) \right] = \mathrm{Ad}(h(t)) W$ lies on the orbit~$\mathcal{O}$. Thus there exists a continuous path in $\R^+ \cdot \mathcal{O}$ with $E=X$ as a limit point; by definition this means $X \in \AC(\mathcal{O})$.%By the representation theory of $SL(2)$ we know that $Z=\sum_{j=0}^n Z_j$ with $[H,Z_j]=-jZ_j$. 
%Then 
%$$
%\Ad(e^{tH})(X+Z)=e^{2t}(X) + \sum_j e^{-tj}Z_j
%$$
%Taking the limit as $t\rightarrow 0$ gives $\Ad(e^{tH})(X+Z)\rightarrow e^{2t}X$. Therefore
%$$
%\lim_{t\rightarrow 0}e^{-2t}\Ad(e^{tH})(X+Z)=X
%$$ which implies the result.
\end{proof}

For a closely related result see \cite[Proposition 3.5]{fm}.

The connection between Lemma~\ref{lem:Kost_and_AC} and Proposition~\ref{prop:AC_chamber} is that Lemma~\ref{lem:Kost_and_AC}  implies the  following observation on the behavior of the asymptotic cone of regular semisimple orbits: 

\begin{lemma}\label{lem:AC_containment} Let~$\Omega$ be a regular nilpotent $G(\R)$-orbit, and let $\mathcal{E}_\Omega \subset \g$ be the union of all regular semisimple orbits~$\mathcal{O}$ such that $\AC(\mathcal{O})$ contains~$\Omega$. Then $\mathcal{E}_\Omega$ is open. 
\end{lemma}

\begin{proof}
Let $F$ be an element of~$\Omega$, and let $\mathcal{O}$  be a regular semisimple orbit such that $\AC(\mathcal{O})$ contains~$\Omega$.
Then $\Kostant{F}$ meets $\mathcal{O}$ by  Lemma~\ref{lem:Kost_and_AC}. 
Fix $X$ in $\mathcal{O} \cap \Kostant{F}$.
We have $X =F+Y$ where $Y$ belongs to the centralizer $\mathrm{Cent}_\g(F)$.
Since this centralizer is transverse to the $G(\R)$-orbits and has dimension $\dim(\g(\R))-\dim(\t(\R))$,
there exists an open neighborhood of~$X$ consisting of (regular semisimple) orbits  which meet $\Kostant(F)$.
Using Lemma~\ref{lem:Kost_and_AC} again, we see that all those orbits $\mathcal{O}'$ satisfy $F \in \AC(\mathcal{O}')$,
which implies that $\Omega = G(\R) \cdot F$ is contained in $\AC(\mathcal{O}')$.
This proves that $\mathcal{E}_\Omega$ contains an open neighborhood of any of its points, hence is open.   \end{proof}


Let us prove Proposition~\ref{prop:AC_chamber}. Given an element $H$ of $\mathcal{C}$ we shall write $\mathcal{O}_H$ for the $G(\R)$-orbit of~$H$. For any regular nilpotent orbit~$\Omega$ let $\ECom$  denote the set of $H \in \mathcal{C}$ such that $\AC(\mathcal{O}_H)$ contains $\Omega$; by Lemma~\ref{lem:AC_containment}  we know that   $\ECom$ is an open subset of~$\mathcal{C}$. 

Now let $\mathcal{S}$ be the unit sphere of $\t$ for the norm induced by the Killing form, and $\mathcal{S} \cap \mathcal{C}$ be the intersection of $\mathcal{S}$ with the given chamber. This is an open subset of $\mathcal{S}$. The previous arguments show that the the radial projection $\mathcal{S}_{\Omega}$ of  $E^{\mathcal{C}}_\Omega$ on $\mathcal{S}$ is an open subset of $\mathcal{S} \cap \mathcal{C}$. 

Given regular nilpotent orbits $\Omega$ and $\Omega'$, we deduce that $\mathcal{S}_\Omega \cap \mathcal{S}_{\Omega'}$ is an open subset of the $\mathcal{S} \cap \mathcal{C}$.

Now consider the set $\Sigma$ of all points on $\mathcal{S} \cap \mathcal{C}$ which arise as the radial projection of the Harish-Chandra parameter~$H$ of some large discrete series representation. Since the Harish-Chandra parameters for discrete series are (a translate of) an integral lattice, the set $\Sigma$ is dense in $\mathcal{S} \cap \mathcal{C}$. Therefore if the open set $\mathcal{S}_\Omega \cap \mathcal{S}_{\Omega'}$ is nonempty, there must exist at least one large discrete series representation~$\pi$ whose Harish-Chandra projects to $\mathcal{S}_\Omega \cap \mathcal{S}_{\Omega'}$. By Lemma~\ref{lem:WF_and_AC} this means $\Omega$ and $\Omega'$ must be contained in $\WF(\pi)$, which implies $\Omega = \Omega'$ since the wavefront set is the closure of a single orbit if $\pi$ is a large discrete series. 

We conclude that if $\Omega \neq \Omega'$, then $\mathcal{S}_\Omega$ and $\mathcal{S}_\Omega$ are disjoint, and therefore $\ECom$ and $\mathcal{E}^{\mathcal{C}}_{\Omega'}$ are disjoint open subsets of~$\mathcal{C}$. Since $\mathcal{C}$ is connected and equal to the union of all subsets $\mathcal{E}^{\mathcal{C}}_{\Omega}$, the chamber $\mathcal{C}$ must be equal to a single $\ECom$. This concludes the proof of Proposition~\ref{prop:AC_chamber}, and therefore also of Proposition~\ref{Sekiguchi_result}.


\subsubsection{A simple description of the wavefront set}\label{sec:alternate_WF}


As we saw the correspondence $\pi \leftrightarrow \WF(\pi)$ can be described using the asymptotic cone of the semisimple orbit of the Harish-Chandra parameter. We used this in the proof of Proposition~\ref{Sekiguchi_result}. Now, the results of Sections~\ref{sec:concrete_sek} and~\ref{sec:sekiguchi_proof} provide another description of $\WF(\pi)$ from the Harish-Chandra parameter: we can compose the maps (1)$\leftrightarrow$(3)$\leftrightarrow$(4). This gives the following rather concrete conclusion.

\begin{lemma}
  Let $\pi$ be a large discrete series representation.
Choose a compact Cartan subgroup $T$ and let $\lambda\in \t^*$ be the Harish-Chandra parameter of $\pi$.
Define $F_\theta=F_\pi$ as in \eqref{e:Fpi}, and choose  $E_\theta$, $H_\theta$ such that $(H_\theta,E_\theta,F_\theta)$ is an $\mathrm{SL}(2)$-triple with $\sigma(E_\theta)=F_\theta$.  Then
$$
\WF(\pi)=G(\R)\cdot F_\R^\ast, \quad \text{where $F_\R = \frac i2(-E_\theta+F_\theta-H_\theta)\in\N\cap i\g(\R)$.}
$$\end{lemma}
Conversely, assume we are given an $L$-packet~$\Pi$ of generic discrete series, and consider a regular semisimple orbit~$\Omega$ in~$\Op \cap i\g(\R)^\ast$. We can determine the unique element~$\pi \in \Pi$ with $\WF(\pi)=\Omega$ as follows. Fix~$\xi_\R \in \Omega$, write $\xi_\R=F_\R^\ast$ with $F_\R \in \g$ and let $E_\R, H_\R \in \g$ be such that $(E_\R, H_\R, F_\R$ is an  $\mathrm{SL}(2)$-triple with $\theta(E_\R)=-F_\R$. Set
Let
$$
H_\theta=E_\R-F_\R.
$$
Then $H_\theta$ is a regular semisimple element, and belongs to~$\k$.
Define a Cartan subalgebra~$\t$ of~$\g$ by $\t=\Cent_{\g}(H_\theta)$, and a set of positive roots by
$$
\Delta^+=\{\alpha\in\Delta(\t,\g)\mid  \alpha(H_\theta)>0\}.
$$
Then $\pi$ is the discrete series representation in $\Pi$ whose Harish-Chandra parameter is dominant for $\Delta^+$. 

\subsection{Application: Whittaker model and Weyl chamber of the Harish-Chandra parameter}

As an application of the above results, here is a simple proof of a fact which is well known to the experts, but not easy to locate in the literature.

\begin{lemma}
\label{l:equalWhittaker}
Suppose $\pi$, $\pi'$ are large discrete series representations of~$G(\R)$. Let~$\lambda, \lambda'$ be representatives for the Harish-Chandra parameters of~$\pi, \pi'$ respectively, and let $\mathscr{C}, \mathscr{C}'$ be the corresponding Weyl chambers. 

Then $\Wh(\pi)=\Wh(\pi')$ if and only if $\mathscr{C}, \mathscr{C}'$ are in the same $W_K$-orbit. 
\end{lemma}

\begin{proof}
If  $\mathscr{C}, \mathscr{C}'$ are in the same $W_K$-orbit, then the Harish-Chandra parameters of $\pi$, $\pi'$ have representatives in the same Weyl chamber; therefore $\Wh(\pi)=\Wh(\pi')$ by Lemma~\ref{lem:Kost_and_AC}.

For the other direction, by Theorem \ref{th:main} we can replace
$\Wh(\pi)$ in the statement with $\WF(\pi)$.  So suppose
$\WF(\pi)=\WF(\pi')$. By Lemma \ref{lem:Kost_and_AC} we may assume
$\lambda=w\lambda'$ for some $w\in W$, i.e. $\pi,\pi'$ are contained
in the same $L$-packet $\Pi$. In that case, by Theorem \ref{th:main} 
we know $\Wh(\pi)=\Wh(\pi')$ if and only if $\pi\simeq \pi'$.
This holds if and only if $\lambda'=w\lambda$ with $w\in W_K$, 
i.e. $\mathscr{C}$ and $\mathscr{C}'$ are $W_K$-conjugate.
\end{proof}

\subsection{Appendix: $SL(2,\R)$ calculation}

This is not for publication, just to make sure we have the formulas in Section \ref{sec:concrete_sek} correct.
These are:
$$
E_\theta=\frac12(E_\R-F_\R+H_\R)\in \N\cap \s.
$$
and
$$
E_\R=\frac 12(E_\theta-F_\theta-H_\theta)\in\N\cap i\g(\R).
$$

Recall $\sigma(X)=\overline X$ and $\theta(X)=-X^t$.

Take
$$
E_\R=\begin{pmatrix}0&i\\0&0\end{pmatrix},
F_\R=\begin{pmatrix}0&0\\-i&0\end{pmatrix},
H_\R=\begin{pmatrix}1&0\\0&-1\end{pmatrix}
$$
These satisfy the standard $SL(2)$ commutation relations: 
\[ [E_\R, F_\R] = H_\R, \quad [H_\R, E_\R] = 2E_\R, \quad [H_\R, F_\R] = -2F_\R. \]

Furthermore $E_\R,F_\R$ are in the $-1$-eigenspace of $\sigma$,
and  we have $\theta(E_\R)=F_\R$ and $\theta(H_\R)=-H_\R$.

Set $g=\frac 1{\sqrt 2}\begin{pmatrix}1&i\\i&1
\end{pmatrix}$, and take $E_\theta,F_\theta,H_\theta$ be the conjugates of $E_\R,F_\R,H_\R$ by~$g$:
$$
E_\theta=\frac12\begin{pmatrix}1&i\\i&-1\end{pmatrix},
F_\theta=\frac12\begin{pmatrix}1&-i\\-i&-1\end{pmatrix},
H_\theta=\begin{pmatrix}0&-i\\i&0\end{pmatrix},
$$


These satisfy the standard $SL(2)$ commutation relations, and $E_\theta,F_\theta$ are in the $-1$-eigenspace of $\theta$.
Also $\sigma(E_\theta)=F_\theta,\sigma(H_\theta)=-H_\theta$.

Furthermore:
$$
\begin{aligned}
  E_\theta&=\frac12(E_\R-F_\R+H_\R)\\
  F_\theta&=\frac12(-E_\R+F_\R+H_\R)\\
  H_\theta&=-E_\R-F_\R
\end{aligned}
$$
and
$$
\begin{aligned}
  E_\R&=\frac12(E_\theta-F_\theta-H_\theta)\\
  F_\R&=\frac12(-E_\theta+F_\theta-H_\theta)\\
  H_\R&=E_\theta+F_\theta
\end{aligned}
$$



\bibliographystyle{plain}
\bibliography{Refs}

\end{document}
