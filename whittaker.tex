%edited starting 8/27/17 for re-submission to Duke
\documentclass[10pt,leqno]{article}
\usepackage{verbatim}
\usepackage{amssymb}
\usepackage{mathtools}
%\usepackage{amsrefs}
\usepackage{rotating}
\usepackage{xcolor}
\usepackage{amsmath}
\usepackage{mathrsfs}
\usepackage{enumerate}
\usepackage[colorlinks=true, linktoc=page, citecolor=blue, linkcolor=blue, urlcolor=blue]{hyperref}

%\usepackage{showkeys}
\usepackage{tabularx}
\setlength\extrarowheight{4pt}   %spacing in tables
\usepackage{theorem}
\usepackage[matrix,tips,frame,color,line,poly,curve]{xy}
\renewcommand{\labelenumi}{(\arabic{enumi})}
\newcommand\kappaarrow[2]{#1\overset\kappa\rightarrow#2}
\newtheorem{theorem}[equation]{Theorem}
\newtheorem{corollary}[equation]{Corollary}
\newtheorem{definition}[equation]{Definition}
\newtheorem{lemma}[equation]{Lemma}
\newtheorem{desideratum}[equation]{Desideratum}
\newtheorem{conjecture}[equation]{Conjecture}
\newtheorem{proposition}[equation]{Proposition}
\newtheorem{remark}[equation]{Remark}
{\theorembodyfont{\rmfamily}\newtheorem{theoremplain}[equation]{Theorem}
\newtheorem{remarkplain}[equation]{Remark}
\newtheorem{editorialremarkplain}[equation]{Editorial Remark}
\newtheorem{exampleplain}[equation]{Example}
\newtheorem{corollaryplain}[equation]{Corollary}
\newtheorem{mytable}[equation]{Table}
}

\renewcommand{\sec}[1]{\section{#1}
\renewcommand{\theequation}{\thesection.\arabic{equation}}
  \setcounter{equation}{0}}
\newcommand{\subsec}[1]{\subsection{#1}
\renewcommand{\theequation}{\thesubsection.\arabic{equation}}
  \setcounter{equation}{0}}

\newcommand{\subsubsec}[1]{\subsubsection{#1}
\renewcommand{\theequation}{\thesubsection.\arabic{equation}}
  \setcounter{equation}{0}}

% Danger, Will Robinson!
\def\danger{\begin{trivlist}\item[]\noindent%
\begingroup\hangindent=3pc\hangafter=-2%\clubpenalty=10000%
\def\par{\endgraf\endgroup}%
\hbox to0pt{\hskip-\hangindent\dbend\hfill}\ignorespaces}
\def\enddanger{\par\end{trivlist}}

\newcommand{\Gext}{\negthinspace\negthinspace\phantom{a}^\delta G}
\newcommand{\thetaG}{\negthinspace\negthinspace\phantom{a}^\theta
  G(\C)}
\newcommand{\thetaK}{\negthinspace\negthinspace\phantom{a}^\theta K(\C)}
\newcommand{\qed}{\hfill $\square$ \medskip}
\newenvironment{proof}[1][Proof]{\noindent\textbf{#1.} }{\qed}
\newcommand\exact[3]{1\rightarrow #1\rightarrow #2\rightarrow #3\rightarrow1}
\newcommand{\Aut}{\mathrm{Aut}}
\newcommand{\Inv}{\mathrm{Invol}}
\newcommand{\sgn}{\mathrm{sgn}}
\newcommand{\diag}{\mathrm{diag}}
\newcommand{\gr}{\mathrm{gr}}
\newcommand{\Out}{\mathrm{Out}}
\newcommand{\Int}{\mathrm{Int}}
\renewcommand{\int}{\mathrm{int}}
\newcommand{\Hom}{\mathrm{Hom}}
\newcommand{\kernel}{\mathrm{kernel}}
\newcommand{\Ad}{\mathrm{Ad}}
\newcommand{\ad}{\mathrm{ad}}
\newcommand{\zinv}{\mathrm{inv}}
\newcommand{\SRF}{\mathrm{SRF}}
\newcommand{\Gad}{G_\mathrm{ad}}
\newcommand{\Gsc}{G_\mathrm{sc}}
\newcommand{\Zsc}{Z_\mathrm{sc}}
\newcommand{\Ztor}{Z_\mathrm{tor}}
\newcommand{\Gbar}{\overline G}
\newcommand{\Kad}{K_\mathrm{ad}}
\newcommand{\Gal}{\mathrm{Gal}}
\newcommand{\Norm}{\mathrm{Norm}}
\newcommand{\Cent}{\mathrm{Cent}}
\newcommand{\Stab}{\mathrm{Stab}}
\newcommand{\I}{\mathcal I}
\renewcommand{\O}{\mathcal O}
\newcommand{\R}{\mathbb R}
\newcommand{\C}{\mathbb C}
\newcommand{\Z}{\mathbb Z}
\newcommand{\W}{\mathbb W}
\newcommand{\Ztwo}{\mathbb Z_2}
\newcommand{\N}{\mathcal N}
\newcommand{\Q}{\mathbb Q}
\newcommand{\E}{\mathbb E}
\newcommand{\G}{G}
\renewcommand{\H}{\mathbb H}
\newcommand{\h}{\mathfrak h}
\newcommand{\n}{\mathfrak n}
\renewcommand{\sl}{\mathfrak s\mathfrak l}
\renewcommand{\P}{\mathfrak p}
\renewcommand{\a}{\mathfrak a}
\newcommand{\zk}{\mathfrak z_\mathfrak k}
\newcommand{\A}{\mathbb A}
\newcommand{\K}{\mathcal K}
\newcommand{\B}{\mathcal B}
\renewcommand{\k}{\mathfrak k}
\newcommand{\spint}{\widetilde{Spin}}
\newcommand{\ch}[1]{#1^\vee}
\newcommand\sigmaqc{\sigma_{\text{qc}}}
\newcommand\thetaqc{\theta_{\text{qc}}}
\newcommand{\Fgal}{F_{\text{gal}}}
\newcommand{\Falg}{F_{\text{alg}}}
\newcommand{\cl}{\mathit{cl}}
\newcommand{\Lie}{\mathrm{Lie}}
\newcommand{\opp}{\text{-opp}}

\renewcommand{\t}{\mathfrak t}
\newcommand{\su}{\mathfrak s\mathfrak u}
\newcommand{\g}{\mathfrak g}
\newcommand\inv{^{-1}}
\newcommand\wh{\widehat}
\newcommand{\GL}{\text{GL}}
\newcommand{\SL}{\text{SL}}
\newcommand{\SO}{\text{SO}}
\newcommand{\SU}{\text{SU}}
\newcommand{\Spin}{\text{Spin}}
\newcommand{\chGGamma}{\phantom{a}^\vee G^\Gamma}
\newcommand{\GGamma}{G^\Gamma}
\newcommand{\s}{\mathfrak s}
\newcommand{\w}{\mathfrak w}
\newcommand{\AV}{\mathrm{AV}}
\newcommand{\WF}{\mathrm{WF}}
\newcommand{\AC}{\mathrm{AC}}
\newcommand{\AVann}{\mathrm{AV}_{\mathrm{ann}}}
\newcommand{\GK}{\mathrm{GK}}
\newcommand{\Op}{\O_p}
\newcommand{\Kostant}[1]{\mathcal{K}(#1)}

\begin{document}
\title{Whittaker Models for Real Groups}
\author{Jeffrey Adams \& Alexandre Afgoustidis}
\maketitle

Let $G$ be a connected complex reductive group, defined over $\R$, and suppose $G(\R)$ is quasisplit.


We assume $G(\R)$ has discrete series representations, and fix an L-packet $\Pi$ of discrete series representations.

By a  {\it Whittaker datum} we mean a $G(\R)$-conjugacy class of pairs  $(B(\R),\eta)$ where $B(\R)$ is a
Borel subgroup of $G(\R)$, 
and $\eta$ is a non-degenerate character of the nilpotent radical $N(\R)$ of $B(\R)$. Non-degenerate means: non-trivial on each simple root space.

TODO: write about generic/large representations, and explain what's in the paper...

\section{Some invariants of irreducible $(\g, K)$-modules}



Suppose $\pi$ is an irreducible $(\g,K)$-module. Then we can associate to~$\pi$ several invariants, which depend only on the equivalence class of~$\pi$: the associated variety, the Gelfand---Kirillov dimension, and the wavefront set.  We do not give full definitions, but try to point to convenient places in the literature.\footnote{AA 20-II-2024: of course if we write this, we should maybe add a few basic references for the notions.}

\subsection{Some notation}

Let~$\sigma$ be the anti-holomorphic involution defining~$G(\R)$. 
Let $\g=\Lie(G), \g(\R)=\Lie(G(\R))$. Choose a Cartan involution $\theta$ for $G$. This means:
$\theta$ is an algebraic involution, commuting with~$\sigma$,
and $G(\R)^\theta$ is a maximal compact subgroup of~$G(\R)$. Set $K=G^\theta$, $K(\R)=G(\R)^\theta=K^\sigma$.
Let $\s$ be the $-1$ eigenspace of $\theta$ acting on~$\g$.

Let $\N$ be the set of nilpotent elements of $\g$.
This is the closure of the principal nilpotent orbit $\Op$.

 \subsection{The invariants}


Let $I_\pi$ be the annihlator of $\pi$ in the universal enveloping algebra.
Let $\AVann(\pi)$ be the associated variety of $I_\pi$. 
This is the closure of a single complex nilpotent orbit $\O$.

The associated variety of $\pi$, denoted $\AV(\pi)$, is the  the union of a set of
$K$-orbits on $\O\cap\s$.

The Gelfand---Kirillov dimension~$\GK(\pi)$ of~$\pi$ can be defined in several different ways~\cite{vogan-gelfand-kirillov}; but one of them is
$$
GK(\pi)=\dim(\AV(\pi))=\frac12\dim(\AVann(\pi)).
$$
(All dimensions are complex unless otherwise stated.)

For all of these facts see \cite{vogan_bowdoin}, in particular Corollary 4.7 and Theorem 8.4.

Another invariant of an irreducible $(\g, K)$-module~$\pi$ is the \emph{wave-front set} of $\pi$, written $\WF(\pi)$. It is the union of a (finite) set of nilpotent $G(\R)$ orbits in
$i\g(\R)^*=i\Hom_\R(\g(\R),\R)$. See \cite{howe_wave_front}, \cite{bv_local_structure}.\footnote{AA 14-II-2024: another convenient reference  for definitions re~$\WF(\pi)$ seems to be  Harris--He--Olafsson (Duke 2016).}

\subsection{Large representations} Set $N=\dim(\N)=\dim(G)-\mathrm{rank}(G)$. The maximal Gelfand--Kirillov dimension of a representation is $N/2$.
We say $\pi$ is {\it large} if $\GK(\pi)=N/2$. See \cite[Section~6]{Vogan78}. 


\begin{lemma}
  \label{l:large}
  The following conditions are equivalent.
  \begin{enumerate}
    \item $\pi$ is large;
\item $\AVann(\pi)=\N=\overline{\Op}$;
\item $\dim(\AVann(\pi))=N$;
  \item $\dim(\AV(\pi))=N/2$;
\item $\GK(\pi)=N/2$;

\end{enumerate}
\end{lemma}
See \cite{vogan_bowdoin}.


\section{The dictionary for large discrete series}


\subsection{Statement of the results}  


The following uses rather deep results of representation theory:
\begin{proposition} Let~$\pi$ be the $(\g, K)$-module for a large discrete series representation of~$G(\R)$.  
\begin{enumerate}[(a)]
\item There is a unique Whittaker datum~$\mathfrak{w}$ such that~$\pi$ is $\mathfrak{w}$-generic.
\item $\AV(\pi)$ is the closure of a single $K$-orbit on~$\mathcal{O}_p \cap \mathfrak{s}$.
\item  $\WF(\pi)$ is the closure of a single $G(\R)$-orbit on $\mathcal{O}_p \cap  i \g(\R)$.
\end{enumerate}
\end{proposition}

\begin{proof} For (a), by \cite{vogan-gelfand-kirillov} and \cite{kostant_whittaker}
a representation $\pi$ is large if and only if it admits a Whittaker model for some Whittaker datum,
and by \cite[Lemma 14.14]{abv} a  large discrete series representation admits a unique Whittaker datum.

Part~(b) is one of the main results of~\cite{vogan_bowdoin}: see \cite[Theorem ?]{vogan_bowdoin}.\footnote{AA 19-II-2024: I'd welcome some help in navigating~\cite{vogan_bowdoin}: the relevant places seem to be  Sections 4, 5 and 8, but I have a few questions.}

Part~(c) follows from~\cite{rossmann_limit_orbits}.\footnote{AA 14-II-2024: maybe a little more detail would be welcome, the Rossmann paper does not seem to state the result in these terms.}
\end{proof}

Given a large discrete series representation~$\pi$ we consider the invariants of~$\pi$ defined by the Proposition: a Whittaker datum  $\mathfrak{w}(\pi) \in \mathfrak{W}(G(\R))$, a $K$-orbit $\AV^\flat(\pi) \in (\mathcal{O}_p \cap \mathfrak{s})/K$  and a nilpotent $G(\R)$-orbit $\WF^\flat(\pi) \in (\mathcal{O}_p \cap  i \g(\R))/G(\R)$. 

Here is our main statement.

\begin{theorem} \label{th:main} Suppose~$\Pi$ is an $L$-packet of discrete series for~$G(\R)$. 
The maps $\pi \mapsto \mathfrak{w}(\pi)$, $\pi \mapsto \AV^\flat(\pi)$ and $\pi\mapsto \WF^\flat(\pi)$ induce bijections between:
\begin{enumerate}
\item[(1)] The set~$\Pi_{\mathrm{large}}$ of large discrete series representations in~$\Pi$ ;
\item[(2)] The set of Whittaker data for $G(\R)$ ;
\item[(3)] The set $(\Op \cap \s)/K$ of~$K$-orbits on $\mathcal{O}_p \cap \s$.
\item[(4)] The set $(\Op \cap i\g(\R))/G(\R)$ of~$G(\R)$-orbits on $\mathcal{O}_p \cap  i \g(\R)$.
\end{enumerate}
\end{theorem}

Before embarking on the proof, let us point out that the maps in the theorem are defined using rather deep results of representation theory. However, we shall prove in Section~\ref{sec:explicit} that all bijections $(i) \leftrightarrow (j)$, for $i,j \in \{(1), \dots, (4)\}$, can be spelled out in elementary terms. Furthermore we shall prove, using elementary arguments, that (3) $\leftrightarrow$ (4) coincides with the Kostant--Sekiguchi correspondence.  



Let us now explain the strategy of the proof, which is mainly an exercise in putting together some references that are rather scattered in the literature. What we shall actually do is: introduce a finite group~$Q(\R)$, and point out that 
\begin{itemize}
\item[(i)] it acts simply transitively on the sets~(1)--(4), and 
\item[(ii)] the maps $\pi \mapsto \mathfrak{w}(\pi)$, $\pi \mapsto \AV^\flat(\pi)$ and $\pi\mapsto \WF^\flat(\pi)$  are equivariant for these actions.
\end{itemize}
The theorem follows immediately from these two observations. 

\subsection{The group $Q(\R)$ and its actions} Let $Z=Z(G)$ be the center of $G$, and let $\Gad$ be the complex reductive group~$G/Z$.
Then $Z$ and $\Gad$ are defined over $\R$, and $Z(\R)$ is equal to the center of $G(\R)$. We may view~$\Gad(\R)$ as the set of inner automorphisms of~$G$ which are defined over~$\R$, and this contains the group $G(\R)_{\mathrm{ad}}=G(\R)/Z(\R)$ as a normal subgroup of finite index\footnote{AA 15-II-2024: is there a trivial way to explain why finite index?}.
Define 
$$
Q(\R)=\Gad(\R)/G(\R)_{\mathrm{ad}}.
$$
This finite group may be viewed as a group of outer automorphisms of~$G(\R)$. 

\paragraph*{$\bullet$ Action on~(1)} Since~$Q(\R)$ can be viewed as a group of automorphisms of~$G(\R)$,  it has a canonical action on equivalence classes of $(\g, K)$-modules.\footnote{AA 15-II-2024: probably better to spell out the action.} This action preserves  the property of being large\footnote{AA 15-II-2024: should say something about this using the fact that it comes from automorphisms, either here or in the section on large reps}. It also preserves $L$-packets by~\cite[Lemma 6.18]{Contragredient}.

\paragraph*{$\bullet$ Action on~(2)}  By definition of Whittaker data, the action of~$Q(\R)$ on $G(\R)$ induces a canonical action on the set of Whittaker data. By \cite[(14.15)]{abv} this action is simply transitive. 

It is an immediate consequence of the definitions that the map $\pi \mapsto \w(\pi)$, taking a large discrete series representation to the Whittaker datum for which it has a Whittaker model, is equivariant for the actions of $Q(\R)$ on (1) and (2). 


\paragraph*{$\bullet$ Action on~(4)} The actions of $Q(\R)$ on (3) and (4) are quite similar, and the proof that they are simply transitive use an interpretation of $Q(\R)$ in terms of group cohomology. 
We begin with the case of the wave-front set (4), because the argument a more familiar version of  Galois cohomology. 

To define the action of~$Q(\R)$ on~(4), first observe that~$\Gad$ acts on~$\g$ by the adjoint action. This action preserves~$\Op$, and its restriction to~$\Gad(\R)$ preserves~$\Op\cap (i\g(\R))$. Since elements on a given~$G(\R)_{\mathrm{ad}}$-orbit are in the same~$G(\R)$-orbit, this induces an action of $Q(\R)=\Gad(\R)/G(\R)_{\mathrm{ad}}$ on~$\Op\cap (i\g(\R))/G(\R)$.

\begin{proposition}\label{prop:action_on_real_orbits}
\begin{enumerate} 
\item The action of~$Q(\R)$ on $\Op\cap (i\g(\R))/G(\R)$ is simply transitive.
\item The map $\pi \mapsto \WF(\pi)$ is $Q(\R)$-equivariant.
\end{enumerate}
\end{proposition}


The proof of~(1) uses group cohomology, and we first recall some general notation. Let~$\Gamma$ be the Galois group~$\Z/2\Z$ of~$\R$. When~$A$ is a group and $\tau$ is an involutive automorphism of~$A$, we may consider the cohomology sets $H^0_\tau(\Gamma, A)$ and $H^1_\tau(\Gamma, A)$ attached to the action of~$\Gamma$ on~$G$ using~$\tau$.  These are pointed sets, and not groups in general if~$A$ isn't abelian. We may view $H^0_\tau(\Gamma, A)$ as the fixed-point-set $A^{\tau}$, and $H^1_\tau(\Gamma,A)$ as the quotient of $A^{-\tau} = \{ a \in A, a\tau(a)=1\}$ by the equivalence relation~$\sim$ generated by $a \sim x a \tau(x^{-1})$ for all $x \in G$. 

We can use group cohomology to define an action of $Q(\R)$ on~(4). By definition we have the exact sequence $1\rightarrow Z \rightarrow G \rightarrow \Gad\rightarrow 1$. It 
gives rise to a long exact sequence of pointed sets:
\begin{equation} \label{long_ptset}
1\rightarrow Z(\R) \rightarrow G(\R) \rightarrow \Gad(\R) \rightarrow H^1_\sigma(\Gamma,Z)\rightarrow H^1_\sigma(\Gamma,G)\rightarrow H^1_\sigma(\Gamma,\Gad).
\end{equation}
By definition the connecting map  $ \Gad(\R) \rightarrow H^1_\sigma(\Gamma,Z)$ sends $g \in \Gad(\R)$ to the equivalence class of $g \sigma(g^{-1})$ in $H^1_\sigma(\Gamma, Z)$. Going through the definitions we see that   two elements  $g_1, g_2$ have the same image if and only if $g_1 g_{2}\inv \in G(\R)_{\mathrm{ad}}$; therefore the image of the connecting map is in canonical bijection with $Q(\R)=\Gad(\R)/G(\R)_{\mathrm{ad}}$, and by exactness of~\eqref{long_ptset} we get:

\begin{lemma}
  \label{l:Q}
The above discussion determines a canonical bijection 
 \begin{equation}\label{bij_q_1} Q(\R) \leftrightarrow \kernel(H^1_\sigma(\Gamma,Z)\rightarrow H^1_\sigma(\Gamma,G)).\end{equation}
\end{lemma}



Using this we define an action of~$Q(\R)$ on $(\Op \cap i\g(\R))/G(\R)$. Suppose~$X \in \g$ is nilpotent. Consider the intersection of the orbit $\Omega_X=\mathrm{Ad}(G) \cdot X$ with $\g(\R)$, and its decomposition into $G(\R)$-orbits. The map 
\begin{align} \label{bij_q_2}\beta_X\colon (\Omega_{X} \cap \g(\R))/G(\R) & \to \kernel\left(H^1_\sigma(\Gamma,\mathrm{Stab}_G(X))\rightarrow H^1_\sigma(\Gamma,G)\right) \\ \mathrm{Ad}(g) \cdot X & \mapsto \text{class of $ g\sigma(g^{-1})$ in $H^1_\sigma(\Gamma,\mathrm{Stab}_G(X))$} \nonumber \end{align} 
is a bijection: see \cite[Lemma 5.2]{galois}. 
In the case where $X$ lies on the principal nilpotent orbit $\Op$, the stabilizer $\mathrm{Stab}_G(X)$ is the direct product of $Z(G)$ with a unipotent group\footnote{AA 19-II-2024: should we insert a reference?}, which implies $H^1(\Gamma, \mathrm{Stab}_G(X)) =H^1(\Gamma,Z)$. Using Lemma~\ref{l:Q}, we get a bijection 
\[\Delta_X\colon (\Op \cap i\g(\R))/G(\R) \leftrightarrow Q(\R).\]
This bijection depends  on the choice of~$X \in \Op$; but it defines a simply transitive action of~$Q(\R)$ on~$(\Op \cap i\g(\R))/G(\R)$, which is independent of the choice of~$X$. 

An important point is that this action coincides with that defined before the statement of the Proposition. \textcolor{blue}{TODO: explain why. Basically the reason is that if we hit $X$ with $q_1q_2$ on the LHS of~\eqref{bij_q_2}, then then going through the definitions this goes to the element~$q_1q_2$ of~$Q(\R)$.}

Let us now prove the Proposition. 

First we prove that~$Q(\R)$ acts freely on the set a  $(\Op \cap i\g(\R))/G(\R)$ of orbits. Let~$\omega$ be an orbit, and let~$X$ be a point of~$\omega$. If~$q$ is an element of~$Q(\R)$, then going through the definitions of~\eqref{bij_q_1}--\eqref{bij_q_2} we see that~$\Delta_X(q \cdot \omega) = q$; therefore if~$q\cdot \omega=\omega$, then~$q=1$.

Next we prove that the action is transitive. Let~$\omega, \omega'$ be $G(\R)$-orbits in  $(\Op \cap i\g(\R))$. Fix an element~$X$ of~$\omega$ and consider the bijections~$\beta_X$ and~$\Delta_X$ above. Let~$q \in Q(\R)$ be the element~$\Delta_X(\omega')$. We have seen that $q = \Delta_X(q \cdot \omega)$, therefore \mbox{$\Delta_X(q \cdot \omega)=\Delta_X(\omega')$}, which proves~$q \cdot \omega = \omega'$ and concludes the proof of~(1). 

Part~(2) of the proposition is comes from general properties of the wavefront set in microlocal analysis, and from the fact that the action of $Q(\R)$ comes from automorphisms of~$G(\R)$. More precisely, if~$q$ is an element of~$Q(\R)$ and $\tilde{q}$ is a representative of~$q$ in $\Gad(\R)$, then the action of~$q$ on equivalence classes of $(\g, K)$-modules comes from the action of the automorphism $\mathrm{int}(\tilde{q})$ of~$G(\R)$ on $(\g, K)$-modules. Now the wavefront set of a distribution on a manifold satisfies general covariance properties under diffeomorphisms of the manifold: this follows from Hörmander's original definitions, see e.g. \cite[Section 2, p.~800]{HarrisHeOlafsson}. Applying this to the present situation, and going through the basics as in   \cite[Section 2]{HarrisHeOlafsson}, it immediately follows that $\pi \mapsto \WF(\pi)$ is equivariant under $Q(\R)$, and this concludes the proof of Proposition~\ref{prop:action_on_K_orbits}.

 
\paragraph*{$\bullet$ Action on~(3)} We turn to the action of~$Q(\R)$ on $(\Op \cap \s)/K$, and to the associated variety. For this it is convenient to use an isomorphic variant of~$Q(\R)$, defined in terms of~$K$ rather than~$G(\R)$. 

Since the Cartan involution $\theta$ is an automorphism of~$G$, it preserves the center~$Z=Z(G)$, and induces an involution~$\theta_{\ad}$ of~$\Gad=G/Z$. Let $Z_K$ be the intersection $Z \cap K$, and let $K_{\ad}$ be the quotient $K/Z_K$. The natural map $K \mapsto \Gad$, induced by the inclusion $K \hookrightarrow G$, maps~$K$ to a subgroup of~$\Gad^{\theta_{\ad}}$, and has kernel~$Z_K$. Therefore it factors to an injective morphism $\iota\colon \Kad \hookrightarrow\Gad^{\theta_{\ad}} $. The image of~$\iota$ is a normal subgroup of~$\Gad^{\theta_{\ad}}$; using it we will identify $\Kad$ with a normal subgroup of $\Gad^{\theta_{\ad}}$. Define
\[ Q'(\R) = \Gad^{\theta_{\ad}}/\Kad.\]

\begin{lemma}\label{lem:q_and_qprime}
\begin{enumerate}
\item Every element of $Q(R)=\Gad(\R)/G(\R)_{\ad}$ has a $\theta_\ad$-invariant representative in $\Gad(\R)$.
\item Given $q \in Q(\R)$, and a $\theta_{\ad}$-invariant representative~$\tilde{q}$ of~$q$, the image of~$\tilde{q}$ in~$Q'(\R)$ depends only on~$q$ and not on the choice of~$\tilde{q}$. Denote this image by~$\Psi(q)$; then $q \mapsto \Psi(q)$ is an isomorphism between $Q(\R)$ and~$Q'(\R)$. 
\end{enumerate}
\end{lemma}

\begin{proof} \textcolor{red}{TODO!}\end{proof}


To define the  action of~$Q(\R)$ on $(\Op \cap \s)/K$, we point out that $Q'(\R)$ acts on $\Op \cap \s$ and that this action preserves the $K$-orbits. First, the adjoint action of~$\Gad$ on~$\g$ preserves both $\Op$ and~$\s$. Second, the action of~$K$ on $\Op \cap \s$ is trivial on~$Z_K$, and factors through an action of~$\Kad$. It is then easy to check that if two elements of $\Op \cap \s$ lie in on the same $\Kad$-orbit, then for every $g \in \Gad^{\theta_{\ad}}$, the elements  $\tilde{g} \cdot x$ and $\tilde{g} \cdot y$ lie on the same $\Kad$-orbit. This determines an action of~$Q'(\R)$ on~$(\Op \cap \s)/K$, and by Lemma~\ref{lem:q_and_qprime} we get an action of~$Q(\R)$ on~$(\Op \cap \s)/K$. This is the action featured in the following statement:


\begin{proposition}\label{prop:action_on_K_orbits}
\begin{enumerate} 
\item The action of~$Q(\R)$ on $(\Op \cap \s)/K$ is simply transitive.
\item The map $\pi \mapsto \AV(\pi)$ is $Q(\R)$-equivariant.
\end{enumerate}
\end{proposition}

The proof is similar to that of Proposition~\ref{prop:action_on_real_orbits}, but we need to use the version of Galois cohomology attached to the Cartan involution $\theta$. Taking up the above notation on Galois cohomology, the exact sequence $1\rightarrow Z \rightarrow G \rightarrow \Gad\rightarrow 1$ gives rise to a long exact sequence of pointed sets:
\begin{equation} \label{long_ptset_theta}
1\rightarrow Z_K \rightarrow K \rightarrow \Gad^{\theta_{\ad}} \rightarrow H^1_\theta(\Gamma,Z)\rightarrow H^1_\theta(\Gamma,G)\rightarrow H^1_{\theta_{\ad}}(\Gamma,\Gad).
\end{equation}
As in Lemma~\ref{l:Q}  this determines a bijection $Q'(\R) \leftrightarrow \kernel(H^1_\theta(\Gamma,Z)\rightarrow H^1_{\theta}(\Gamma,G))$. The argument for part~(1) of Proposition~\ref{prop:action_on_real_orbits} works seamlessly if we replace~$\sigma$ with~$\theta$, and proves that the action of~$Q'(\R)$ on $(\Op \cap \s)/K$ is simply transitive. Since the action of~$Q(\R)$ is defined by transport of structure from that of~$Q'(\R)$, this proves~(1). 

As for~(2), it is again an immediate consequence of the precise definition of the associated variety. Stating this needs a little more care, since it is not obvious that $\Gad$ acts on representations of~$G(\R)$. However, the adjoint action of~$\Gad^{\theta_{\ad}}$ on $G$ preserves $K$; this this induces an action of~$\Gad^{\theta_{\ad}}$ on $(\g, K)$-modules. The restriction of that action to~$\Kad$ is trivial, and this gives an action of $Q'(\R)$ acts on equivalence classes of $(\g, K)$-modules. Now, the definition of the associated variety uses filtrations (see the Introduction of~\cite{vogan_bowdoin}) which are all invariant under $\Gad^{\theta_\ad}$, and inspecting the definitions it becomes clear that the map $\pi \mapsto \AV(\pi)$, taking a large discrete series $(\g,K)$-module  (or rather its equivalence class) to its associated variety, is equivariant under~$Q'(\R)$. In view of the relationship between the actions of $Q'(\R)$ and $Q(\R)$ on $(\Op \cap \s)/K$, this proves part~(2) of Proposition~\ref{prop:action_on_K_orbits}.
\qed


\section{Explicit versions of the dictionary}\label{sec:explicit}

\subsection{From Whittaker data to real orbits ((2) $\leftrightarrow$ (4))}

First we describe the set of Whittaker data more explicitly.
Let $B$ be a Borel subgroup of $G$. Let $N$ be the nilradical of $B$, and $\overline N$ the opposite
nilpotent subgroup. Let $\n,\overline\n$ be the Lie algebras of $N,\overline N$.
If $B$ is defined over $\R$ we consider $N(\R),\n(\R), \overline\n(\R)$, etc.
Let $\kappa(\,,\,)$  be the Killing form. 
For $X\in i\overline \n(\R)$ define a unitary character $\psi_X$ of $N(\R)$ by:
$$
\psi_X(e^Y)=e^{2\pi \kappa(X,Y)}\quad(Y\in \n(\R)).
$$
The map $X\rightarrow \psi_X$ is an isomorphism between $i\overline\n(\R)$ and the unitary characters of $N(\R)$.
It is easy to see $\psi_X$ is non-degenerate if and only if $X\in i\g(\R)$ is a principal nilpotent element.
In this case we write $(B(\R),X)$ for the Whittaker datum $(B(\R),\psi_X)$.

It is clear that $(B(\R),X)\mapsto G(\R)\cdot X$ is a bijection between Whittaker data and
$(\Op(\R)\cap i\g(\R))/G(\R)$. The main result of \cite{matumoto} (Theorem A) says:

\begin{lemma}
The bijection (2)$\leftrightarrow$(4) takes the Whittaker datum $(B(\R),X)$ to
the principal orbit $G(\R)\cdot X$. 
 \end{lemma} 



\subsection{Connection between ((3) $\leftrightarrow$ (4)) and the Kostant--Sekiguchi correspondence } 


\begin{lemma} The bijection (3)--(4) is induced by the Kostant--Sekiguchi correspondence.
\end{lemma} 

\begin{proof} If we allow ourselves to use deep results, then this is a particular case of the main theorem of~\cite{SV1} (Theorem 1.4). In a perfect world, we could also give a more elementary argument, using the description of the Kostant--Sekiguchi correspondence in~\cite{galois} and the descriptions of the actions above; but  I'm not sure it's possible to do it without encountering a basepoint issue. Another possibility, suggested by Jeff, is to extract it from  arguments in Adams--Kaletha (e.g. the fact that $\WF(\pi)$ is the asymptotic cone of the $G(\R)$-orbit of the HC parameter of~$\pi$, and the connection with the Kostant section). \textcolor{red}{AA (27-III-2024): I think this last strategy should indeed work, but there are details to discuss with Jeff to complete the argument, and if everything turns out OK then  I'll write up the proof here.}\end{proof}

As a consequence we can make the bijection (3) $\leftrightarrow$ (4) explicit, following \cite[Section~1]{avav}.


Suppose $E_\R\in \Op\cap i\g(\R)$. Then we may choose an $SL(2)$-triple $(H_\R,E_\R,F_\R)$ with $F_\R$  contained in $i\g(\R)$,
which implies $H_\R\in \g(\R)$. 
After conjugating by~$G(\R)$ we may further assume $\theta(E_\R)=-F_\R$.
The Sekiguchi correspondence takes the $G(\R)$-orbit of~$E_\R$ to the $K$-orbit of
$$
E_\theta=\frac12(-iE_\R-iF_\R+H_\R)\in \N\cap \s.
$$


Computing the map in the other direction goes as follows.
Suppose $E_\theta\in \Op\cap\s$. Choose an $SL(2)$-triple  $(E_\theta H_\theta,F_\theta)$ with
$F_\theta\in\s$, which implies $H_\theta\in\k$. After conjugating by $K$ we may further assume $\sigma(E_\theta)=F_\theta$.
Then the Sekiguchi correspondence takes the $K$-orbit of $E_\theta$ to the $G(\R)$-orbit of 
$$
E_\R=\frac i2(E_\theta-F_\theta-H_\theta)\in\N\cap i\g(\R).
$$


\subsection{From large discrete series to $K$-orbits ((1) $\leftrightarrow$ (3))}

Suppose $\pi$ is a  discrete series representation. Choose a compact Cartan subgroup $T$, and let
$\lambda\in\t^*$ be the Harish-Chandra parameter of $\pi$. Let $\Delta$ be the set of roots of $\t$ in $\g$, 
let $\Delta^+(\lambda)=\{\alpha\mid \langle\lambda,\ch\alpha\rangle>0\}$,
and let $S(\lambda)\in\Delta^+(\lambda)$ be the simple roots.
For $\alpha\in \Delta$ let $\g_\alpha$ be the corresponding root space, and choose a non-zero element $X_\alpha\in\g_\alpha$ for each $\alpha$.

\begin{lemma}[(1)$\leftrightarrow$(3)]\label{l:pi_to_av}
Suppose $\pi$ is a large discrete series representation with Harish-Chandra parameter $\lambda\in\t^*$.
Set

\begin{equation}
  \label{e:Epi}
  E_\pi=\sum_{\alpha\in S(\lambda)}X_\alpha\in \N.
\end{equation}

Then $E_\pi\in\s$ is a regular nilpotent element, and
$$
\AV(\pi)=\overline{K\cdot E_\pi}.
$$
\end{lemma}

\begin{proof}
If $\pi$ is any discrete series representation, with Harish-Chandra parameter $\lambda$, let
$\n_{\lambda}=\sum_{\alpha\in\Delta^+(\lambda)}\g_\alpha$.
Then by \cite[Proposition 6.8]{vogan_irreducibility} $AV(\pi)=K\cdot(\n_\lambda\cap\s)$. 

Now assume $\pi$ is large. This implies $X_\alpha\in\s$ for all $\alpha\in S(\lambda)$, so $E_\pi\in \s$. 
By~\cite{kostant_tds} $E_\pi$ is a regular nilpotent element, so the closure of $K\cdot E_\pi$ in $\s$ is equal to $K\cdot(\n_\lambda\cap\s)$. 
\end{proof}




\subsection{From large discrete series to real orbits ((1) $\leftrightarrow$ (4))}


There are several ways to describe this bijection.\footnote{AA 23-II-2024: I think we should mention the version with the asymptotic cone of the orbit of~$H_\pi$, which is extremely clean and concrete. } We choose to compose the maps (1)$\leftrightarrow$(3)$\leftrightarrow$(4).
Here is the conclusion.

\begin{lemma}
  Let $\pi$ be a large discrete series representation.
Choose a compact Cartan subgroup $T$ and let $\lambda\in \t^*$ be the Harish-Chandra parameter of $\pi$.
Define $E_\theta=E_\pi$ as in \eqref{e:Epi}, and choose an $SL(2)$-triple $(H_\theta,E_\theta,F_\theta)$ as usual.
After conjugating by $K$ we may assume $\sigma(E_\theta)=F_\theta$.  Then\footnote{AA 14-II-2023: no closure?}
$$
\WF(\pi)=G(\R)\cdot\frac i2(E_\theta-F_\theta-H_\theta)\in\N\cap i\g(\R)
$$
Conversely if $E_\R\in \Op\cap i\g(\R)$, choose an $SL(2)$-triple $(H_\R,E_\R,F_\R)$ which (after conjugating by $G(\R)$) satisfies $\theta(E_\R)=-F_\R$.
Let
$$
H_\theta=E_\R-F_\R.
$$
Then $H_\theta\in\k$ is a regular semisimple element.
Set $\t=\Cent_{\g}(H_\theta)$, and let
$$
\Delta^+=\{\alpha\in\Delta(T,G)\mid  \alpha(H_\theta)>0\}
$$
Then $\pi$ is the discrete series representation in $\Pi$ whose Harish-Chandra parameter is dominant for $\Delta^+$. 
\end{lemma}

We can also express (1)$\mapsto$(4) using the {\it Kostant section}. Recall that if~$X$ is a regular nilpotent element of~$\g$ and fits into an $\SL(2)$-triple $(X, H, Y)$, the Kostant section~$\Kostant{X}$ is the affine subspace $X + \mathrm{Cent}_{\g}(Y)$ of~$\g$. It depends on the choice of $\SL(2)$-triple, but if $X \in \g(\R)$, then the $G(\R)$-conjugacy class of~$\Kostant{X}$ depends only on the $G(\R)$-conjugacy class of~$X$. 

Suppose $\pi$ is a discrete series representation.
Choose a compact Cartan subgroup $T$, and let $\lambda\in\t^*$ be the Harish-Chandra parameter of $\pi$.
Identify $\lambda$ with an element $H_\pi\in i\t\subset i\g$ using the isomorphism $\t\simeq \t^*$ induced by the Killing form. 

\begin{proposition}[\cite{adams_kaletha}]
Suppose $\pi$ is a generic discrete series representation. Then $\pi$ is $\w$-generic
for a unique Whittaker datum $\w$.
Write $\w=\w_X$ for some regular nilpotent element $X\in i\g(\R)$.
Then $H_\pi$ is $G(\R)$-conjugate to an element of the Kostant section of $X$.\footnote{AA 23-II-2024: would probably be useful to explain the connection with $\pi \mapsto \WF(\pi)$ in more detail...\\ AA 27-III-2024: furthermore, if the Kostant section is used in the proof that $\AV \leftrightarrow \WF$ coincides with Sekiguchi, then we should move this subsection to an earlier place, and then probably this discussion will have to be reworked.  }
\end{proposition}


This confirms that  the corresponding statement  in the $p$-adic case \cite{debacker_reeder_generic, kaletha_epipelagic}
applies to the real case as well.

\bibliographystyle{plain}
\bibliography{Refs}

\end{document}
%%% Local Variables: 
%%% mode: latex
%%% TeX-master: t
%%% End: s_{1}:0 -> -1 6


