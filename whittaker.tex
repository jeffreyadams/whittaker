%edited starting 8/27/17 for re-submission to Duke
\documentclass[10pt,leqno]{article}
\usepackage{verbatim}
\usepackage{amssymb, amscd}
\usepackage{mathtools}
%\usepackage{amsrefs}
\usepackage{rotating}
\usepackage{xcolor}
\usepackage{amsmath}
\usepackage{mathrsfs}
\usepackage{enumerate}
\usepackage[colorlinks=true, linktoc=page, citecolor=blue, linkcolor=blue, urlcolor=blue]{hyperref}

%\usepackage{showkeys}
\usepackage{tabularx}
\setlength\extrarowheight{4pt}   %spacing in tables
\usepackage{theorem}
\usepackage[matrix,tips,frame,color,line,poly,curve]{xy}
\renewcommand{\labelenumi}{(\arabic{enumi})}
\newcommand\kappaarrow[2]{#1\overset\kappa\rightarrow#2}
\newtheorem{theorem}[equation]{Theorem}
\newtheorem{corollary}[equation]{Corollary}
\newtheorem{definition}[equation]{Definition}
\newtheorem{lemma}[equation]{Lemma}
\newtheorem{desideratum}[equation]{Desideratum}
\newtheorem{conjecture}[equation]{Conjecture}
\newtheorem{proposition}[equation]{Proposition}
\newtheorem{remark}[equation]{Remark}
{\theorembodyfont{\rmfamily}\newtheorem{theoremplain}[equation]{Theorem}
\newtheorem{remarkplain}[equation]{Remark}
\newtheorem{editorialremarkplain}[equation]{Editorial Remark}
\newtheorem{exampleplain}[equation]{Example}
\newtheorem{corollaryplain}[equation]{Corollary}
\newtheorem{mytable}[equation]{Table}
}

\renewcommand{\sec}[1]{\section{#1}
\renewcommand{\theequation}{\thesection.\arabic{equation}}
  \setcounter{equation}{0}}
\newcommand{\subsec}[1]{\subsection{#1}
\renewcommand{\theequation}{\thesubsection.\arabic{equation}}
  \setcounter{equation}{0}}

\newcommand{\subsubsec}[1]{\subsubsection{#1}
\renewcommand{\theequation}{\thesubsection.\arabic{equation}}
  \setcounter{equation}{0}}

% Danger, Will Robinson!
\def\danger{\begin{trivlist}\item[]\noindent%
\begingroup\hangindent=3pc\hangafter=-2%\clubpenalty=10000%
\def\par{\endgraf\endgroup}%
\hbox to0pt{\hskip-\hangindent\dbend\hfill}\ignorespaces}
\def\enddanger{\par\end{trivlist}}

\newcommand{\Gext}{\negthinspace\negthinspace\phantom{a}^\delta G}
\newcommand{\thetaG}{\negthinspace\negthinspace\phantom{a}^\theta
  G(\C)}
\newcommand{\thetaK}{\negthinspace\negthinspace\phantom{a}^\theta K(\C)}
\newcommand{\qed}{\hfill $\square$ \medskip}
\newenvironment{proof}[1][Proof]{\noindent\textbf{#1.} }{\qed}
\newcommand\exact[3]{1\rightarrow #1\rightarrow #2\rightarrow #3\rightarrow1}
\newcommand{\Aut}{\mathrm{Aut}}
\newcommand{\Inv}{\mathrm{Invol}}
\newcommand{\sgn}{\mathrm{sgn}}
\newcommand{\diag}{\mathrm{diag}}
\newcommand{\gr}{\mathrm{gr}}
\newcommand{\Out}{\mathrm{Out}}
\newcommand{\Int}{\mathrm{Int}}
\renewcommand{\int}{\mathrm{int}}
\newcommand{\Hom}{\mathrm{Hom}}
\newcommand{\kernel}{\mathrm{kernel}}
\newcommand{\Ad}{\mathrm{Ad}}
\newcommand{\ad}{\mathrm{ad}}
\newcommand{\zinv}{\mathrm{inv}}
\newcommand{\SRF}{\mathrm{SRF}}
\newcommand{\Gad}{G_\mathrm{ad}}
\newcommand{\Gsc}{G_\mathrm{sc}}
\newcommand{\Zsc}{Z_\mathrm{sc}}
\newcommand{\Ztor}{Z_\mathrm{tor}}
\newcommand{\Gbar}{\overline G}
\newcommand{\Kad}{K_\mathrm{ad}}
\newcommand{\Gal}{\mathrm{Gal}}
\newcommand{\Norm}{\mathrm{Norm}}
\newcommand{\Cent}{\mathrm{Cent}}
\newcommand{\Stab}{\mathrm{Stab}}
\newcommand{\I}{\mathcal I}
\renewcommand{\O}{\mathcal O}
\newcommand{\R}{\mathbb R}
\newcommand{\C}{\mathbb C}
\newcommand{\Z}{\mathbb Z}
\newcommand{\W}{\mathbb W}
\newcommand{\Ztwo}{\mathbb Z_2}
\newcommand{\N}{\mathcal N}
\newcommand{\Q}{\mathbb Q}
\newcommand{\E}{\mathbb E}
\newcommand{\G}{G}
\renewcommand{\H}{\mathbb H}
\newcommand{\h}{\mathfrak h}
\newcommand{\n}{\mathfrak n}
\renewcommand{\sl}{\mathfrak s\mathfrak l}
\renewcommand{\P}{\mathfrak p}
\renewcommand{\a}{\mathfrak a}
\newcommand{\zk}{\mathfrak z_\mathfrak k}
\newcommand{\A}{\mathbb A}
\newcommand{\K}{\mathcal K}
\newcommand{\B}{\mathcal B}
\renewcommand{\k}{\mathfrak k}
\newcommand{\spint}{\widetilde{Spin}}
\newcommand{\ch}[1]{#1^\vee}
\newcommand\sigmaqc{\sigma_{\text{qc}}}
\newcommand\thetaqc{\theta_{\text{qc}}}
\newcommand{\Fgal}{F_{\text{gal}}}
\newcommand{\Falg}{F_{\text{alg}}}
\newcommand{\cl}{\mathit{cl}}
\newcommand{\Lie}{\mathrm{Lie}}
\newcommand{\opp}{\text{-opp}}

\renewcommand{\t}{\mathfrak t}
\newcommand{\su}{\mathfrak s\mathfrak u}
\newcommand{\g}{\mathfrak g}
\newcommand\inv{^{-1}}
\newcommand\wh{\widehat}
\newcommand{\GL}{\text{GL}}
\newcommand{\SL}{\text{SL}}
\newcommand{\SO}{\text{SO}}
\newcommand{\SU}{\text{SU}}
\newcommand{\Spin}{\text{Spin}}
\newcommand{\chGGamma}{\phantom{a}^\vee G^\Gamma}
\newcommand{\GGamma}{G^\Gamma}
\newcommand{\s}{\mathfrak s}
\newcommand{\w}{\mathfrak w}
\newcommand{\AV}{\mathrm{AV}}
\newcommand{\WF}{\mathrm{WF}}
\newcommand{\AC}{\mathrm{AC}}
\newcommand{\AVann}{\mathrm{AV}_{\mathrm{ann}}}
\newcommand{\GK}{\mathrm{GK}}
\newcommand{\Op}{\O_p}
\newcommand{\Kostant}[1]{\mathcal{K}(#1)}

\begin{document}
\title{Whittaker Models for Real Groups}
\author{Jeffrey Adams \& Alexandre Afgoustidis}
\maketitle

Let $G$ be a connected complex reductive group, defined over $\R$, and suppose $G(\R)$ is quasisplit.


We assume $G(\R)$ has discrete series representations, and fix an L-packet $\Pi$ of discrete series representations.

By a  {\it Whittaker datum} we mean a $G(\R)$-conjugacy class of pairs  $(B(\R),\eta)$ where $B(\R)$ is a
Borel subgroup of $G(\R)$, 
and $\eta$ is a non-degenerate character of the nilpotent radical $N(\R)$ of $B(\R)$. Non-degenerate means: non-trivial on each simple root space.

TODO: write about generic/large representations, and explain what's in the paper...

\section{Some invariants of irreducible $(\g, K)$-modules}



Suppose $\pi$ is an irreducible $(\g,K)$-module. Then we can associate to~$\pi$ several invariants, which depend only on the equivalence class of~$\pi$: the associated variety, the Gelfand---Kirillov dimension, and the wavefront set.  We do not give full definitions, but try to point to convenient places in the literature.\footnote{AA 20-II-2024: of course if we write this, we should maybe add a few basic references for the notions.}

\subsection{Some notation}

Let~$\sigma$ be the anti-holomorphic involution defining~$G(\R)$. 
Let $\g=\Lie(G), \g(\R)=\Lie(G(\R))$. Choose a Cartan involution $\theta$ for $G$. This means:
$\theta$ is an algebraic involution, commuting with~$\sigma$,
and $G(\R)^\theta$ is a maximal compact subgroup of~$G(\R)$. Set $K=G^\theta$, $K(\R)=G(\R)^\theta=K^\sigma$.
Let $\s$ be the $-1$ eigenspace of $\theta$ acting on~$\g$.

Let $\N$ be the set of nilpotent elements of $\g$.
This is the closure of the principal nilpotent orbit $\Op$.

 \subsection{The invariants}


Let $I_\pi$ be the annihlator of $\pi$ in the universal enveloping algebra.
Let $\AVann(\pi)$ be the associated variety of $I_\pi$. 
This is the closure of a single complex nilpotent orbit $\O$.

The associated variety of $\pi$, denoted $\AV(\pi)$, is the  the union of a set of
$K$-orbits on $\O\cap\s$.

The Gelfand---Kirillov dimension~$\GK(\pi)$ of~$\pi$ can be defined in several different ways~\cite{vogan-gelfand-kirillov}; but one of them is
$$
GK(\pi)=\dim(\AV(\pi))=\frac12\dim(\AVann(\pi)).
$$
(All dimensions are complex unless otherwise stated.)

For all of these facts see \cite{vogan_bowdoin}, in particular Corollary 4.7 and Theorem 8.4.

Another invariant of an irreducible $(\g, K)$-module~$\pi$ is the \emph{wave-front set} of $\pi$, written $\WF(\pi)$. It is the union of a (finite) set of nilpotent $G(\R)$ orbits in
$i\g(\R)^*=i\Hom_\R(\g(\R),\R)$. See \cite{howe_wave_front}, \cite{bv_local_structure}.\footnote{AA 14-II-2024: another convenient reference  for definitions re~$\WF(\pi)$ seems to be  Harris--He--Olafsson (Duke 2016).}

\subsection{Large representations} Set $N=\dim(\N)=\dim(G)-\mathrm{rank}(G)$. The maximal Gelfand--Kirillov dimension of a representation is $N/2$.
We say $\pi$ is {\it large} if $\GK(\pi)=N/2$. See \cite[Section~6]{Vogan78}. 


\begin{lemma}
  \label{l:large}
  The following conditions are equivalent.
  \begin{enumerate}
    \item $\pi$ is large;
\item $\AVann(\pi)=\N=\overline{\Op}$;
\item $\dim(\AVann(\pi))=N$;
  \item $\dim(\AV(\pi))=N/2$;
\item $\GK(\pi)=N/2$;

\end{enumerate}
\end{lemma}
See \cite{vogan_bowdoin}.


\section{The dictionary for large discrete series}


\subsection{Statement of the results}  


The following uses rather deep results of representation theory:
\begin{proposition} Let~$\pi$ be the $(\g, K)$-module for a large discrete series representation of~$G(\R)$.  
\begin{enumerate}[(a)]
\item There is a unique Whittaker datum~$\mathfrak{w}$ such that~$\pi$ is $\mathfrak{w}$-generic.
\item $\AV(\pi)$ is the closure of a single $K$-orbit on~$\mathcal{O}_p \cap \mathfrak{s}$.
\item  $\WF(\pi)$ is the closure of a single $G(\R)$-orbit on $\mathcal{O}_p \cap  i \g(\R)$.
\end{enumerate}
\end{proposition}

\begin{proof} For (a), by \cite{vogan-gelfand-kirillov} and \cite{kostant_whittaker}
a representation $\pi$ is large if and only if it admits a Whittaker model for some Whittaker datum,
and by \cite[Lemma 14.14]{abv} a  large discrete series representation admits a unique Whittaker datum.

Part~(b) is one of the main results of~\cite{vogan_bowdoin}: see \cite[Theorem ?]{vogan_bowdoin}.\footnote{AA 19-II-2024: I'd welcome some help in navigating~\cite{vogan_bowdoin}: the relevant places seem to be  Sections 4, 5 and 8, but I have a few questions.}

Part~(c) follows from~\cite{rossmann_limit_orbits}.\footnote{AA 14-II-2024: maybe a little more detail would be welcome, the Rossmann paper does not seem to state the result in these terms.}
\end{proof}

Given a large discrete series representation~$\pi$ we consider the invariants of~$\pi$ defined by the Proposition: a Whittaker datum  $\mathfrak{w}(\pi) \in \mathfrak{W}(G(\R))$, a $K$-orbit $\AV^\flat(\pi) \in (\mathcal{O}_p \cap \mathfrak{s})/K$  and a nilpotent $G(\R)$-orbit $\WF^\flat(\pi) \in (\mathcal{O}_p \cap  i \g(\R))/G(\R)$. 

Here is our main statement.

\begin{theorem} \label{th:main} Suppose~$\Pi$ is an $L$-packet of discrete series for~$G(\R)$. 
The maps $\pi \mapsto \mathfrak{w}(\pi)$, $\pi \mapsto \AV^\flat(\pi)$ and $\pi\mapsto \WF^\flat(\pi)$ induce bijections between:
\begin{enumerate}
\item[(1)] The set~$\Pi_{\mathrm{large}}$ of large discrete series representations in~$\Pi$ ;
\item[(2)] The set of Whittaker data for $G(\R)$ ;
\item[(3)] The set $(\Op \cap \s)/K$ of~$K$-orbits on $\mathcal{O}_p \cap \s$.
\item[(4)] The set $(\Op \cap i\g(\R))/G(\R)$ of~$G(\R)$-orbits on $\mathcal{O}_p \cap  i \g(\R)$.
\end{enumerate}
\end{theorem}

Before embarking on the proof, let us point out that the maps in the theorem are defined using rather deep results of representation theory. However, we shall prove in Section~\ref{sec:explicit} that all bijections $(i) \leftrightarrow (j)$, for $i,j \in \{(1), \dots, (4)\}$, can be spelled out in elementary terms. Furthermore we shall prove, using elementary arguments, that (3) $\leftrightarrow$ (4) coincides with the Kostant--Sekiguchi correspondence.  



Let us now explain the strategy of the proof, which is mainly an exercise in putting together some references that are rather scattered in the literature. What we shall actually do is: introduce a finite group~$Q_\sigma(G)$, and point out that 
\begin{itemize}
\item[(i)] it acts simply transitively on the sets~(1)--(4), and 
\item[(ii)] the maps $\pi \mapsto \mathfrak{w}(\pi)$, $\pi \mapsto \AV^\flat(\pi)$ and $\pi\mapsto \WF^\flat(\pi)$  are equivariant for these actions.
\end{itemize}
The theorem follows immediately from these two observations.\footnote{AA 01-IV-2024 --- perhaps it would be reader-friendly to point out the basic fact on group actions that we are using here: suppose a group $Q$ acts on two sets $A, B$ and let $f\colon A \to B$ be an equivariant map. If $Q$ acts transitively on~$B$  then~$f$ is surjective, and if $Q$ acts transitively on~$A$ and freely on~$B$ then~$f$ is injective.  }

\subsection{The groups $Q_\sigma(G)$ and $Q_{\theta}(G)$}

\subsubsection{Definition of the two groups}Let $Z=Z(G)$ be the center of $G$, and let $\Gad$ be the complex reductive group~$G/Z$. The group $Z(\R)=Z^{\sigma}$ is equal to the center of $G(\R)$. The automorphism~$\sigma$ of~$G$ preserves $Z$, and descends to an automorphism~$\sigma_{\ad}$ of~$\Gad$. We set $\Gad(\R)=\Gad^{\sigma_{\ad}}$; we may view it as the set of inner automorphisms of~$G$ which are defined over~$\R$, and this contains the group $G(\R)_{\mathrm{ad}}=G(\R)/Z(\R)$ as a normal subgroup of finite index\footnote{AA 15-II-2024: is there a trivial way to explain why finite index?}.
Define 
$$
Q_\sigma(G)=\Gad(\R)/G(\R)_{\mathrm{ad}}.
$$
This finite group may be viewed as a group of outer automorphisms of~$G(\R)$. 

We shall need a variant of~$Q_{\sigma}(\R)$ in which we replace the real form~$\sigma$ by the Cartan involution~$\theta$. As above, since~$\theta$ is an automorphism of~$G$, it preserves~$Z$,  and induces an involution~$\theta_{\ad}$ of~$\Gad$. Recall $K=G^{\theta}$; consider the intersection $Z_K=Z \cap K$, and set  $K_{\ad}=K/Z_K$. The inclusion  $K \hookrightarrow G$ induces a map $K \to \Gad$, whose image is a subgroup of~$\Gad^{\theta_{\ad}}$ and whose kernel is~$Z_K$. Therefore it factors to an injective morphism $\iota\colon \Kad \hookrightarrow\Gad^{\theta_{\ad}} $, whose image is a normal subgroup of~$\Gad^{\theta_{\ad}}$. Using it we will identify $\Kad$ with a normal subgroup of $\Gad^{\theta_{\ad}}$. Define
\[ Q_\theta(G) = \Gad^{\theta_{\ad}}/\Kad.\]
Unlike $Q_{\sigma}(G)$, this group does not have an obvious action on~$G(\R)$. Nevertheless the groups $Q_{\sigma}(G)$ and $Q_{\theta}(G)$ are isomorphic. We will spell out an explicit isomorphism below, using group cohomology arguments.

\subsubsection{Interpretation in terms of group cohomology}

The groups $Q_{\sigma}(G)$ and $Q_{\theta}(G)$ are instances of the following construction: if~$\tau$ is an involutive automorphism of~$G$, then it preserves~$Z$ and induces an involution~$\tau_{\ad}$ of~$\Gad$; we may then set $Q_{\tau}(G)=\Gad^{\tau_{\ad}}/(G^{\tau})_{\ad}$, where $(G^{\tau})_{\ad}$ is the image of~$G^{\tau}$ in~$\Gad$. In this situation~$Q_{\tau}(G)$ has a simple interpretation in Galois cohomology, which we now recall. 

Let us first set up some notation. Let~$\Gamma$ be the Galois group~$\Z/2\Z$ of~$\R$. When~$A$ is a group and $\tau$ is an involutive automorphism of~$A$, we may consider the cohomology sets $H^0_\tau(\Gamma, A)$ and $H^1_\tau(\Gamma, A)$ attached to the action of~$\Gamma$ on~$G$ using~$\tau$.  These are pointed sets, and not groups in general if~$A$ isn't abelian. We may view $H^0_\tau(\Gamma, A)$ as the fixed-point-set $A^{\tau}$, and $H^1_\tau(\Gamma,A)$ as the quotient of $A^{-\tau} = \{ a \in A, a\tau(a)=1\}$ by the equivalence relation~$\sim$ generated by $a \sim x a \tau(x^{-1})$ for all $x \in G$. 


We now take~$A=G$, and continue to assume~$\tau$ is an involutive automorphism. Then the short exact sequence $1\rightarrow Z \rightarrow G \rightarrow \Gad\rightarrow 1$ 
gives rise to a ``long" exact sequence of pointed sets:
\begin{equation} \label{long_ptset}
1\rightarrow Z^\tau \rightarrow G^\tau \rightarrow \Gad^{\tau_{\ad}} \rightarrow H^1_\tau(\Gamma,Z)\rightarrow H^1_\tau(\Gamma,G)\rightarrow H^1_{\tau_\ad}(\Gamma,\Gad).
\end{equation}
By definition the connecting map  $ \Gad^{\tau_{\ad}} \rightarrow H^1_\tau(\Gamma,Z)$ sends $g \in \Gad^{\tau_{\ad}}$ to the equivalence class of $g \tau_{\ad}(g^{-1})$ in $H^1_\tau(\Gamma, Z)$, and the meaning of all the other maps is obvious. 
Going through the definitions we see that  two elements  $g_1, g_2$ have the same image if and only if $g_1 g_{2}\inv \in (G^\tau)_{\ad}$. 
Therefore the image of the connecting map is in canonical bijection with $Q_\tau(G)$, and by exactness of~\eqref{long_ptset} we get a canonical bijection
\begin{equation} \label{interp_q_cohomology} Q_{\tau}(G) \simeq \ker(H^1_\tau(\Gamma,Z)\rightarrow H^1_\tau(\Gamma,G)).\end{equation}
The previous argument is in the category of pointed sets, and one should keep in mind on the right-hand side $H^1_\tau(\Gamma,G))$  is not a group in general, therefore~\eqref{interp_q_cohomology} is at first sight only a set-theoretic bijection. However, 
$H^1_\tau(\Gamma,Z)$ is a group since $Z$ is abelian; therefore both sides of~\eqref{interp_q_cohomology} have canonical abelian group structures, and chasing definitions one easily sees that~\eqref{interp_q_cohomology} is actually an isomorphism of abelian groups.
 

\subsubsection{Isomorphism between $Q_{\sigma}(G)$ and $Q_{\theta}(G)$}

Using the group cohomology interpretation in the previous paragraph, we can construct a canonical  isomorphism between $Q_{\sigma}(G)$ and $Q_{\theta}(G)$. For this recall $\theta\sigma = \sigma\theta$ and $G^{\sigma\theta}$ is a compact real form of~$G$. Furthermore $\sigma$ and $\theta$ coincide on $G^{\sigma\theta}$, therefore  $H_{\sigma}(\Gamma, G^{\sigma\theta})$ and  $H_{\sigma}(\Gamma, G^{\sigma\theta})$ are the same set, which we will denote by  $H_{\sigma\mid\theta}(\Gamma, G^{\sigma\theta})$. Now the inclusion $G^{\sigma\theta} \hookrightarrow G$ induces canonical maps 
\begin{equation} \label{bij_cohom} H^{1}_{\sigma}(\Gamma, G) \longleftarrow H_{\sigma\mid\theta}(\Gamma, G^{\sigma\theta}) \longrightarrow H^{1}_{\theta}(\Gamma, G).\end{equation}
By \cite[Corollary~4.4 and Corollary 4.7]{galois}, both of these maps are \emph{bijections}. Replacing~$G$ by $Z$ we get bijections $H^{1}_{\sigma}(\Gamma, Z) \leftarrow H_{\sigma\mid\theta}(\Gamma, Z^{\sigma\theta}) \rightarrow H^{1}_{\theta}(\Gamma, Z)$, and these fit with the bijections~\eqref{bij_cohom} into a commutative diagram
\[
\begin{CD}
H^{1}_{\sigma}(\Gamma, Z) @<<< H_{\sigma\mid\theta}(\Gamma, Z^{\sigma\theta})@>>> H^{1}_{\theta}(\Gamma, Z)
 \\
@V{\varphi_{\sigma}}VV @VVV @V{\varphi_{\theta}}VV  \\
H^{1}_{\sigma}(\Gamma, G) @<<< H_{\sigma\mid\theta}(\Gamma, G^{\sigma\theta})@>>> H^{1}_{\theta}(\Gamma, G)
\end{CD}
\]
where $\varphi_{\sigma}$ and $\varphi_{\theta}$ are induced by the inclusion $Z \hookrightarrow G$. Since the horizontal arrows are isomorphisms in the category of pointed sets, they induce a canonical bijection between the kernels of $\varphi_{\sigma}$ and $\varphi_{\theta}$; by~\eqref{interp_q_cohomology} this gives a canonical bijection 
\[ Q_{\sigma}(G) \simeq Q_{\theta}(G).\] 
As discussed after~\eqref{interp_q_cohomology}, this is actually an isomorphism of abelian groups. 




\subsection{Actions of $Q_\sigma(G)$ on representations and invariants}


\subsubsection*{On representations} 
Since~$Q_\sigma(G)$ can be viewed as a group of automorphisms of~$G(\R)$,  it has a canonical action on equivalence classes of representations of~$G(\R)$, and therefore on~$(\g, K)$-modules. This action preserves  the property of being large\footnote{AA 15-II-2024: should say something about this using the fact that it comes from automorphisms, either here or in the section on large reps}. It also preserves $L$-packets by~\cite[Lemma 6.18]{Contragredient}, and acts transitively on an~$L$-packet\footnote{AA 01-IV-2023: this point was missing, I think it's easy using Harish-Chandra's parametrization, the fact that every element of~$Q_{\sigma}(G)$ has a representative in the Weyl group, plus the description of discrete series L-packets in the Congragredient paper. The point about the action preserving $L$-packets also follows from this, it's perhaps a better argument than [Contragredient, Lemma 6.18] which may be better suited to the $\theta$-version.}

\subsubsection*{On~Whittaker data}  By definition of Whittaker data, the action of~$Q_{\sigma}(G)$ on $G(\R)$ induces a canonical action on the set of Whittaker data. By \cite[(14.15)]{abv} this action is simply transitive. 

It is an immediate consequence of the definitions that the map $\pi \mapsto \w(\pi)$, taking a large discrete series representation to the Whittaker datum for which it has a Whittaker model, is equivariant for the actions of $Q_{\sigma}(G)$ on (1)--(2). 


\subsubsection*{On real nilpotent orbits} 

To define the action of~$Q_{\sigma}(G)$ on~(4), first observe that~$\Gad$ acts on~$\g$ by the adjoint action. This action preserves~$\Op$, and its restriction to~$\Gad(\R)$ preserves~$\Op\cap i\g(\R)$. Since elements on a given~$G(\R)_{\mathrm{ad}}$-orbit are in the same~$G(\R)$-orbit, this induces an action of $Q_{\sigma}(G)=\Gad(\R)/G(\R)_{\mathrm{ad}}$ on~$(\Op\cap i\g(\R))/G(\R)$.

\begin{proposition}\label{prop:action_on_real_orbits}
\begin{enumerate} 
\item The action of~$Q_{\sigma}(G)$ on $(\Op\cap i\g(\R))/G(\R)$ is simply transitive.
\item The map $\pi \mapsto \WF(\pi)$ is $Q_{\sigma}(G)$-equivariant.
\end{enumerate}
\end{proposition}

\begin{proof}
The proof of~(1) is a standard argument in group cohomology. Suppose $\omega,\omega'$ are $G(\R)$ orbits in $\Op\cap i\g(\R)$,
and choose $X\in \omega,X'\in\omega'$. Then there exists $g\in G$ such that $\Ad(g)(X)=X'$. Since $X\in \g(\R)$ the condition $X'\in i\g(\R)$ 
is $g\inv \sigma(g)\in \Stab_G(X)$.  Since $X$ is principal $\Stab_G(X)=ZU$ where $U$ is a unipotent group. 
A standard fact is that $H^1(\sigma,U)=1$, and from this we see that, after multiplying $g$ on the right by an element of $u$, we can assume $g\inv \sigma(g)\in Z$. This is equivalent to: the image of $g$ in $\Gad$ is in $\Gad(\R)$. On the other hand $\omega=\omega'$ 
if and only if $X'=\Ad(g)X$ for some $g\in G(\R)$. This competes the proof of (1).

Part~(2) of the proposition is comes from general properties of the wavefront set in microlocal analysis, and from the fact that the action of $Q_{\sigma}(G)$ comes from automorphisms of~$G(\R)$. More precisely, if~$q$ is an element of~$Q_{\sigma}(G)$ and $\tilde{q}$ is a representative of~$q$ in $\Gad(\R)$, then the action of~$q$ on equivalence classes of $(\g, K)$-modules comes from the action of the automorphism $\mathrm{int}(\tilde{q})$ of~$G(\R)$ on $(\g, K)$-modules. Now the wavefront set of a distribution on a manifold satisfies general covariance properties under diffeomorphisms of the manifold: this follows from Hörmander's original definitions, see e.g. \cite[Section 2, p.~800]{HarrisHeOlafsson}. Applying this to the present situation, and going through the basics as in   \cite[Section~2]{HarrisHeOlafsson}, it immediately follows that $\pi \mapsto \WF(\pi)$ is equivariant under $Q_{\sigma}(G)$.
\end{proof}

It may be interesting to point out a variant on the proof of~(1). Given $X \in \Op \cap i\g(\R)$, the map 
\begin{align} \label{bij_q_2} (\Omega_{X} \cap i\g(\R))/G(\R) & \to \kernel\left(H^1_\sigma(\Gamma,\mathrm{Stab}_G(X))\rightarrow H^1_\sigma(\Gamma,G)\right) \\ \mathrm{Ad}(g) \cdot X & \mapsto \text{class of $ g\sigma(g^{-1})$ in $H^1_\sigma(\Gamma,\mathrm{Stab}_G(X))$} \nonumber \end{align} 
is a bijection: see \cite[Lemma 5.2]{galois}. 
Using this we get bijections
$$
\begin{aligned}
(\Op\cap i\g(\R))/G(\R)&\simeq \ker(H^1_\sigma(\Gamma,\Stab_G(X))\rightarrow H^1_\sigma(\Gamma,G))\\
&=
\ker(H^1_\sigma(\Gamma,Z)\rightarrow H^1_\sigma(\Gamma,G))\\
&\simeq Q_{\sigma}(G)
\end{aligned}
$$
where the middle line uses $H^1_\sigma(\Gamma,\Stab_G(X))=H^1_\sigma(\Gamma,ZU)=H^1_\sigma(\Gamma,Z)$ as explained in the proof. The resulting bijection  $(\Op\cap i\g(\R))/G(\R)\simeq Q_\sigma(\R)$ is equivariant (for the actions of $Q_\sigma(G)$ on $(\Op\cap i\g(\R))/G(\R)$ described above, and the action of  $Q_\sigma(G)$ on itself by multiplication), and this gives another proof of~(1). 

\subsection{Actions of $Q_\theta(G)$ on representations and invariants}

To discuss the map taking a $(\g, K)$-module to the associated variety, it seems better to use $Q_{\theta}(G)$ rather than~$Q_{\sigma}(G)$. 


 \subsubsection*{On $(\g, K)$-modules} 
 
 Recall $Q_{\theta}(\R)=\Gad^{\theta_{\ad}}/\Kad$. By definition $\Gad^{\theta_{\ad}}$ is contained in the normalizer $\mathrm{Norm}_G(K)$; therefore it acts on~$(\g, K)$-modules. Any element of~$\Kad$ takes a $(\g,K)$-module to an equivalent one, therefore $Q_{\theta}(\R)$ acts on~$(\g, K)$-module. This preserves the property of being a discrete series, and of being large. Furthermore, by \cite[Lemma 6.18 and Remark 6.19]{Contragredient} the action preserves $L$-packets. Therefore $Q_{\theta}(\R)$ acts transitively on the large representations in a discrete series $L$-packet.
  
\subsubsection*{On nilpotent $K$-orbits.} 



Let us define an action of~$Q_{\theta}(G)$ on $(\Op \cap \s)/K$. First, the adjoint action of~$\Gad$ on~$\g$ preserves both $\Op$ and~$\s$. Second, the action of~$K=G^\theta$ on $\Op \cap \s$ is trivial on~$Z_K=Z^\theta$, and factors through an action of~$\Kad$. It is then easy to check that if two elements of $\Op \cap \s$ lie in on the same $\Kad$-orbit, then for every $g \in \Gad^{\theta_{\ad}}$, the elements  $\tilde{g} \cdot x$ and $\tilde{g} \cdot y$ lie on the same $\Kad$-orbit. This determines an action of~$Q_\theta(G)$ on $(\Op \cap \s)/K$, which is the one featured in the next statement.

\begin{proposition}\label{prop:action_on_K_orbits}
\begin{enumerate} 
\item The action of~$Q_{\theta}(G)$ on $(\Op \cap \s)/K$ is simply transitive.
\item The map $\pi \mapsto \AV(\pi)$ is $Q_{\theta}(G)$-equivariant.
\end{enumerate}
\end{proposition}

The proof of~(1) is exactly the same as that  to that of Proposition~\ref{prop:action_on_real_orbits}(1), replacing~$\sigma$ by the Cartan involution $\theta$.

As for~(2), it is a consequence of the precise definition of the associated variety. This uses filtrations of the universal enveloping algebra of~$\g$ which are all invariant under $\Gad^{\theta_{\ad}}$: see for instance the Introduction of~\cite{vogan_bowdoin}. Inspecting the definitions in \emph{loc. cit.} it becomes clear that the map $\pi \mapsto \AV(\pi)$, taking a large discrete series $(\g,K)$-module  (or rather its equivalence class) to its associated variety, is equivariant under~$Q_{\theta}(G)$, proving~(2). 
\qed


\section{Explicit versions of the dictionary}\label{sec:explicit}

\subsection{From Whittaker data to real orbits ((2) $\leftrightarrow$ (4))}

First we describe the set of Whittaker data more explicitly.
Let $B$ be a Borel subgroup of $G$. Let $N$ be the nilradical of $B$, and $\overline N$ the opposite
nilpotent subgroup. Let $\n,\overline\n$ be the Lie algebras of $N,\overline N$.
If $B$ is defined over $\R$ we consider $N(\R),\n(\R), \overline\n(\R)$, etc.
Let $\kappa(\,,\,)$  be the Killing form. 
For $X\in i\overline \n(\R)$ define a unitary character $\psi_X$ of $N(\R)$ by:
$$
\psi_X(e^Y)=e^{2\pi \kappa(X,Y)}\quad(Y\in \n(\R)).
$$
The map $X\rightarrow \psi_X$ is an isomorphism between $i\overline\n(\R)$ and the unitary characters of $N(\R)$.
It is easy to see $\psi_X$ is non-degenerate if and only if $X\in i\g(\R)$ is a principal nilpotent element.
In this case we write $(B(\R),X)$ for the Whittaker datum $(B(\R),\psi_X)$.

It is clear that $(B(\R),X)\mapsto G(\R)\cdot X$ is a bijection between Whittaker data and
$(\Op(\R)\cap i\g(\R))/G(\R)$. The main result of \cite{matumoto} (Theorem A) says:

\begin{lemma}
The bijection (2)$\leftrightarrow$(4) takes the Whittaker datum $(B(\R),X)$ to
the principal orbit $G(\R)\cdot X$. 
 \end{lemma} 



\subsection{Connection between ((3) $\leftrightarrow$ (4)) and the Kostant--Sekiguchi correspondence } 


\begin{lemma} The bijection (3)--(4) is induced by the Kostant--Sekiguchi correspondence.
\end{lemma} 

\begin{proof} If we allow ourselves to use deep results, then this is a particular case of the main theorem of~\cite{SV1} (Theorem 1.4). In a perfect world, we could also give a more elementary argument, using the description of the Kostant--Sekiguchi correspondence in~\cite{galois} and the descriptions of the actions above; but  I'm not sure it's possible to do it without encountering a basepoint issue. Another possibility, suggested by Jeff, is to extract it from  arguments in Adams--Kaletha (e.g. the fact that $\WF(\pi)$ is the asymptotic cone of the $G(\R)$-orbit of the HC parameter of~$\pi$, and the connection with the Kostant section). \textcolor{red}{AA (27-III-2024): I think this last strategy should indeed work, but there are details to discuss with Jeff to complete the argument, and if everything turns out OK then  I'll write up the proof here.}\end{proof}

As a consequence we can make the bijection (3) $\leftrightarrow$ (4) explicit, following \cite[Section~1]{avav}.


Suppose $E_\R\in \Op\cap i\g(\R)$. Then we may choose an $SL(2)$-triple $(H_\R,E_\R,F_\R)$ with $F_\R$  contained in $i\g(\R)$,
which implies $H_\R\in \g(\R)$. 
After conjugating by~$G(\R)$ we may further assume $\theta(E_\R)=-F_\R$.
The Sekiguchi correspondence takes the $G(\R)$-orbit of~$E_\R$ to the $K$-orbit of
$$
E_\theta=\frac12(-iE_\R-iF_\R+H_\R)\in \N\cap \s.
$$


Computing the map in the other direction goes as follows.
Suppose $E_\theta\in \Op\cap\s$. Choose an $SL(2)$-triple  $(E_\theta H_\theta,F_\theta)$ with
$F_\theta\in\s$, which implies $H_\theta\in\k$. After conjugating by $K$ we may further assume $\sigma(E_\theta)=F_\theta$.
Then the Sekiguchi correspondence takes the $K$-orbit of $E_\theta$ to the $G(\R)$-orbit of 
$$
E_\R=\frac i2(E_\theta-F_\theta-H_\theta)\in\N\cap i\g(\R).
$$


\subsection{From large discrete series to $K$-orbits ((1) $\leftrightarrow$ (3))}

Suppose $\pi$ is a  discrete series representation. Choose a compact Cartan subgroup $T$, and let
$\lambda\in\t^*$ be the Harish-Chandra parameter of $\pi$. Let $\Delta$ be the set of roots of $\t$ in $\g$, 
let $\Delta^+(\lambda)=\{\alpha\mid \langle\lambda,\ch\alpha\rangle>0\}$,
and let $S(\lambda)\in\Delta^+(\lambda)$ be the simple roots.
For $\alpha\in \Delta$ let $\g_\alpha$ be the corresponding root space, and choose a non-zero element $X_\alpha\in\g_\alpha$ for each $\alpha$.

\begin{lemma}[(1)$\leftrightarrow$(3)]\label{l:pi_to_av}
Suppose $\pi$ is a large discrete series representation with Harish-Chandra parameter $\lambda\in\t^*$.
Set

\begin{equation}
  \label{e:Epi}
  E_\pi=\sum_{\alpha\in S(\lambda)}X_\alpha\in \N.
\end{equation}

Then $E_\pi\in\s$ is a regular nilpotent element, and
$$
\AV(\pi)=\overline{K\cdot E_\pi}.
$$
\end{lemma}

\begin{proof}
If $\pi$ is any discrete series representation, with Harish-Chandra parameter $\lambda$, let
$\n_{\lambda}=\sum_{\alpha\in\Delta^+(\lambda)}\g_\alpha$.
Then by \cite[Proposition 6.8]{vogan_irreducibility} $AV(\pi)=K\cdot(\n_\lambda\cap\s)$. 

Now assume $\pi$ is large. This implies $X_\alpha\in\s$ for all $\alpha\in S(\lambda)$, so $E_\pi\in \s$. 
By~\cite{kostant_tds} $E_\pi$ is a regular nilpotent element, so the closure of $K\cdot E_\pi$ in $\s$ is equal to $K\cdot(\n_\lambda\cap\s)$. 
\end{proof}




\subsection{From large discrete series to real orbits ((1) $\leftrightarrow$ (4))}


There are several ways to describe this bijection.\footnote{AA 23-II-2024: I think we should mention the version with the asymptotic cone of the orbit of~$H_\pi$, which is extremely clean and concrete. } We choose to compose the maps (1)$\leftrightarrow$(3)$\leftrightarrow$(4).
Here is the conclusion.

\begin{lemma}
  Let $\pi$ be a large discrete series representation.
Choose a compact Cartan subgroup $T$ and let $\lambda\in \t^*$ be the Harish-Chandra parameter of $\pi$.
Define $E_\theta=E_\pi$ as in \eqref{e:Epi}, and choose an $SL(2)$-triple $(H_\theta,E_\theta,F_\theta)$ as usual.
After conjugating by $K$ we may assume $\sigma(E_\theta)=F_\theta$.  Then\footnote{AA 14-II-2023: no closure?}
$$
\WF(\pi)=G(\R)\cdot\frac i2(E_\theta-F_\theta-H_\theta)\in\N\cap i\g(\R)
$$
Conversely if $E_\R\in \Op\cap i\g(\R)$, choose an $SL(2)$-triple $(H_\R,E_\R,F_\R)$ which (after conjugating by $G(\R)$) satisfies $\theta(E_\R)=-F_\R$.
Let
$$
H_\theta=E_\R-F_\R.
$$
Then $H_\theta\in\k$ is a regular semisimple element.
Set $\t=\Cent_{\g}(H_\theta)$, and let
$$
\Delta^+=\{\alpha\in\Delta(T,G)\mid  \alpha(H_\theta)>0\}
$$
Then $\pi$ is the discrete series representation in $\Pi$ whose Harish-Chandra parameter is dominant for $\Delta^+$. 
\end{lemma}

We can also express (1)$\mapsto$(4) using the {\it Kostant section}. Recall that if~$X$ is a regular nilpotent element of~$\g$ and fits into an $\SL(2)$-triple $(X, H, Y)$, the Kostant section~$\Kostant{X}$ is the affine subspace $X + \mathrm{Cent}_{\g}(Y)$ of~$\g$. It depends on the choice of $\SL(2)$-triple, but if $X \in \g(\R)$, then the $G(\R)$-conjugacy class of~$\Kostant{X}$ depends only on the $G(\R)$-conjugacy class of~$X$. 

Suppose $\pi$ is a discrete series representation.
Choose a compact Cartan subgroup $T$, and let $\lambda\in\t^*$ be the Harish-Chandra parameter of $\pi$.
Identify $\lambda$ with an element $H_\pi\in i\t\subset i\g$ using the isomorphism $\t\simeq \t^*$ induced by the Killing form. 

\begin{proposition}[\cite{adams_kaletha}]
Suppose $\pi$ is a generic discrete series representation. Then $\pi$ is $\w$-generic
for a unique Whittaker datum $\w$.
Write $\w=\w_X$ for some regular nilpotent element $X\in i\g(\R)$.
Then $H_\pi$ is $G(\R)$-conjugate to an element of the Kostant section of $X$.\footnote{AA 23-II-2024: would probably be useful to explain the connection with $\pi \mapsto \WF(\pi)$ in more detail...\\ AA 27-III-2024: furthermore, if the Kostant section is used in the proof that $\AV \leftrightarrow \WF$ coincides with Sekiguchi, then we should move this subsection to an earlier place, and then probably this discussion will have to be reworked.  }
\end{proposition}


This confirms that  the corresponding statement  in the $p$-adic case \cite{debacker_reeder_generic, kaletha_epipelagic}
applies to the real case as well.

\bibliographystyle{plain}
\bibliography{Refs}

\end{document}
%%% Local Variables: 
%%% mode: latex
%%% TeX-master: t
%%% End: s_{1}:0 -> -1 6


