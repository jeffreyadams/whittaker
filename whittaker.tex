%edited starting 8/27/17 for re-submission to Duke
\documentclass[10pt,leqno]{article}
\usepackage{verbatim}
\usepackage{amssymb}
\usepackage{mathtools}
%\usepackage{amsrefs}
\usepackage{rotating}
\usepackage{amsmath}
%\usepackage{showkeys}
\usepackage{tabularx}
\setlength\extrarowheight{4pt}   %spacing in tables
\usepackage{theorem}
\usepackage[matrix,tips,frame,color,line,poly,curve]{xy}
\renewcommand{\labelenumi}{(\arabic{enumi})}
\newcommand\kappaarrow[2]{#1\overset\kappa\rightarrow#2}
\newtheorem{theorem}[equation]{Theorem}
\newtheorem{corollary}[equation]{Corollary}
\newtheorem{definition}[equation]{Definition}
\newtheorem{lemma}[equation]{Lemma}
\newtheorem{desideratum}[equation]{Desideratum}
\newtheorem{conjecture}[equation]{Conjecture}
\newtheorem{proposition}[equation]{Proposition}
\newtheorem{remark}[equation]{Remark}
{\theorembodyfont{\rmfamily}\newtheorem{theoremplain}[equation]{Theorem}
\newtheorem{remarkplain}[equation]{Remark}
\newtheorem{editorialremarkplain}[equation]{Editorial Remark}
\newtheorem{exampleplain}[equation]{Example}
\newtheorem{corollaryplain}[equation]{Corollary}
\newtheorem{mytable}[equation]{Table}
}

\renewcommand{\sec}[1]{\section{#1}
\renewcommand{\theequation}{\thesection.\arabic{equation}}
  \setcounter{equation}{0}}
\newcommand{\subsec}[1]{\subsection{#1}
\renewcommand{\theequation}{\thesubsection.\arabic{equation}}
  \setcounter{equation}{0}}

\newcommand{\subsubsec}[1]{\subsubsection{#1}
\renewcommand{\theequation}{\thesubsection.\arabic{equation}}
  \setcounter{equation}{0}}

% Danger, Will Robinson!
\def\danger{\begin{trivlist}\item[]\noindent%
\begingroup\hangindent=3pc\hangafter=-2%\clubpenalty=10000%
\def\par{\endgraf\endgroup}%
\hbox to0pt{\hskip-\hangindent\dbend\hfill}\ignorespaces}
\def\enddanger{\par\end{trivlist}}

\newcommand{\Gext}{\negthinspace\negthinspace\phantom{a}^\delta G}
\newcommand{\thetaG}{\negthinspace\negthinspace\phantom{a}^\theta
  G(\C)}
\newcommand{\thetaK}{\negthinspace\negthinspace\phantom{a}^\theta K(\C)}
\newcommand{\qed}{\hfill $\square$ \medskip}
\newenvironment{proof}[1][Proof]{\noindent\textbf{#1.} }{\qed}
\newcommand\exact[3]{1\rightarrow #1\rightarrow #2\rightarrow #3\rightarrow1}
\newcommand{\Aut}{\mathrm{Aut}}
\newcommand{\Inv}{\mathrm{Invol}}
\newcommand{\sgn}{\mathrm{sgn}}
\newcommand{\diag}{\mathrm{diag}}
\newcommand{\gr}{\mathrm{gr}}
\newcommand{\Out}{\mathrm{Out}}
\newcommand{\Int}{\mathrm{Int}}
\renewcommand{\int}{\mathrm{int}}
\newcommand{\Hom}{\mathrm{Hom}}
\newcommand{\kernel}{\mathrm{kernel}}
\newcommand{\Ad}{\mathrm{Ad}}
\newcommand{\zinv}{\mathrm{inv}}
\newcommand{\SRF}{\mathrm{SRF}}
\newcommand{\Gad}{G_\mathrm{ad}}
\newcommand{\Gsc}{G_\mathrm{sc}}
\newcommand{\Zsc}{Z_\mathrm{sc}}
\newcommand{\Ztor}{Z_\mathrm{tor}}
\newcommand{\Gbar}{\overline G}
\newcommand{\Kad}{K_\mathrm{ad}}
\newcommand{\Gal}{\mathrm{Gal}}
\newcommand{\Norm}{\mathrm{Norm}}
\newcommand{\Cent}{\mathrm{Cent}}
\newcommand{\Stab}{\mathrm{Stab}}
\newcommand{\I}{\mathcal I}
\renewcommand{\O}{\mathcal O}
\newcommand{\R}{\mathbb R}
\newcommand{\C}{\mathbb C}
\newcommand{\Z}{\mathbb Z}
\newcommand{\W}{\mathbb W}
\newcommand{\Ztwo}{\mathbb Z_2}
\newcommand{\N}{\mathcal N}
\newcommand{\Q}{\mathbb Q}
\newcommand{\E}{\mathbb E}
\newcommand{\G}{G}
\renewcommand{\H}{\mathbb H}
\newcommand{\h}{\mathfrak h}
\newcommand{\n}{\mathfrak n}
\renewcommand{\sl}{\mathfrak s\mathfrak l}
\renewcommand{\P}{\mathfrak p}
\renewcommand{\a}{\mathfrak a}
\newcommand{\zk}{\mathfrak z_\mathfrak k}
\newcommand{\A}{\mathbb A}
\newcommand{\K}{\mathcal K}
\newcommand{\B}{\mathcal B}
\renewcommand{\k}{\mathfrak k}
\newcommand{\spint}{\widetilde{Spin}}
\newcommand{\ch}[1]{#1^\vee}
\newcommand\sigmaqc{\sigma_{\text{qc}}}
\newcommand\thetaqc{\theta_{\text{qc}}}
\newcommand{\Fgal}{F_{\text{gal}}}
\newcommand{\Falg}{F_{\text{alg}}}
\newcommand{\cl}{\mathit{cl}}
\newcommand{\Lie}{\mathrm{Lie}}
\newcommand{\opp}{\text{-opp}}

\renewcommand{\t}{\mathfrak t}
\newcommand{\su}{\mathfrak s\mathfrak u}
\newcommand{\g}{\mathfrak g}
\newcommand\inv{^{-1}}
\newcommand\wh{\widehat}
\newcommand{\GL}{\text{GL}}
\newcommand{\SL}{\text{SL}}
\newcommand{\SO}{\text{SO}}
\newcommand{\SU}{\text{SU}}
\newcommand{\Spin}{\text{Spin}}
\newcommand{\chGGamma}{\phantom{a}^\vee G^\Gamma}
\newcommand{\GGamma}{G^\Gamma}
\newcommand{\s}{\mathfrak s}
\newcommand{\w}{\mathfrak w}
\newcommand{\AV}{\mathrm{AV}}
\newcommand{\WF}{\mathrm{WF}}
\newcommand{\AVann}{\mathrm{AV}_{\mathrm{ann}}}
\newcommand{\GK}{\mathrm{GK}}
\newcommand{\Op}{\O_p}
\begin{document}
\title{Whittaker Models for Real Groups}
\author{Jeffrey Adams}
\maketitle

Let $G$ be a connected, complex reductive group, defined over $\R$, and suppose $G(\R)$ is quasisplit.
We fix some notation.

Let $\g=\Lie(G), \g(\R)=\Lie(G(\R))$. Choose a Cartan involution $\theta$ for $G$. This means:
$\theta$ is an algebraic involution, commuting with the anti-holomorphic involution $\sigma$ defining $G(\R)$,
and $G(\R)^\theta$ is a maximal compact subgroup of $G(\R)$. Set $K=G^\theta$, $K(\R)=G(\R)^\theta=K^\sigma$.
Let $\s$ be the $-1$ eigenspace of $\theta$ acting on $\g$.

We assume $G(\R)$ has discrete series representations, and fix an L-packet $\Pi$ of discrete series representations.

By a  {\it Whittaker datum} we mean a $G(\R)$-conjugacy class of pairs  $(B(\R),\eta)$ where $B(\R)$ is a
Borel subgroup of $G(\R)$, 
and $\eta$ is a non-degenerate character of the nilpotent radical $N(\R)$ of $B(\R)$. Non-degenerate means: non-trivial on each simple root space.


Let $\N$ be the set of nilpotent elements of $G$ on $\g$.
This is the closure of the principal nilpotent orbit $\Op$.

Suppose $\pi$ is an irreducible $(\g,K)$-module. Let $I_\pi$ be the annihlator of $\pi$ in the universal enveloping algebra.
Let $\AVann(\pi)$ be the associated variety of $I_\pi$. 
This is the closure of a single complex nilpotent orbit $\O$.
Write $\GK(\pi)$ for the Gelfand-Kirillov dimension of $\pi$
\cite{vogan-gelfand-kirillov}. 
All dimensions are complex unless otherwise stated.
The associated variety of $\pi$, denoted $\AV(\pi)$, is the  the union of a set of
$K$-orbits on $\O\cap\s$.
Then
$$
GK(\pi)=\dim(AV(\pi))=\frac12\dim(\AVann(\pi)
$$
For all of these facts see \cite{vogan_bowdoin}, in particular Corollary 4.7 and Theorem 8.4.

Set $N=\dim(\N)=\dim(G)-\mathrm{rank}(G)$. The maximal Gelfand-Kirillov dimension of a representation is $N/2$.
We say $\pi$ is {\it large} if $\GK(\pi)=N/2$. See \cite[Section 6]{Vogan78}.


The wave-front set of $\pi$, written $\WF(\pi)$, is the union of a (finite) set of nilpotent $G(\R)$ orbits in
$i\g(\R)^*=i\Hom_\R(\g(\R),\R)$. See \cite{howe_wave_front}, \cite{bv_local_structure}.

\begin{lemma}
  \label{l:large}
  The following conditions are equivalent.
  \begin{enumerate}
    \item $\pi$ is large;
\item $\AVann(\pi)=\N=\overline{\Op}$;
\item $\dim(\AVann(\pi))=N$;
  \item $\dim(AV(\pi))=N/2$;
\item $\GK(\pi)=N/2$;

\end{enumerate}
\end{lemma}
See \cite{vogan_bowdoin}.


Let $Z=Z(G)$ be the center of $G$, and $\Gad=G/Z$.
Then $Z$ and $\Gad$ are defined over $\R$, and $Z(\R)=Z(G(\R))$.
Set $G(\R)_{\mathrm{ad}}=G(\R)/Z(\R)$.
Define
$$
Q(\R)=\Gad(\R)/G(\R)_{\mathrm{ad}}
$$
Thus $\Gad(\R)$ is the set of inner automorphisms of $G$ which are defined over $\R$, containing
$G(\R)_{\mathrm{ad}}=G(\R)/Z(\R)$ as a normal subgroup. This is a subgroup of $\Out(G(\R))$,
and therefore acts on representations, conjugacy classes, etc.


\begin{lemma}
  \label{l:Q}
  $Q(\R)\simeq\kernel(H^1(\Gamma,Z)\rightarrow H^1(\Gamma,G))$
\end{lemma}

This is well known:
the exact sequence $1\rightarrow Z \rightarrow G \rightarrow \Gad\rightarrow 1$
gives rise to the long exact sequence:
$$
1\rightarrow Z(\R) \rightarrow G(\R) \rightarrow \Gad(\R) \rightarrow H^1(\Gamma,Z)\rightarrow H^1(\Gamma,G)\rightarrow\dots
$$.
\section{The Main Result}

We're in the setting of the previous section:  $G(\R)$ is quasisplit, and has discrete series representations.

\begin{theorem}
  \label{t:main}
There are canonical bijections between the following sets.

\begin{enumerate}
\item Large discrete series representations in $\Pi$
\item Whittaker data for $G(\R)$
\item $(\Op\cap \s)/K$
\item  $(\Op\cap \g(\R))/G(\R)$
\end{enumerate}

The group $Q(\R)$ acts simply transitively on each of these sets, and the bijections commute with this action.
\end{theorem}

This is mostly an exercise in looking up some references.

\medskip

\begin{proof}
Suppose $\pi$ is a large discrete series representation.
We will show:
\begin{enumerate}
\item $\pi$ admits a Whittaker model for a unique choice of Whittaker datum;
\item $AV(\pi)$ is the closure of a $K$-orbit on $\Op\cap\s$;
\item $WF(\pi)$ is the closure of a $G(\R)$-orbit on $\Op\cap i\g(\R)$.
\end{enumerate}
This defines the maps from (1) to (2,3,4). The fact these are all bijections will follow from the action of $Q(\R)$.

By \cite{Vogan78} and \cite{kostant_whittaker}
a representation $\pi$ is large if and only if it admits a Whittaker model for some Whittaker datum,
and by \cite[Lemma 14.14]{abv} a  large discrete series representation admits a unique Whittaker datum.
This defines the map (1)$\mapsto$(2).

Suppose $\pi$ is a large irreducible representation.
By Lemma \ref{l:large} $\AV(\pi)$ is the union of a set of $K$-orbits on $\Op\cap\s$.
If $\pi$ is a discrete series representation $AV(\pi)$ is the closure of a single $K$-orbit on $\s$.
This defines the map (1)$\mapsto$(3).

By \cite{rossmann_limit_orbits}, $\WF(\pi)$ is the closure of a single orbit
so $\pi\mapsto WF(\pi)$ gives  (1)$\mapsto$(4).

The group $Q(\R)$ acts on  $(\g,K)$-modules, and preserves $\Pi$ and the property of being large.
Therefore, by \cite[(14.15)(d)]{abv}, it acts simply transitively on the set (1).
By \cite[(14.15)(c)]{abv} $Q(\R)$ acts simply transitively on the set (2).

It is well known that if $X\in\g$ is nilpotent then  $G\cdot X\cap \g(\R)/G(\R)$ is in bijection with
$\kernel(H^1(\Gamma,\Stab_G(X)) \rightarrow H^1(\Gamma,G))$. For example see \cite[Lemma 5.2]{galois}.
In the case of the principal nilpotent orbit $\Op$, $\Stab_G(X)$ is $Z(G)$ times a unipotent group,
which implies $H^1(\Gamma,\Stab_G(X))=H^1(\Gamma,Z)$. Therefore by Lemma \ref{l:Q} $Q(\R)$ acts simply transitively
on (4).
Finally the Kostant-Sekiguchi correspondence \cite{sekiguchi} is a bijection between (3) and (4).

Therefore $Q(\R)$ acts on the sets (1-4), and simply transitively on (1). The maps clearly commute with this action. This completes the proof.
\end{proof}

There is one obvious shortcoming of this result: we'd like to know that the Sekiguchi correspondence takes $\AV(\pi)$ to $\WF(\pi)$.
This is true by \cite{SV1} (and also follows in this special case by more elementary arguments). More generally we would like to describe the maps between the various sets more explicitly.

\section{Further results}

We describe maps between the sets (1-4) of Theorem \ref{t:main}.

\medskip

\noindent (2)$\leftrightarrow$(4)

First we describe the set of Whittaker data more explicitly.
Let $B$ be a Borel subgroup of $G$. Let $N$ be the nilradical of $B$, and $\overline N$ the opposite
nilpotent subgroup. Let $\n,\overline\n$ be the Lie algebras of $N,\overline N$.
If $B$ is defined over $\R$ we consider $N(\R),\n(\R), \overline\n(\R)$, etc.
Let $\kappa(\,,\,)$  be the Killing form. 
For $X\in i\overline \n(\R)$ define a unitary character $\psi_X$ of $N(\R)$ by:
$$
\psi_X(e^Y)=e^{2\pi \kappa(X,Y)}\quad(Y\in \n(\R)).
$$
The map $X\rightarrow \psi_X$ is an isomorphism between $i\overline\n(\R)$ and the unitary characters of $N(\R)$.
It is easy to see $\psi_X$ is non-degenerate if and only if $X\in i\g(\R)$ is a principal nilpotent element.
In this case we write $(B(\R),X)$for the Whittaker datum $(B(\R),\psi_X)$.
It is clear that $(B(\R),X)\mapsto G(\R)\cdot X$ is a bijection between Whittaker data and
$(\Op(\R)\cap i\g(\R))/G(\R)$.

The main result of \cite{matumoto} (Theorem A) says:

\begin{lemma}[(2)$\leftrightarrow$(4)]  
The bijection (2)$\leftrightarrow$(4) takes the Whittaker datum $(B(\R),X)$ to
the principal orbit $G(\R)\cdot X\subset \Op\cap i\g(\R)$. 
 \end{lemma} 


\medskip

\noindent (1)$\leftrightarrow$(3)
Suppose $\pi(\lambda)$ is a  discrete series representation. Choose a compact Cartan subgroup $T$, and let
$\lambda\in\t^*$ be the Harish-Chandra parameter of $\pi$. Let $\Delta$ be the set of roots of $\t$ in $\g$, 
let $\Delta^+(\lambda)=\{\alpha\mid \langle\lambda,\ch\alpha\rangle>0\}$,
and let $S(\lambda)\in\Delta^+(\lambda)$ be the simple roots.
For $\alpha\in \Delta$ let $\g_\alpha$ be the corresponding root space, and choose a non-zero element $X_\alpha\in\g_\alpha$ for each $\alpha$.

\begin{lemma}[(1)$\leftrightarrow$(3)]
Suppose $\pi$ is a large discrete series representation with Harish-Chandra parameter $\lambda\in\t^*$.
Set

\begin{equation}
  \label{e:Epi}
  E_\pi=\sum_{\alpha\in S(\lambda)}X_\alpha\in \N.
\end{equation}

Then $E_\pi\in\s$ is a regular nilpotent element, and
$$
\AV(\pi)=\overline{K\cdot E_\pi}.
$$
\end{lemma}

\begin{proof}
If $\pi$ is any discrete series representation, with Harish-Chandra parameter $\lambda$, let
$\n=\sum_{\alpha\in\Delta^+(\lambda)}\g_\alpha$.
Then by \cite[Proposition 6.8]{vogan_irreducibility} $AV(\pi)=K\cdot(\n\cap\s)$. 

Now assume $\pi$ is large. This implies $X_\alpha\in\s$ for all $\alpha\in S(\lambda)$, so $E_\pi\in \s$. 
By \cite{kostant_tds} $E_\pi$ is a regular nilpotent element, so the closure of $K\cdot E_\pi$ in $\s$ is $K\cdot(\n\cap\s)$. 
\end{proof}

\medskip

\noindent (3)$\leftrightarrow$(4)

This is the Kostant-Sekiguchi correspondence. We make this explicit, following \cite[Section 1]{avav}.

Suppose $E_\R\in \Op\cap i\g(\R)$. Then we may choose an $SL(2)$-triple $(H_\R,E_\R,F_\R)$ with $F_\R$  contained in $i\g(\R)$,
which implies $H_\R\in \g(\R)$. 
After conjugating by $G(\R)$ we may further assume $\theta(E_\R)=-F_\R$.
The Sekiguchi correspondence takes
$$
\Op\cap i\g(\R)\ni E_\R\mapsto E_\theta=\frac12(-iE_\R-iF_\R+H_\R)\in \N\cap \s.
$$


Computing the map in the other direction goes as follows.
$E_\theta\in \Op\cap\s$, choose $(H_\theta,E_\theta,F_\theta)$, with
$F_\theta\in\s$, which implies $H_\theta\in\k$. After conjugating by $K$ we may further assume $\sigma(E_\theta)=F_\theta$.
The correspondence takes
$$
\Op\cap\s\ni E_\theta\mapsto E_\R=\frac i2(E_\theta-F_\theta-H_\theta)\in\N\cap i\g(\R).
$$

\medskip

\noindent (1)$\leftrightarrow$(4)

There are several ways to describe this bijection. We choose to compose the maps (1)$\leftrightarrow$(3)$\leftrightarrow$(4).
Here is the conclusion.

\begin{lemma}
  Let $\pi$ be a large discrete series representation.
Choose a compact Cartan subgroup $T$ and let $\lambda\in \t^*$ be the Harish-Chandra parameter of $\pi$.
Define $E_\theta=E_\pi$ as in \eqref{e:Epi}, and choose an $SL(2)$-triple $(H_\theta,E_\theta,F_\theta)$ as usual.
After conjugating by $K$ we may assume $\sigma(E_\theta)=F_\theta$.  Then
$$
\WF(\pi)=G(\R)\cdot\frac i2(E_\theta-F_\theta-H_\theta)\in\N\cap i\g(\R)
$$
Conversely if $E_\R\in \Op\cap i\g(\R)$, choose an $SL(2)$-triple $(H_\R,E_\R,F_\R)$ which (after conjugating by $G(\R)$) satisfies $\theta(E_\R)=-F_\R$.
Let
$$
H_\theta=E_\R-F_\R.
$$
Then $H_\theta\in\k$ is a regular semisimple element.
Set $\t=\Cent_{\g}(H_\theta)$, and let
$$
\Delta^+=\{\alpha\in\Delta(T,G)\mid  \alpha(H_\theta)>0\}
$$
Then $\pi$ is the discrete series representation in $\Pi$ whose Harish-Chandra parameter is dominant for $\Delta^+$. 
\end{lemma}

We can also express (1)$\mapsto$(4) using the {\it Kostant section}.

Suppose $\pi$ is a discrete series representation.
Choose a compact Cartan subgroup $T$, and let $\lambda\in\t^*$ be the Harish-Chandra parameter of $\pi$.
Identify $\lambda$ with an element $H_\pi\in i\t\subset i\g$ using the isomorphism $\t\simeq \t^*$ induced by the Killing form. 

\begin{proposition}[\cite{adams_kaletha}]
Suppose $\pi$ is a generic discrete series representation. Then $\pi$ is $\w$-generic
for a unique Whittaker datum $\w$.
Write $\w=\w_X$ for some regular nilpotent element $X\in i\g(\R)$.
Then $H_\pi$ is $G(\R)$-conjugate to an element of the Kostant section of $X$.
\end{proposition}


This confirms that  the corresponding statement  in the p-adic case \cite{debacker_reeder_generic} and \cite{kaletha_epipelagic}
applies to the real case as well.
\bibliographystyle{plain}
\bibliography{Refs}

\end{document}
%%% Local Variables: 
%%% mode: latex
%%% TeX-master: t
%%% End: s_{1}:0 -> -1 6


